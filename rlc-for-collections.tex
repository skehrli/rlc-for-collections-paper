%%% Some sections of this file are only needed for papers in ACM-sponsored
%%% conferences, not IEEE-sponsored conferences (such as ICSE, every two
%%% years).
%%%
%%% You may be better off using most of this paper template repository, but
%%% mostly replacing this main file by that in the conference's author kit
%%% (copying over the content in this template that is specific to your
%%% paper, such as the \input and \bibliography directives, etc.).

\documentclass[acmsmall,screen,review,anonymous]{acmart}
\settopmatter{printfolios=true,printccs=false,printacmref=false}

%%
%% \BibTeX command to typeset BibTeX logo in the docs
\AtBeginDocument{%
  \providecommand\BibTeX{{%
    \normalfont B\kern-0.5em{\scshape i\kern-0.25em b}\kern-0.8em\TeX}}}

%% Rights management information.  This information is sent to you
%% when you complete the rights form.  These commands have SAMPLE
%% values in them; it is your responsibility as an author to replace
%% the commands and values with those provided to you when you
%% complete the rights form.
% \setcopyright{acmcopyright}
% \copyrightyear{2018}
% \acmYear{2018}
% \acmDOI{10.1145/1122445.1122456}

%% These commands are for a PROCEEDINGS abstract or paper.
% \acmConference[Woodstock '18]{Woodstock '18: ACM Symposium on Neural
%   Gaze Detection}{June 03--05, 2018}{Woodstock, NY}
% \acmBooktitle{Woodstock '18: ACM Symposium on Neural Gaze Detection,
%   June 03--05, 2018, Woodstock, NY}
% \acmPrice{15.00}
% \acmISBN{978-1-4503-9999-9/18/06}
\acmConference[Conference \the\year]{The 1st Conference on Conferences}{1 - 4 January, \the\year}{City, Country}


%%
%% Submission ID.
%% Use this when submitting an article to a sponsored event. You'll
%% receive a unique submission ID from the organizers
%% of the event, and this ID should be used as the parameter to this command.
%%\acmSubmissionID{123-A56-BU3}

%%
%% The majority of ACM publications use numbered citations and
%% references.  The command \citestyle{authoryear} switches to the
%% "author year" style.
%%
%% If you are preparing content for an event
%% sponsored by ACM SIGGRAPH, you must use the "author year" style of
%% citations and references.
%% Uncommenting
%% the next command will enable that style.
%%\citestyle{acmauthoryear}


% tables
\usepackage{tabularx}
\usepackage{booktabs}

%\usepackage{algorithm}
%\usepackage[noend]{algpseudocode}

% xspace command
\usepackage{xspace}

% for the researchquestions environment
\usepackage{enumitem}

% From https://tex.stackexchange.com/questions/177025/
%% \makeatletter
%% \newcounter{algorithmicH}% New algorithmic-like hyperref counter
%% \let\oldalgorithmic\algorithmic
%% \renewcommand{\algorithmic}{%
%%   \stepcounter{algorithmicH}% Step counter
%%   \oldalgorithmic}% Do what was always done with algorithmic environment
%% \renewcommand{\theHALG@line}{ALG@line.\thealgorithmicH.\arabic{ALG@line}}
%% \makeatother

% lstlisting command
\usepackage{listings}
\usepackage[scaled]{beramono}
\newcommand*\LSTfont{\Small\fontencoding{T1}\ttfamily\SetTracking{encoding=*}{-60}\lsstyle}
\lstset{language=Java,
  frame=none,
  aboveskip=1.5pt,
  belowskip=0pt,
  showstringspaces=false,
  columns=flexible,
  basicstyle=\LSTfont,
  numbers=none,
  numberstyle=\tiny\color{black},
  keywordstyle=\color{black},
  commentstyle=\color{black},
  stringstyle=\color{black},
  breaklines=true,
  breakatwhitespace=true,
  tabsize=3,
  %emph={@NonNegative,@Positive,@GTENegativeOne,@LTLengthOf,@LTEqLengthOf,@IndexFor,@IndexOrHigh,@IndexOrLow,@HasSubsequence,@LessThan,@SameLen,@SearchIndexFor,@MinLen,@ArrayLen,@IntVal,@IntRange,@LengthOf,@UpperBoundUnknown,@LowerBoundUnknown,int,double,List,Map,Object,SerialDate,Long,Integer,DefaultPolarItemRenderer,LegendItem,PolarPlot,XYDataset,long,T,String,string,byte,InputStream,CategoryDataset,DatasetRenderingOrder,ArrayList,Entry,Values,Number,ValuesContract,ImmutableIntArray,Dataset,XYZDataset}, emphstyle=\color{blue}
}

% Graphics
% \usepackage{tikz}
% \usetikzlibrary{arrows,automata,positioning}

% Change font and line spacing for figure captions
\usepackage{setspace,caption}
\captionsetup{labelfont={small,bf}, textfont={small,bf,stretch=0.8}, labelsep=colon, margin=0pt}

\usepackage{flushend} % balanced columns on last page

\usepackage{url} % URLs; used in plume-bib

% cref command; best to load last
\usepackage{cleveref}
\newcommand{\crefrangeconjunction}{--}

%%
%% end of the preamble, start of the body of the document source.
\begin{document}

%%% Todo comments
%% Comment out one of these two definitions.
%\newcommand{\todo}[1]{\relax}
\newcommand{\todo}[1]{{\color{red}\bfseries [[#1]]}}
\newcommand{\manu}[1]{\todo{#1 --MS}}

% Don't show todo commands if this macro is defined.
\ifdefined\notodocomments
  \renewcommand{\todo}[1]{\relax}
\fi

\newif\ifanonymous
\anonymoustrue

\newcommand{\anonurl}[1]{\ifanonymous URL removed for anonymity.\else\url{#1}\fi}
\newcommand{\footnoteanonurl}[1]{\footnote{\anonurl{#1}}}

%% The anonymized name of the tool
%% TODO: come up with a better name.
\newcommand{\toolanon}{Plumber\xspace}
%% The name of the tool to use
\newcommand{\Tool}{\toolanon}
%% TODO: remove this.  (The tool name is capitalized and the LaTeX source
%% reads best when the macro is also capitalized.)
\newcommand{\tool}{\toolanon}

% \|name| or \mathid{name} denotes identifiers and slots in formulas
\def\|#1|{\mathid{#1}}
\newcommand{\mathid}[1]{\ensuremath{\mathit{#1}}}
% \<name> or \codeid{name} denotes computer code identifiers
\def\<#1>{\codeid{#1}}
% \protected\def\codeid#1{\ifmmode{\mbox{\sf{#1}}}\else{\sf #1}\fi}
% \protected\def\codeid#1{\ifmmode{\mbox{\ttfamily{#1}}}\else{\ttfamily #1}\fi}
\protected\def\codeid#1{\ifmmode{\mbox{\smaller\ttfamily{#1}}}\else{\smaller\ttfamily #1}\fi}

\newcommand{\CalledMethodsBottom}{\<@Call\-ed\-Meth\-ods\-Bottom>\xspace}
\newcommand{\EnsuresCalledMethods}{\<@En\-sures\-Call\-ed\-Meth\-ods>\xspace}
\newcommand{\MustCall}{\codeid{@Must\-Call}\xspace}
\newcommand{\MustCallAlias}{\codeid{@Must\-Call\-Alias}\xspace}
\newcommand{\MustCallUnknown}{\codeid{@Must\-Call\-Unknown}\xspace}
\newcommand{\ResetMustCall}{\<@Reset\-Must\-Call>\xspace}

% "trule" stands for ``type rule''
\newcommand{\trule}[2]{\[\frac{#1}{#2}\]}
\newcommand{\truleinline}[2]{\ensuremath{#1\mathrel{\vdash}#2}}
\newcommand{\hastype}[1]{\mathbin{:}\trtext{#1}}
\newcommand{\trcode}[1]{\codeid{\smaller\smaller #1}}
\newcommand{\trtext}[1]{\mbox{\smaller\smaller #1}}
\newcommand{\trquoted}[1]{\trcode{"}#1\trcode{"}}


\hyphenation{type-state}        % LaTeX defaults to "types-tate"

% Reduce indentation in lists.
\setlength{\leftmargini}{.75\leftmargini}
\setlength{\leftmarginii}{.75\leftmarginii}
\setlength{\leftmarginiii}{.75\leftmarginiii}

\newcommand{\prefigcaption}{\vspace{-10pt}}
\newcommand{\posttablecaption}{\vspace{-10pt}}


%%
%% The "title" command has an optional parameter,
%% allowing the author to define a "short title" to be used in page headers.
\title{Software Engineering Paper Template}

%%
%% The "author" command and its associated commands are used to define
%% the authors and their affiliations.
%% Of note is the shared affiliation of the first two authors, and the
%% "authornote" and "authornotemark" commands
%% used to denote shared contribution to the research.


%%
%% By default, the full list of authors will be used in the page
%% headers. Often, this list is too long, and will overlap
%% other information printed in the page headers. This command allows
%% the author to define a more concise list
%% of authors' names for this purpose.
%\renewcommand{\shortauthors}{Trovato and Tobin, et al.}

%%
%% The abstract is a short summary of the work to be presented in the
%% article.
\begin{abstract}
Resource leaks occur when a programmer-managed resource---like a socket or
database connection in a language like Java, or memory in C---is not properly disposed.
They are notoriously difficult to test for, and can cause serious problems such
as resource starvation, denial-of-service, or even security vulnerabilities.
There exist lightweight verification systems that, if they issue no
warnings about a program, guarantee that that program does not leak any resources.
However, these systems lack support for \emph{collections} of resources---that is,
data structures like arrays, lists, or maps: extant tools will always issue a warning
about storing a resource into such a data structure, because they lack the ability to
prove that those data structures manage the resources properly. This limitation
prevents them from verifying many real-world programs that manage resources in bulk.

We propose a novel extension to prior resource leak analyses that permits them to verify
that collections of resources are properly disposed. Our core approach assumes (and enforces) that
collections of resources are \emph{homogenous}: each element of the collection must call
the same methods to be disposed, and the elements are initialized and disposed en masse.
We extend this core approach with two new kinds of limited alias analysis that are important
for making the analysis useful in practice: a specialized variant of lightweight ownership,
and a novel aliasing logic for iterators over collections that permits collections of resources
to be modified via their iterators safely. We implemented these ideas as an extension of
the Checker Framework's Resource Leak Checker for Java and conducted a set of case studies that shows
their utility in practice \todo{give specific numbers}.

\label{dummy-label-for-etags:abstract}

\end{abstract}

%%
%% Note: the 'printccs' field at the top of this file must be set to `true'
%%       in order for the CCS concepts to appear in the pdf.
%%
%% The code below is generated by the tool at http://dl.acm.org/ccs.cfm.
%% Please copy and paste the code instead of the example below.
%%
% \begin{CCSXML}
% <ccs2012>
%  <concept>
%   <concept_id>10010520.10010553.10010562</concept_id>
%   <concept_desc>Computer systems organization~Embedded systems</concept_desc>
%   <concept_significance>500</concept_significance>
%  </concept>
%  <concept>
%   <concept_id>10010520.10010575.10010755</concept_id>
%   <concept_desc>Computer systems organization~Redundancy</concept_desc>
%   <concept_significance>300</concept_significance>
%  </concept>
%  <concept>
%   <concept_id>10010520.10010553.10010554</concept_id>
%   <concept_desc>Computer systems organization~Robotics</concept_desc>
%   <concept_significance>100</concept_significance>
%  </concept>
%  <concept>
%   <concept_id>10003033.10003083.10003095</concept_id>
%   <concept_desc>Networks~Network reliability</concept_desc>
%   <concept_significance>100</concept_significance>
%  </concept>
% </ccs2012>
% \end{CCSXML}

% \ccsdesc[500]{Computer systems organization~Embedded systems}
% \ccsdesc[300]{Computer systems organization~Redundancy}
% \ccsdesc{Computer systems organization~Robotics}
% \ccsdesc[100]{Networks~Network reliability}

%%
%% Keywords. The author(s) should pick words that accurately describe
%% the work being presented. Separate the keywords with commas.
% \keywords{datasets, neural networks, gaze detection, text tagging}

%% A "teaser" image appears between the author and affiliation
%% information and the body of the document, and typically spans the
%% page.
%% \begin{teaserfigure}
%%   \includegraphics[width=\textwidth]{sampleteaser}
%%   \caption{Seattle Mariners at Spring Training, 2010.}
%%   \Description{Enjoying the baseball game from the third-base
%%   seats. Ichiro Suzuki preparing to bat.}
%%   \label{fig:teaser}
%% \end{teaserfigure}

%%
%% This command processes the author and affiliation and title
%% information and builds the first part of the formatted document.
\maketitle

\section{Introduction}
\label{sec:intro}

%% Resource leaks are a classic problem, and extant approaches to preventing
%% them are either too expensive or unsound.

A resource leak occurs when some finite resource managed by the
programmer is not explicitly disposed of. In an unmanaged language
like C, that explicit resource might be memory; in a managed language
like Java, it might be a file descriptor, a socket, or a database
connection.  Resource leaks continue to cause severe failures, even in
modern, heavily-used Java applications~\cite{ghanavati2020memory}.
This state-of-the-practice does not differ much from two decades
ago~\cite{WeimerN04}.
Microsoft engineers consider resource leaks to be one of the most
significant development challenges~\cite{LoNZ2015}.
% The fact that
% resource leaks remain such a serious problem despite decades of
% research and improvements in languages and tooling shows that
Preventing resource leaks remains an urgent, difficult, open problem.

%% An ideal tool for preventing resource leaks would be fast, sound, and precise.

Ideally, a tool for preventing resource leaks would be:
\vspace{-5pt}
\begin{itemize}
\item \emph{applicable} to
  % leakable resources in
  existing code with few code changes,
\item \emph{sound}, so that undetected resource leaks do not slip into
  the program;
\item \emph{precise}, so that developers are not bothered by excessive false positive
  warnings; and
\item \emph{fast}, so that it scales to real-world programs and
  developers can use it regularly.
\end{itemize}
\vspace{-5pt}

%% Extant approaches fail at least one of these: bug finders like the analysis
%% in ecj are unsound; complex analyses based on typestate are either too slow
%% (cite the Eurosys 2019 paper on Grapple, with numbers on run time), unsound, or both.
\noindent
Extant approaches fail at least one of these criteria.
Language-based features may not apply to all uses of resource variables:
Java's try-with-resources statement~\cite{try-with-resources}, for example, can
only close resource types that implement the \<java.lang.AutoCloseable> interface,
and cannot handle
common resource usage patterns that span multiple procedures. 
Heuristic bug-finding tools for leaks, such as those built into Java IDEs including
Eclipse~\cite{ecj-resource-leak} and IntelliJ
IDEA~\cite{idea-resource-leak}, 
are fast and applicable to legacy
code, but they are unsound.
%% They are equally as imprecise as our tool, so don't bad-mouth them.
% and can be highly imprecise (\cref{sec:compare}).
Inter-procedural typestate or dataflow analyses~\cite{TorlakC10,zuo2019grapple}
achieve more precise
%% In our experiments, they found fewer defects, and we didn't compare complexity.
% results---and can find more complex defects than bug-finders,
results---though they usually remain unsound---but
their whole-program analysis can require hours to analyze a large-scale Java program.
Finally, ownership type
systems~\cite{clarke2013ownership} as employed in languages like
Rust~\cite{klabnik2018rust} can prevent nearly all resource leaks (see
\cref{sec:rw-language}), but using them would require a significant rewrite for
a legacy codebase, a substantial task which is often infeasible.
% require
% significantly rewriting the program, 
% and prevents them from being applicable to legacy codebases and coding styles.

The goal of a leak detector for a Java-like language is to ensure that required
methods (such as \<close()>) are called on all relevant objects; we deem this
a \emph{must-call} property.  Verifying a must-call property requires
checking that required methods (or \emph{must-call obligations}) have been
called at any point where an object may become unreachable.  A static
verifier does this by computing an
under-approximation of invoked methods.  Our key insight is that checking of
must-call properties is an \emph{accumulation problem}, and hence does not
require heavyweight whole-program analysis. Our contribution is a resource leak
verifier that leverages this insight to satisfy all four requirements: it is
applicable, sound, precise, and fast.\looseness=-1

An accumulation analysis~\cite{KelloggRSSE2020}
is a special-case of typestate analysis~\cite{StromY86}.
Typestate analysis attaches a finite-state machine (FSM)
to each program element of a given type, and transitions the state of the
FSM whenever a relevant operation is performed.
In an accumulation analysis,
the order of operations performed cannot change what is subsequently
permitted, and executing more operations cannot add additional
restrictions.  Unlike arbitrary typestate analyses, accumulation analyses can
be build in a sound, modular fashion without \emph{any} whole-program alias
analysis, improving scalability and usability.

Recent work~\cite{KelloggRSSE2020} presented an accumulation analysis for
verifying that certain methods are invoked on each object
before a specific call (e.g., \<build()>).
Resource leak checking is similar in that certain methods must be invoked
on each object before it becomes unreachable.
An object becomes unreachable when its references go out
of scope or are overwritten.
By making an analogy between object-unreachability points and
method calls, we show that resource leak checking is an accumulation
problem and hence is amenable to sound, modular, and lightweight analysis.

%% Old version of the above paragraph.
% Recent work~\cite{KelloggRSSE2020} presented an accumulation analysis for
% verifying that certain methods are invoked on each object reference
% before a specific call (e.g., \<build()>).
% Conceptually,
% we can build a resource leak verifier by extending this technique
% to check that required methods are invoked before a reference goes out
% of scope or is overwritten, either of which could cause an object to become unreachable.
% This reduction of possible object-unreachability points to
% method calls shows
% that resource leak checking is an accumulation
% problem and hence is amenable to sound, modular, and lightweight analysis.

There are two key challenges for this leak-checking approach.  First,
due to subtyping, the declared type of a reference may not accurately represent
its must-call obligations; we devised a simple type system to soundly capture
these obligations.  Second, the approach is sound, but highly imprecise (more so than in
previous work~\cite{KelloggRSSE2020}) without targeted reasoning about
aliasing.  The most important patterns to
handle are:
\begin{itemize}
\item copying of resources via parameters and returns, or storing of resources in
final fields (the RAII pattern~\cite{raii});
\item wrapper types, which share their must-call obligations with one of their fields; and,
\item storing resources in non-final fields, which might be lazily initialized or
  written more than once.
\end{itemize}
To address this need,
we introduced an intra-procedural
dataflow analysis for alias tracking, and extended it with three sound
techniques to improve precision:
\begin{itemize}
\item a lightweight ownership transfer system. This system
  indicates which reference is responsible for resolving a must-call
  obligation. Unlike typical ownership type systems, our approach does
  not impact the privileges of non-owning references.
\item resource aliasing, for cases
  %% "in which" is better English, but "when" saves a line.
  % in which
  when
  a resource's must-call obligations
  can be resolved by closing one of multiple references.
\item a system for creating new obligations at locations other than the
  constructor, which allows our system to handle lazy initialization or re-initialization.
\end{itemize}
Variants of some of these ideas exist in previous work.  We bring
them together in a general, modular manner, with full verification and
the ability for programmers to easily extend checking to their own
types and must-call properties.
%
Our approach occupies a novel point in the design space for a leak detector:
unlike most prior work, it is sound; it is an order of magnitude faster than
state-of-the-art whole-program analyses; it has a false positive rate similar
to a state-of-the-practice heuristic bug-finder; and, though it does require manual
annotations from the programmer, its annotation burden is reasonable: about
1 annotation for every 1,500 lines of non-comment, non-blank code. 

Our contributions are:
\begin{itemize}
\item the insight that the resource leak problem is an accumulation
  problem, and
  % a novel set of
  an analysis approach based on
  this fact (\cref{sec:base-type-systems}).
\item three
  % novel
  innovations that improve the precision of our analysis via targeted reasoning about aliasing:
  a lightweight ownership transfer system
  (\cref{sec:lightweight-ownership}), a lightweight resource-alias
  tracking analysis (\cref{sec:must-call-choice}), and a system for
  handling lazy or multiple initialization (\cref{sec:reset-must-call}).
\item an open-source implementation for Java,
  called \tool (\cref{sec:implementation}).
\item an empirical evaluation: case studies on heavily-used
  Java programs (\cref{sec:case-studies}),
  an ablation study that shows the contributions of each innovation to
  \tool's precision (\cref{sec:ablation}), and a comparison to
  other state-of-the-art approaches that demonstrates the unique strengths
  of our approach (\cref{sec:compare}).
%% \item a comparison to alternative approaches to resource leak
%%   checking, that demonstrates that our approach occupies a unique
%%   point in the design space: faster and more usable than heavy-weight
%%   typestate systems, but sound and able to find even subtle bugs,
%%   unlike heuristic bug-finders (\cref{sec:compare}).
\end{itemize}
  

% LocalWords:  unmanaged leakable finalizers RAII


\section{Technique}
\label{sec:technique}

\todo{This section or sections explains the new techniques we used to solve the problem.
  Often, you'll want to put multiple sections here: for example, you might have an overview
  followed by a section of proofs---but it will depend on the exact content of the paper.
  Optionally, it may be preceded by a ``Background'' section that gives relevant context:
  for example, in papers about new Checker Framework checkers, we often include a background
  section that explains what type qualifiers and pluggable types are, the
  annotation syntax, etc.}

\todo{Give all sections descriptive names, not something bland like
  ``Technique'' that could appear in any paper.  Make your section names
  work for you.}


\section{Implementation}
\label{sec:implementation}

\todo{Any paper that involves building a tool should include an ``implementation'' section.
  It may discuss a new tool you have built and/or the experimental infrastructure.
  Discuss the key non-obvious design decisions and complications.
  You should also mention: 1) what language the tool targets, 2) any important libraries that
  the tool relies on (\eg the Checker Framework), and 3) that the tool is
  open-source.

  Sometimes this section is very short.}


\section{Evaluation}
\label{sec:evaluation}

% TODO: if this table is after tab:case-studies, then it doesn't appear? WTF


% a line in tab:case-studies
% arguments: project name, original LoC, # of resources (-AcountMustCall), diff size, # of annotations, TPs, Confirmed TPs, FPs, run time in seconds
\newcommand{\osstablerow}[8]{\textbf{\smaller{#1}} & #2 & #3 & #4 & #5 & #6 & #7 & #8}

\begin{table*}
  \caption{Verifying the absence of resource leaks in case studies.
    % LoC is computed using scc. Be careful when computing LoC to run scc from relevant top-level
    % directory that only contains source code (not test code!) to make sure that the count is accurate.
    % for Zookeeper, this directory is zookeeper/zookeeper-server/src/main/java
    Throughout, ``LoC'' is lines of non-comment, non-blank Java code.
    ``Resources'' is the number of resources created by the program.
    % to compute the diff size, do something like the following (this is what I did for Zookeeper) from the with-annotations branch:
    % > git diff origin/with-checker -- '*.java'
    % then count the number of individual changesets and report that
    ``Annos.'' is number of manually-written annotations to specify
    existing methods.
    ``Code changes'' is the number of distinct changes program text we made,
    not including changes that will be erased at compile time
    (such as annotations or warning suppressions).
    ``TPs'' is true positives.
    ``FPs'' is false positives, where the our analysis could not
  guarantee that the call was safe, but manual analysis revealed that no
  run-time failure was possible.
  % ``RT(s)'' is the wall-clock run time of our analysis.
  }
  \label{tab:case-studies}
  \posttablecaption

  \begin{tabular}{@{}lrr|rr|rr|r@{}}
    Project:module                                               &      LoC      & Resources   &  Annos.  &  Code changes   & TPs      & FPs & Wall-clock time      \\
    \hline
    \osstablerow{apache/zookeeper:zookeeper-server}                   {45,410}        {232}       {98}          {5}          {12}       {47}   {1m 24s}        \\
    \osstablerow{apache/hadoop:hadoop-hdfs-project/hadoop-hdfs}       {151,233}        {366}       {129}          {7}        {22}        {37}   {16m 21s}        \\
    \osstablerow{apache/hbase:hbase-server, hbase-client}             {170,569}        {62}          {35}          {1}        {11}        {27}   {7m 45s}        \\
    \hline
    \osstablerow{\textbf{Total}}                                      {563,855}        {660}         {262}          {13}        {45}       {111}   {-}        \\
  \end{tabular}
\end{table*}


Our evaluation has three parts:
\begin{itemize}
\item case studies on open-source projects, which show that our approach
  is scalable and finds real bugs (\cref{sec:case-studies}).
\item an evaluation of the contributions of our enhancements:
  lightweight ownership, % (\cref{sec:lightweight-ownership}),
  resource aliasing, % (\cref{sec:must-call-choice}), and
  and obligation creation % (\cref{sec:reset-must-call})
  (\cref{sec:ablation}).
\item a comparison to previous leak detectors:  both heuristic bug-finding
  and heavy-weight whole-program
  % typestate
  analysis (\cref{sec:compare}).
\end{itemize}

\subsection{Case studies on open-source projects}
\label{sec:case-studies}

We selected 3 popular open-source projects that were analyzed by prior work~\cite{zuo2019grapple}.
For each project, we selected and analyzed one or two modules
containing significant uses of leakable resources. We used
the latest version of the source code that was available
when we analyzed the code.

Our methodology was:
(1)
We modified the
build system to run our analysis on the module, analyzing uses of resource
classes that are defined in the JDK\@.
It also reports the maximum possible number of resources that could be
leaked:  each obligation at a formal parameter or method call.
(2) We manually
annotated each program with must-call, called-methods, and ownership
annotations (see \cref{sec:annos}).
The
annotated versions of the case study programs (including
the commit hashes that we analyzed) are available at
\todo{an anonymized fork of the projects}.
% the ``Annos'' column in \cref{tab:case-studies}).
(3) We iteratively ran the analysis to correct our annotations.
We measured the run time
as the median of 5 trials on
% This is Mike's home machine.  Are there any other relevant parameters?
a machine with an Intel Core i7-10700 CPU running at 2.90GHz and 64GiB of RAM\@.
Our analysis is embarassingly parallel, but our implementation is
single-threaded because javac is single-threaded.
(4) We manually categorized each warning as revealing a
real resource leak (a true positive or TP) or as correct code that our
system is unable to prove correct (a false positive or FP\@).

\Cref{tab:case-studies} summarizes the results. \Tool found multiple
serious resource leak bugs in every program. \Tool issues
more false positives than true positives in each program, but
the number is small enough to be examined by a single developer in a
few hours.  This is a small price to pay for knowing that the program is
free of resource leaks.  The annotations in the program are
also a benefit: as a form of machine-checked documentation, they
express the programmer's intent and, unlike traditional comments,
cannot become out-of-date if the checker is passing.

\subsubsection{True and false positive examples}
\label{sec:examples}

This section describes some examples of warnings reported by \Tool
in our case studies.\looseness=-1

\begin{figure}
  \lstinputlisting{hadoop-bug.txt}
  \prefigcaption
  \caption{A true positive that \Tool found in Hadoop. Our pull request
    to fix this bug was accepted by Hadoop's developers.}
  \label{fig:hadoop-bug}
\end{figure}

\Cref{fig:hadoop-bug} contains code from Hadoop. If an IO error
occurs any time between the allocation of the \<FileInputStream>
in the first line of the method and the \<return> statement
at the end---for example, if \<channel.position(section.getOffset())>
throws an \<IOException>, as it is specified to do---then the
only reference to the stream is lost. Hadoop's developers
assigned this bug a priority of ``Major'' and accepted our
patch.\footnoteanonurl{https://github.com/apache/hadoop/pull/2652}
One developer suggested using a try-with-resources statement instead
of our patch (which catches the exception and closes the stream),
but we pointed out that
the file needs to remain open if no error occurs so that it can be
returned.  \manu{this example is also interesting in that verifying the caller
is non-trivial.  dunno if we have space, but if we do, we may want to include}
\todo{say something like we are in the process of submitting more bug reports and fixes?}  

\begin{figure}
  \lstinputlisting{zookeeper-optional.txt}
  \prefigcaption
  \caption{Code from the ZooKeeper case study that causes \Tool
  to issue a false positive.}
  \label{fig:zookeeper-optional}
\end{figure}

The most common false
positive pattern (with 15 instances) was caused by
a generic container object like \<java.util.Optional> taking ownership of a resource, such
as the example in \cref{fig:zookeeper-optional}. Our lightweight ownership
system does not support transferring ownership to generic parameters,
so \Tool issues an error when \<Optional.of> is returned. In this case, the use
of the \<Optional> class complicates the code; if \<Optional> was replaced
by \<null>,
%% Leave out of submitted version.
% as some advocate~\cite{ErnstNothingIsBetterThanOptional},
\Tool could verify this code. We leave expanding the lightweight ownership system to
support Java generics as future work.

Most false positives were caused by unique coding patterns.
One example is a series of catch
statements, each of which set an \<error> boolean to \<true>, followed
by a \<finally> statement that closed the relevant socket if \<error>
was true.  \Tool reports an error because it cannot reason about arbitrarily
complex path conditions.


\subsubsection{Annotations and code changes}
\label{sec:annos}

%% Here's the full table of annotations:

%% -              ZK  HB  HD  Total
%% Owning         28  12  43  83
%% NotOwning      8   5   17  30  = 113
%% ----
%% MustCall       13  7   11  31
%% InheritableMC  1   2   7   9   = 41
%% ---
%% MustCallChoice 4   0   28  34
%% PolyMustCall   4   0   0   4   = 36
%% ---
%% ResetMustCall  19  5   5   29  = 29
%% ---
%% EnsuresCM      21  4   18  43  = 43


\begin{table}
  \caption{The total number of annotations that we wrote.}
  \label{tab:annos}
  \posttablecaption
  % counts @InheritableMustCall as @MustCall, for simplicity of presentation
  % counts @polymustcall with MCC, for the same reason
  \begin{tabularx}{\columnwidth}{@{}Xr@{}}
    Annotation                           &      Count     \\
    \hline
    \<@Owning> and \<@NotOwning>            &      113   \\
    \<@EnsuresCalledMethods>                &      43       \\
    \<@MustCall>                            &      41       \\
    \<@MustCallAlias>                       &      36       \\
    \CreateObligation                       &      29      \\
  \end{tabularx}
\end{table}

% Because most relevant \MustCall and \<@CalledMethods> annotations
% are inferred intra-procedurally, most annotations that we had to write in
% the case studies pertained to ownership or the side-effects of procedures.
We wrote less than one annotation per 2,000 lines of code (\cref{tab:annos}).

\todo{Double-check all numbers.}
We also
made 13 small, semantics-preserving changes to the programs to reduce
false positives from our analysis.
%
In 8 places, we added \<final> to a field; this allows our checker to verify it without using
the rules for non-final owning fields given in \cref{sec:reset-must-call}, which are stricter.
In 3 of those, we also removed assignments to the field after it was closed whose right-hand
side was \<null>; in 1 other we added an \<else> clause in the constructor that assigned the field
a \<null> value.
%
In 3 places, we re-ordered two statements to remove an
infeasible control-flow-graph edge.
% In 2 places, we re-ordered a \<try> statement and a null-check.
% % In 1 place, we re-ordered two calls to \<close()> that closed
% the same resource (via resource aliases). The first call in the
% unmodified code was conditional, but the second was
% not, leading to an infeasible control-flow graph edge.
%
In 2 places, we extracted an expression into a local variable, permitting
flow-sensitive reasoning or targetting by a \CreateObligation annotation.
% multiple calls to a method
% into a single call that is assigned to a local variable; this allows
% flow-sensitive reasoning about the local
% variable.
% % , and the code is also more efficient.
% In 1 place, we extracted an expression into a local variable
% so that a \CreateObligation annotation could target it.


\subsection{Evaluating our enhancements}
\label{sec:ablation}

% a line in tab:ablation
% note that the FULL column is always zero, so it's not included here
% arguments: project name, no-LO, no-RA, no-AF
\newcommand{\abltablerow}[4]{\textbf{\smaller{#1}} & #2 & #3 & #4}

\begin{table}
  \caption{The contribution of the lightweight
    ownership, resource aliasing,
    and our system for creating obligations at points other than constructors
    in reducing false positives. Each entry is the number of extra
    false positive warnings reported by the variant with the given feature disabled on the given project.}
  \label{tab:ablation}
  \posttablecaption
  
  \begin{tabularx}{\columnwidth}{@{}Xrrr@{}}
    Project                              &      without LO & without RA & without CO     \\
    \hline
    \abltablerow{apache/zookeeper}              {66}            {97}             {10}                               \\
    \abltablerow{apache/hadoop}                   {95}            {113}             {0}                               \\
    \abltablerow{apache/hbase}                  {52}            {57}             {0}                               \\
    \hline
    \abltablerow{\textbf{Total}}                {213}            {267}             {10}                               \\
  \end{tabularx}
\end{table}

The base analysis of \cref{sec:base-type-systems} produces significantly
more false positives, because it lacks 
lightweight ownership (\cref{sec:lightweight-ownership}),
resource aliasing (\cref{sec:must-call-choice}), and
creating obligations at points other than constructors (\cref{sec:reset-must-call}).
Each contributes to \Tool's precision by eliminating false positives.
To evaluate the contribution of each enhancement, we individually disabled each
feature and re-ran the experiments of \cref{sec:case-studies}.
Since all variants are sound (no false
negatives), any difference in warnings is a false positive that is prevented
by the feature.

\Cref{tab:ablation} shows that each of lightweight
ownership and resource aliases prevents more false positive errors than the total number
of remaining false positives on each benchmarks---showing that removing either
would make our technique produce an unreasonable number of false positives.
The system for creating new obligations at points other than constructors reduces
false positives by a smaller amount: non-final, owning field re-assignments are rare;
and we encountered the unconnected socket pattern described in \cref{sec:unconnected-sockets}
only in ZooKeeper. Nevertheless, this feature allows our tool to handle an important, if rare,
coding pattern.

\subsection{Comparison to other tools}
\label{sec:compare}

Our approach represents a novel point in the design space of resource leak checkers.
%
This section compares our approach with two other modern tools that detect resource leaks:
\begin{itemize}
\item The analysis built into the Eclipse Compiler for Java (ecj), which is the default approach
  for detecting resource leaks in the Eclipse IDE~\cite{ecj-resource-leak}.
\item Grapple~\cite{zuo2019grapple} represents a significant, recent
  improvement in the scalability of typestate-based tools that require a whole-program alias analysis.
\end{itemize}
In brief, both tools are unsound.
Both tools are applicable to legacy code. % , without the need to write annotations.
Eclipse is very fast (nearly instantaneous) and suffers 67--98\% false
positives. % while missing 70-90\% of resource leaks.
According to its authors, Grapple is precise (13\% false positive rate) but orders
of magnitude slower than \Tool.
Different users can select whichever tool matches
their priorities.


\subsubsection{Eclipse}
\label{sec:eclipse}

The Eclipse analysis is a simple dataflow analysis
augmented with heuristics. Because it is tightly integrated with
the compiler, it scales well and runs quickly. It contains
heuristics for ownership, resource wrappers, and resource-free
closeables, among others; these are all hard-coded into the analysis and cannot
be adjusted by the user. It does not support annotations to express
specifications that differ from its defaults.
It supports two levels of analysis: detecting high-confidence resource
leaks and detecting ``potential'' resource
leaks (a superset of high-confidence resource leaks).

We ran this analysis on the same version of Zookeeper's zookeeper-server
module that we ran \Tool on in \cref{sec:case-studies}. It is easy to apply
to legacy code and it is fast---nearly instantaneous once Eclipse
has loaded the project.

In ``high-confidence'' mode on zookeeper-server, Eclipse reports 3
warnings: 1 true positive (thus, it misses 9 real resource leaks) and 2
false positives.
In ``potential'' leak mode, the analysis reported 180
warnings:  3 true positives (it misses 7 real resource leaks) and 177 false
positives.
The most common cause of false
positives was the unchangeable, default ownership transfer assumption
at method invocations: it warned at each call that returns a resource-alias, such as
\<Socket\#getInputStream>.

These results demonstrate that the Eclipse bug-finder is unsound in both
modes: it misses 7 real warnings. Further, the analysis is imprecise: in potential leak mode,
the vast majority of warnings are false positives. However, this sort
of bug-finding tool does have two advantages: it is fast, and it is easily
applicable to legacy code. Compared to our tool, the Eclipse analysis
is much faster and equally applicable to legacy code, but both less sound
and less precise.

%% Chandra rightly pointed out that this para make us look lazy. Let's just
%% say what we did, not why we did it.
%%
%% Given that these results on zookeeper-server
%% clearly show the strengths and weaknesses of this tool compared to our own,
%% we did not run it on our other case study programs.

\subsubsection{Grapple}
\label{sec:grapple}

% a line in tab:grapple
% these numbers come from tables 2 and 3 in the grapple paper from EuroSys.
% for TPs and FPs, I added together the IO and Socket columns of table 2.
\newcommand{\grappletablerow}[4]{\textbf{\smaller{#1}} & #2 & #3 & #4}

\begin{table}
  \caption{The Grapple tool's performance; reproduced from~\cite{zuo2019grapple}.}
  \label{tab:grapple}
  \posttablecaption
  
  \begin{tabularx}{\columnwidth}{@{}Xrrr@{}}
    Project                              &  TPs    &    FPs         & Run time      \\
    \hline
    \grappletablerow{ZooKeeper}             {6}         {0}           {01h 07m 02s}     \\
    \grappletablerow{HDFS}                  {5}         {2}           {01h 54m 52s}    \\
    \grappletablerow{HBase}                 {15}        {2}           {33h 51m 59s}     \\
    \hline
    \grappletablerow{\textbf{Total}}        {26}        {4}           {-}          \\
  \end{tabularx}
\end{table}

% \todo{Be very sure that Grapple is unsound - read their paper again, and
%    point out why in this paragraph.}
%\todo{be sure to note that Grapple is not modular, unlike us}
Grapple~\cite{zuo2019grapple} is a modern typestate-based resource leak analysis
focused on high precision and (relative) scalability. Grapple models its alias and
dataflow analyses as dynamic transitive-closure computations over graphs, and
leverages novel path encodings and techniques from predecessor-system
Graspan~\cite{wang2017graspan} to achieve both context- and path-sensitivity.  
% Grapple, like all previous
% typestate-based analyses, requires a whole-program alias analysis.
Grapple contains four checkers, of which two are useful for detecting
resource leaks.  Unlike \Tool, Grapple is unsound; e.g., it performs a fixed bounded unrolling
of loops to make path sensitivity tractable.

The Grapple authors have already evaluated their tool on earlier
versions\todo{Why don't we repeat our experiments on those earlier versions?
This should be easy to do since it's only a few hours of work and we are now
familiar with the programs.} of the case study programs in
\cref{sec:case-studies}~\cite{zuo2019grapple}; \Cref{tab:grapple} reproduces the
results from their paper.  Based on these numbers, Grapple reports fewer false
positives than \Tool.  Unfortunately, we were not able to successfully run
Grapple's leak detection, and full details on their true and false-positive
warnings are currently unavailable (we have requested them from the authors), so
we cannot study the precision differences further.  The run times in
\Cref{tab:grapple} show a stark difference with our work; \Tool runs in minutes,
whereas Grapple can take many hours. Further, to our best knowledge, Grapple is
not modular, so any code modification will necessitate a full re-analysis.
\Tool only needs to re-analyze modified code and possibly its dependents after
a change, not unmodified dependencies.

% We requested the analysis and Grapple's output, but as of this writing
% (months later) the Grapple authors have not yet provided them.
% \manu{I think this is strictly-speaking true but a bit harsh.  We
% may want to tone it down a bit.}

\todo{Still keep this?} Note that the TP and FP numbers not perfectly comparable between a modular
and a whole-program analysis.
Our tool reports violations of a user-supplied specification
(which takes effort to write but provides documentation benefits), so it
can ensure that a library is correct for all possible clients.  By
contrast, Grapple checks a library in the context of one specific client.

% LocalWords:  LoC SHAs leakable


\section{Limitations and threats to validity}
\label{sec:threats}

Like any tool that analyzes source code, \tool only
gives guarantees for code that it checks: the guarantee
excludes native code, the implementation of unchecked libraries (such as the JDK),
and code generated dynamically or by other annotation processors
such as Lombok.
%% MK: the sentence says that we can't guarantee that code generated by annotation
%% processors is checked, which is true, because they can misbehave in the way
%% that Lombok does. Most annotation processors don't, but the point is that
%% they can, so we can't promise that our analysis will be applied.
%% \todo{nit: is it annotation processors in general that cause an
%% issue, or Lombok because of the insane things it does?  I think for a typical
%% annotation processor doing code generation we shouldn't have problems?}
Though
the Checker Framework can handle 
reflection soundly~\cite{BarrosJMVDdAE2015}, by default (and in our case studies)
\tool compromises this guarantee
by assuming that objects returned by reflective invocations
do not carry must-call obligations.  (Users can customize this behavior.)
% Like any sound static analysis, \Tool does
% issue some false positives; these must be suppressed by
% the programmer and the correctness of the relevant code
% must be proved using another method.
Within the bounds
of a user-written warning suppression, \tool assumes that 1)
any errors issued can be ignored, and 2) all annotations
written by the programmer are correct.

\Tool is sound with respect to specifications of which types have a
\<@MustCall> obligation that must be satisfied.  We wrote such specifications
for the Java standard library, focusing on IO-related code in the \<java.io> and
\<java.nio> packages.  Any missing specifications of \<@MustCall> obligations
could lead \tool to miss resource leaks.

The results of our experiments may not generalize, compromising the
external validity of the experimental results.
% In particular, our subject programs are
% heavily-used, heavily-tested, and contain a high-density of resource
% usage---so, in a sense, they represent a worst-case scenario for
% \tool.  Nevertheless,
\Tool may produce more false positives, require
more annotations, or be more difficult to use if applied to other
programs.
Case studies on legacy code represents a worst case for a source code
analysis tool.  Using \tool from
the inception of a project would be easier, since programmers know their
intent as they write code and annotations could be written along with the
code.  It would also be more useful, since it would guide the programmers
to a better design that requires fewer annotations and has no resource leaks.
The need for annotations could be viewed as a limitation of our approach.
However, the annotations serve as concise documentation of
properties relevant to resource leaks---and unlike traditional, natural-language
documentation, machine-checked annotations cannot become out-of-date.

Like any practical system, it is possible that there might
be defects in the implementation of \tool or in the design of
its analyses. We have mitigated this threat with code review and an extensive
test suite:
% to collect these numbers, run these commands and sum the results:
% cd object-construction-checker/tests
% scc mustcall socket nolightweightownership mustcall-onlyjdk noresourcealias noaccumulationframes
% cd ../../must-call-checker/tests
% scc
119 test classes containing 3,776 lines of non-comment, non-blank code.
This test suite is publicly available and distributed with \tool.

% LocalWords:  checkable MustCall io nio


\section{Related Work}
\label{sec:relatedwork}

Previous work on resource leak detection has typically been based on either
using program analysis to detect leaks or adding language features to prevent
them. Here we discuss the most relevant work from each of these categories.

\subsection{Analysis-based approaches}\label{sec:rw-analysis}

%\todo{Re-organize and edit the below in a way that makes sense.}

\paragraph{Static analysis}
Tracker~\cite{TorlakC10} performs inter-procedural dataflow analysis to detect
resource leaks, with various additional features to make their tool practical,
including issue prioritization and handling of wrapper types.  Tracker avoids
whole-program alias analysis to improve scalability, instead using a local,
access-path-based approach.  While Tracker scales well to large programs, it is
deliberately unsound, unlike \Tool.

The Eclipse Compiler for Java includes a simple dataflow-based
bug-finder for resource leaks~\cite{ecj-resource-leak}. Their analysis
uses a fixed set of ownership heuristics and a fixed list of wrapper
classes; unlike \Tool, it is unsound. It is fast.  Similar analyses---with similar trade-offs
compared to \Tool---are present in other heuristic bug-finding tools,
including SpotBugs~\cite{spotbugs-resource-leak},
PMD~\cite{pmd-resource-leak}, and Infer~\cite{infer-resource-leak}.
\Cref{sec:eclipse} experimentally evaluates the Eclipse analysis.

Typestate analysis~\cite{StromY86,FinkYDRG2008} can be used to
find resource leaks.  Grapple~\cite{zuo2019grapple} is the most recent system to
use this approach, leveraging a disk-based graph engine to achieve unprecedented
scalability on a single machine.  Compared to \Tool, Grapple is more precise but suffers
from unsoundness and much longer run times.
\Cref{sec:grapple} gives a more detailed comparison to Grapple.

Relda and Relda2~\cite{guo2013characterizing,wu2016relda2} are unsound
resource-leak detection approaches \todo{similar to Eclipse,then?} that are
augmented with call graphs to model the Android framework's use of callbacks for
releasing resources.

The CLOSER~\cite{dillig2008closer} automatically inserts Java code to dispose of
resources when they are no longer ``live'' according to its dataflow analysis.
Their approach requires an expensive alias analysis for soundness, as well as
manually-provided aliasing specifications for linked libraries.
\todo{\Tool requires that too, though only lightweight specifications.
  Mention that?}
\Tool uses accumulation
analysis~\cite{KelloggRSSE2020,FahndrichLeino03} to achieve soundness without
the need
for a whole-program alias analysis.
We
build on the core analysis described in~\cite{KelloggRSSE2020}.

\paragraph{Dynamic analysis}
Some approaches use dynamic analysis to ameliorate leaks.  Resco \cite{dai2013resco}
operates similarly to a garbage collector, tracking resources whose program
elements have become unreachable. When a given resource (such as file
descriptors) is close to exhaustion, the runtime runs Resco to clean up any
resources of that type that are unreachable.  With a static approach such as
ours, a tool like Resco becomes unnecessary---leaks become impossible.

Automated test generation can also be used to try to detect resource
leaks. For example, leaks in Android applications can be found by
repeatedly running neutral---eventually returning to the same
state---GUI actions~\cite{wu2018sentinel,zhang2016automated}.
Other techniques focus on taking advantage of common misuse of
the Android activity lifecycle~\cite{amalfitano2020memories}.
Testing can only show the presence of bugs, not their absence;
\Tool verifies that no resource leaks are present.

\paragraph{Data sets and surveys}
In terms of data sets for leak detection, The DroidLeaks
benchmark~\cite{liu2019droidleaks} is a set of Android apps with known resource
leak bugs. Unfortunately, it includes only the compiled apps.
\Tool runs on source code, so we were unable to run \Tool on DroidLeaks.

\todo{This~\cite{ghanavati2020memory} is a recent study on what kinds
  of resource and memory leak bugs happen in large Java projects. I'm
  sure we can use it somewhere.}



\subsection{Language-based approaches}\label{sec:rw-language}

\paragraph{Ownership types and Rust} Ownership type
systems~\cite{clarke2013ownership} impose control over aliasing, which
in turn enables guaranteeing other high-level properties, like absence of
resource leaks.  We do not discuss the vast literature on ownership type systems
here (see Clarke et al.~\cite{clarke2013ownership} for a survey). Instead, we
focus on ownership types in Rust~\cite{klabnik2018rust} as the most popular
practical example of using ownership to prevent resource leaks.

For a detailed overview of ownership in Rust, see Chapter 4
of~\cite{klabnik2018rust}; we give a brief overview here.  In Rust, ownership is
used to manage both memory and other resources.  Every value associated with a
resource must have a \emph{unique} owning pointer, and when an owning pointer's
lifetime ends, the value is ``dropped,'' ensuring all resources are freed.
Rust's ownership type system statically prevents
not only resource leaks, but also other important issues like ``double-free'' bugs
(releasing a resource more than once) and ``use-after-free'' bugs (using a
resource after it has been released). However, this power comes with a cost; to
enforce uniqueness, non-owning pointers must be invalidated after an ownership transfer,
and can no longer be used to perform any operations.  Maintaining multiple
usable pointers to a value requires use of language features like references and
borrowing, and even then, borrowed pointers have restricted privileges.

\Tool has less power than Rust's ownership types; it cannot prevent resources
from being closed multiple times or being used after they are closed.  However,
\Tool's lightweight ownership annotations impose \emph{no} restrictions on
aliasing; they are simply an aid to help the tool identify how a resource will
be closed.  Hence, lightweight ownership is much better suited to preventing
resource leaks in existing, large Java code bases; adapting such programs to use
a full Rust-style ownership type system would be an enormous and impractical
undertaking.

\paragraph{Other approaches} Compensation stacks~\cite{WeimerN04}
generalize C++ destructors and Java's
try-with-resources, to avoid resource leak problems in Java.  While
compensation stacks make resource leaks less likely, they do not provide a
guarantee that leaks will not occur, unlike \Tool.

An alternative to applying a typestate analysis to an existing program
is to rewrite the program in a typestate-oriented programming
language~\cite{AldrichSSS2009,garcia2014typestate}.  The type systems of such
languages are expressive enough to prevent resource leaks, but they impose
restrictions on aliasing and a high type annotation burden, making translation
of existing code impractical.

%\todo{Is there other relevant related work? Find out.}

%\todo{We may want to cite RAII (\url{https://en.wikipedia.org/wiki/Resource_acquisition_is_initialization}) and
%state that our handling of owning fields can verify it.}



% Accumulation analysis is a special-case of typestate analysis~\cite{StromY86}.
% In general, a typestate analysis requires a whole-program alias analysis
% for soundness, making them impractical for resource leak detection
% in industrial-size Java programs. \todo{However, people have probably tried
%   so go figure that out.}

% LocalWords:  Relda Relda2 Resco runtime DroidLeaks


\section{Conclusion}

\todo{Recap the contributions of the paper.  (In fact, it is good to think of
  this as a ``contributions'' section even if it's titled ``conclusion''.)
  This section can do so in more depth or with more reference to techniques
  and insights than the abstract and introduction were able to do, since
  readers of those sections wouldn't have read the whole paper yet.
  In other cases, this section is brief.}


\todo{The page limit is {\conferencePageLimit} pages.  This is page \thepage.}

% \begin{acks}
% \end{acks}

%% The next two lines define the bibliography style to be used, and
%% the bibliography file.
\bibliographystyle{ACM-Reference-Format}
\bibliography{local,plume-bib/bibstring-abbrev,plume-bib/types,plume-bib/dispatch,plume-bib/ernst,plume-bib/soft-eng,plume-bib/crossrefs}

%%
%% If your work has an appendix, this is the place to put it.

\end{document}
\endinput
%%
