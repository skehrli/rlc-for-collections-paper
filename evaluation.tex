\section{Evaluation}
\label{sec:evaluation}

\todo{All experimental papers need a good evaluation section.
  This section might include a list of research questions.
  Often, there are separate sections for the experimental methodology and
  the results.

  One of the first things that I do when
  drafting the outline for a new paper is to design the ``main table'': that is, I add
  to this section a table that has row and column headers (but not actual numbers). I find this
  forces me to think about what I'm actually planning to measure, which helps design both
  better narratives (for the intro) and better experiments. This section might also be split
  up into multiple sections if there is a logical grouping of the experiments (\eg open- vs.\
  closed-source subjects, comparisions with other tools, etc.).}

\todo{It's common to include ``research questions'' in the structure
  of your evaluation narrative.  You can use the \<\textbackslash
  researchquestions> environment to automatically number and format
  them, like this:
  \begin{researchquestions}
  \item How do you write a good evaluation section?
  \end{researchquestions}
}
