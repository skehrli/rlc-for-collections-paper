\section{Evaluation}
\label{sec:evaluation}

% TODO: if this table is after tab:case-studies, then it doesn't appear? WTF


% a line in tab:case-studies
% arguments: project name, original LoC, # of resources (-AcountMustCall), diff size, # of annotations, TPs, Confirmed TPs, FPs, run time in seconds
\newcommand{\osstablerow}[8]{\textbf{\smaller{#1}} & #2 & #3 & #4 & #5 & #6 & #7 & #8}

\begin{table*}
  \caption{Verifying the absence of resource leaks in case studies.
    % LoC is computed using scc. Be careful when computing LoC to run scc from relevant top-level
    % directory that only contains source code (not test code!) to make sure that the count is accurate.
    % for Zookeeper, this directory is zookeeper/zookeeper-server/src/main/java
    Throughout, ``LoC'' is lines of non-comment, non-blank Java code.
    ``Resources'' is the number of resources created by the program.
    % to compute the diff size, do something like the following (this is what I did for Zookeeper) from the with-annotations branch:
    % > git diff origin/with-checker -- '*.java'
    % then count the number of individual changesets and report that
    ``Annos.'' is number of manually-written annotations to specify
    existing methods.
    ``Code changes'' is the number of distinct changes program text we made,
    not including changes that will be erased at compile time
    (such as annotations or warning suppressions).
    ``TPs'' is true positives.
    ``FPs'' is false positives, where the our analysis could not
  guarantee that the call was safe, but manual analysis revealed that no
  run-time failure was possible.
  % ``RT(s)'' is the wall-clock run time of our analysis.
  }
  \label{tab:case-studies}
  \posttablecaption

  \begin{tabular}{@{}lrr|rr|rr|r@{}}
    Project:module                                               &      LoC      & Resources   &  Annos.  &  Code changes   & TPs      & FPs & Wall-clock time      \\
    \hline
    \osstablerow{apache/zookeeper:zookeeper-server}                   {45,410}        {232}       {98}          {5}          {12}       {47}   {1m 24s}        \\
    \osstablerow{apache/hadoop:hadoop-hdfs-project/hadoop-hdfs}       {151,233}        {366}       {129}          {7}        {22}        {37}   {16m 21s}        \\
    \osstablerow{apache/hbase:hbase-server, hbase-client}             {170,569}        {62}          {35}          {1}        {11}        {27}   {7m 45s}        \\
    \hline
    \osstablerow{\textbf{Total}}                                      {563,855}        {660}         {262}          {13}        {45}       {111}   {-}        \\
  \end{tabular}
\end{table*}


Our evaluation has three parts:
\begin{itemize}
\item case studies on open-source projects, which show that our approach
  is scalable and finds real bugs (\cref{sec:case-studies}).
\item an evaluation of the contributions of our enhancements:
  lightweight ownership, % (\cref{sec:lightweight-ownership}),
  resource aliasing, % (\cref{sec:must-call-choice}), and
  and obligation creation % (\cref{sec:reset-must-call})
  (\cref{sec:ablation}).
\item a comparison to previous leak detectors:  both heuristic bug-finding
  and heavy-weight whole-program
  % typestate
  analysis (\cref{sec:compare}).
\end{itemize}

\subsection{Case studies on open-source projects}
\label{sec:case-studies}

We selected 3 popular open-source projects that were analyzed by prior work~\cite{zuo2019grapple}.
For each project, we selected and analyzed one or two modules
containing significant uses of leakable resources. We used
the latest version of the source code that was available
when we analyzed the code.

Our methodology was:
(1)
We modified the
build system to run our analysis on the module, analyzing uses of resource
classes that are defined in the JDK\@.
It also reports the maximum possible number of resources that could be
leaked:  each obligation at a formal parameter or method call.
(2) We manually
annotated each program with must-call, called-methods, and ownership
annotations (see \cref{sec:annos}).
The
annotated versions of the case study programs (including
the commit hashes that we analyzed) are available at
\url{https://github.com/fse-main-307}.
% the ``Annos'' column in \cref{tab:case-studies}).
(3) We iteratively ran the analysis to correct our annotations.
We measured the run time
as the median of 5 trials on
% This is Mike's home machine.  Are there any other relevant parameters?
a machine with an Intel Core i7-10700 CPU running at 2.90GHz and 64GiB of RAM\@.
Our analysis is embarassingly parallel, but our implementation is
single-threaded because javac is single-threaded.
(4) We manually categorized each warning as revealing a
real resource leak (a true positive or TP) or as correct code that our
system is unable to prove correct (a false positive or FP\@).

\Cref{tab:case-studies} summarizes the results. \Tool found multiple
serious resource leak bugs in every program. \Tool issues
more false positives than true positives in each program, but
the number is small enough to be examined by a single developer in a
few hours.  This is a small price to pay for knowing that the program is
free of resource leaks.  The annotations in the program are
also a benefit: as a form of machine-checked documentation, they
express the programmer's intent and, unlike traditional comments,
cannot become out-of-date if the checker is passing.

\subsubsection{True and false positive examples}
\label{sec:examples}

This section describes some examples of warnings reported by \Tool
in our case studies.\looseness=-1

\begin{figure}
  \lstinputlisting{hadoop-bug.txt}
  \prefigcaption
  \caption{A true positive that \Tool found in Hadoop. Our pull request
    to fix this bug was accepted by Hadoop's developers.}
  \label{fig:hadoop-bug}
\end{figure}

\Cref{fig:hadoop-bug} contains code from Hadoop. If an IO error
occurs any time between the allocation of the \<FileInputStream>
in the first line of the method and the \<return> statement
at the end---for example, if \<channel.position(section.getOffset())>
throws an \<IOException>, as it is specified to do---then the
only reference to the stream is lost. Hadoop's developers
assigned this bug a priority of ``Major'' and accepted our
patch.\footnoteanonurl{https://github.com/apache/hadoop/pull/2652}
One developer suggested using a try-with-resources statement instead
of our patch (which catches the exception and closes the stream),
but we pointed out that
the file needs to remain open if no error occurs so that it can be
returned.  \manu{this example is also interesting in that verifying the caller
is non-trivial.  dunno if we have space, but if we do, we may want to include}
\todo{say something like we are in the process of submitting more bug reports and fixes?}  

\begin{figure}
  \lstinputlisting{zookeeper-optional.txt}
  \prefigcaption
  \caption{Code from the ZooKeeper case study that causes \Tool
  to issue a false positive.}
  \label{fig:zookeeper-optional}
\end{figure}

The most common false
positive pattern (with 15 instances) was caused by
a generic container object like \<java.util.Optional> taking ownership of a resource, such
as the example in \cref{fig:zookeeper-optional}. Our lightweight ownership
system does not support transferring ownership to generic parameters,
so \Tool issues an error when \<Optional.of> is returned. In this case, the use
of the \<Optional> class complicates the code; if \<Optional> was replaced
by \<null>,
%% Leave out of submitted version.
% as some advocate~\cite{ErnstNothingIsBetterThanOptional},
\Tool could verify this code. We leave expanding the lightweight ownership system to
support Java generics as future work.

Most false positives were caused by unique coding patterns.
One example is a series of catch
statements, each of which set an \<error> boolean to \<true>, followed
by a \<finally> statement that closed the relevant socket if \<error>
was true.  \Tool reports an error because it cannot reason about arbitrarily
complex path conditions.


\subsubsection{Annotations and code changes}
\label{sec:annos}

%% Here's the full table of annotations:

%% -              ZK  HB  HD  Total
%% Owning         28  12  43  83
%% NotOwning      8   5   17  30  = 113
%% ----
%% MustCall       13  7   11  31
%% InheritableMC  1   2   7   9   = 41
%% ---
%% MustCallChoice 4   0   28  34
%% PolyMustCall   4   0   0   4   = 36
%% ---
%% ResetMustCall  19  5   5   29  = 29
%% ---
%% EnsuresCM      21  4   18  43  = 43


\begin{table}
  \caption{The total number of annotations that we wrote.}
  \label{tab:annos}
  \posttablecaption
  % counts @InheritableMustCall as @MustCall, for simplicity of presentation
  % counts @polymustcall with MCC, for the same reason
  \begin{tabularx}{\columnwidth}{@{}Xr@{}}
    Annotation                           &      Count     \\
    \hline
    \<@Owning> and \<@NotOwning>            &      113   \\
    \<@EnsuresCalledMethods>                &      43       \\
    \<@MustCall>                            &      41       \\
    \<@MustCallAlias>                       &      36       \\
    \CreateObligation                       &      29      \\
  \end{tabularx}
\end{table}

% Because most relevant \MustCall and \<@CalledMethods> annotations
% are inferred intra-procedurally, most annotations that we had to write in
% the case studies pertained to ownership or the side-effects of procedures.
We wrote less than one annotation per 2,000 lines of code (\cref{tab:annos}).

\todo{Double-check all numbers.}
We also
made 13 small, semantics-preserving changes to the programs to reduce
false positives from our analysis.
%
In 8 places, we added \<final> to a field; this allows our checker to verify it without using
the rules for non-final owning fields given in \cref{sec:reset-must-call}, which are stricter.
In 3 of those, we also removed assignments to the field after it was closed whose right-hand
side was \<null>; in 1 other we added an \<else> clause in the constructor that assigned the field
a \<null> value.
%
In 3 places, we re-ordered two statements to remove an
infeasible control-flow-graph edge.
% In 2 places, we re-ordered a \<try> statement and a null-check.
% % In 1 place, we re-ordered two calls to \<close()> that closed
% the same resource (via resource aliases). The first call in the
% unmodified code was conditional, but the second was
% not, leading to an infeasible control-flow graph edge.
%
In 2 places, we extracted an expression into a local variable, permitting
flow-sensitive reasoning or targetting by a \CreateObligation annotation.
% multiple calls to a method
% into a single call that is assigned to a local variable; this allows
% flow-sensitive reasoning about the local
% variable.
% % , and the code is also more efficient.
% In 1 place, we extracted an expression into a local variable
% so that a \CreateObligation annotation could target it.


\subsection{Evaluating our enhancements}
\label{sec:ablation}

% a line in tab:ablation
% note that the FULL column is always zero, so it's not included here
% arguments: project name, no-LO, no-RA, no-AF
\newcommand{\abltablerow}[4]{\textbf{\smaller{#1}} & #2 & #3 & #4}

\begin{table}
  \caption{The contribution of the lightweight
    ownership, resource aliasing,
    and our system for creating obligations at points other than constructors
    in reducing false positives. Each entry is the number of extra
    false positive warnings reported by the variant with the given feature disabled on the given project.}
  \label{tab:ablation}
  \posttablecaption
  
  \begin{tabularx}{\columnwidth}{@{}Xrrr@{}}
    Project                              &      without LO & without RA & without CO     \\
    \hline
    \abltablerow{apache/zookeeper}              {66}            {97}             {10}                               \\
    \abltablerow{apache/hadoop}                   {95}            {113}             {0}                               \\
    \abltablerow{apache/hbase}                  {52}            {57}             {0}                               \\
    \hline
    \abltablerow{\textbf{Total}}                {213}            {267}             {10}                               \\
  \end{tabularx}
\end{table}

The base analysis of \cref{sec:base-type-systems} produces significantly
more false positives, because it lacks 
lightweight ownership (\cref{sec:lightweight-ownership}),
resource aliasing (\cref{sec:must-call-choice}), and
creating obligations at points other than constructors (\cref{sec:reset-must-call}).
Each contributes to \Tool's precision by eliminating false positives.
To evaluate the contribution of each enhancement, we individually disabled each
feature and re-ran the experiments of \cref{sec:case-studies}.
Since all variants are sound (no false
negatives), any difference in warnings is a false positive that is prevented
by the feature.

\Cref{tab:ablation} shows that each of lightweight
ownership and resource aliases prevents more false positive errors than the total number
of remaining false positives on each benchmarks---showing that removing either
would make our technique produce an unreasonable number of false positives.
The system for creating new obligations at points other than constructors reduces
false positives by a smaller amount: non-final, owning field re-assignments are rare;
and we encountered the unconnected socket pattern described in \cref{sec:unconnected-sockets}
only in ZooKeeper. Nevertheless, this feature allows our tool to handle an important, if rare,
coding pattern.

\subsection{Comparison to other tools}
\label{sec:compare}

Our approach represents a novel point in the design space of resource leak checkers.
%
This section compares our approach with two other modern tools that detect resource leaks:
\begin{itemize}
\item The analysis built into the Eclipse Compiler for Java (ecj), which is the default approach
  for detecting resource leaks in the Eclipse IDE~\cite{ecj-resource-leak}.
\item Grapple~\cite{zuo2019grapple} represents a significant, recent
  improvement in the scalability of typestate-based tools that require a whole-program alias analysis.
\end{itemize}
In brief, both tools are unsound.
Both tools are applicable to legacy code. % , without the need to write annotations.
Eclipse is very fast (nearly instantaneous) and suffers 67--98\% false
positives. % while missing 70-90\% of resource leaks.
According to its authors, Grapple is precise (13\% false positive rate) but orders
of magnitude slower than \Tool.
Different users can select whichever tool matches
their priorities.


\subsubsection{Eclipse}
\label{sec:eclipse}

The Eclipse analysis is a simple dataflow analysis
augmented with heuristics. Because it is tightly integrated with
the compiler, it scales well and runs quickly. It contains
heuristics for ownership, resource wrappers, and resource-free
closeables, among others; these are all hard-coded into the analysis and cannot
be adjusted by the user. It does not support annotations to express
specifications that differ from its defaults.
It supports two levels of analysis: detecting high-confidence resource
leaks and detecting ``potential'' resource
leaks (a superset of high-confidence resource leaks).

We ran this analysis on the same version of Zookeeper's zookeeper-server
module that we ran \Tool on in \cref{sec:case-studies}. It is easy to apply
to legacy code and it is fast---nearly instantaneous once Eclipse
has loaded the project.

In ``high-confidence'' mode on zookeeper-server, Eclipse reports 3
warnings: 1 true positive (thus, it misses 11 real resource leaks) and 2
false positives.
In ``potential'' leak mode, the analysis reported 180
warnings:  3 true positives (it misses 9 real resource leaks) and 177 false
positives.
The most common cause of false
positives was the unchangeable, default ownership transfer assumption
at method invocations: it warned at each call that returns a resource-alias, such as
\<Socket\#getInputStream>.

These results demonstrate that the Eclipse bug-finder is unsound in both
modes: it misses 9 real warnings. Further, the analysis is imprecise: in potential leak mode,
the vast majority of warnings are false positives.
%% MK: I think this sentence is redundant
%% However, this sort
%% of bug-finding tool does have two advantages: it is fast, and it is easily
%% applicable to legacy code.
Compared to our tool, the Eclipse analysis
is much faster and easier to apply to legacy code
(no annotations are required), but both less sound
and less precise.

%% Chandra rightly pointed out that this para make us look lazy. Let's just
%% say what we did, not why we did it.
%%
%% Given that these results on zookeeper-server
%% clearly show the strengths and weaknesses of this tool compared to our own,
%% we did not run it on our other case study programs.

\subsubsection{Grapple}
\label{sec:grapple}

% a line in tab:grapple
% these numbers come from tables 2 and 3 in the grapple paper from EuroSys '19.
% for TPs and FPs, I added together the IO and Socket columns of table 2.
\newcommand{\grappletablerow}[4]{\textbf{\smaller{#1}} & #2 & #3 & #4}

\begin{table}
  \caption{The Grapple tool's performance; reproduced from~\cite{zuo2019grapple}.}
  \label{tab:grapple}
  \posttablecaption
  
  \begin{tabularx}{\columnwidth}{@{}Xrrr@{}}
    Project                              &  TPs    &    FPs         & Run time      \\
    \hline
    \grappletablerow{ZooKeeper}             {6}         {0}           {01h 07m 02s}     \\
    \grappletablerow{HDFS}                  {5}         {2}           {01h 54m 52s}    \\
    \grappletablerow{HBase}                 {15}        {2}           {33h 51m 59s}     \\
    \hline
    \grappletablerow{\textbf{Total}}        {26}        {4}           {-}          \\
  \end{tabularx}
\end{table}

% \todo{Be very sure that Grapple is unsound - read their paper again, and
%    point out why in this paragraph.}
%\todo{be sure to note that Grapple is not modular, unlike us}
Grapple~\cite{zuo2019grapple} is a modern typestate-based resource leak analysis
focused on high precision and (relative) scalability. Grapple models its alias and
dataflow analyses as dynamic transitive-closure computations over graphs, and
leverages novel path encodings and techniques from predecessor-system
Graspan~\cite{wang2017graspan} to achieve both context- and path-sensitivity.  
% Grapple, like all previous
% typestate-based analyses, requires a whole-program alias analysis.
Grapple contains four checkers, of which two are useful for detecting
resource leaks.  Unlike \Tool, Grapple is unsound; e.g., it performs a fixed bounded unrolling
of loops to make path sensitivity tractable.

The Grapple authors have already evaluated their tool on earlier
versions\todo{Why don't we repeat our experiments on those earlier versions?
This should be easy to do since it's only a few hours of work and we are now
familiar with the programs.} of the case study programs in
\cref{sec:case-studies}~\cite{zuo2019grapple}; \Cref{tab:grapple} reproduces the
results from their paper.  Based on these numbers, Grapple reports fewer false
positives than \Tool.  Unfortunately, we were not able to successfully run
Grapple's leak detection, and full details on their true and false-positive
warnings are currently unavailable (we have requested them from the authors), so
we cannot study the precision differences further.  The run times in
\Cref{tab:grapple} show a stark difference with our work; \Tool runs in minutes,
whereas Grapple can take many hours. Further, to our best knowledge, Grapple is
not modular, so any code modification will necessitate a full re-analysis.
\Tool only needs to re-analyze modified code and possibly its dependents after
a change, not unmodified dependencies.

% We requested the analysis and Grapple's output, but as of this writing
% (months later) the Grapple authors have not yet provided them.
% \manu{I think this is strictly-speaking true but a bit harsh.  We
% may want to tone it down a bit.}

\todo{Still keep this?} Note that the TP and FP numbers not perfectly comparable between a modular
and a whole-program analysis.
Our tool reports violations of a user-supplied specification
(which takes effort to write but provides documentation benefits), so it
can ensure that a library is correct for all possible clients.  By
contrast, Grapple checks a library in the context of one specific client.

% LocalWords:  LoC SHAs leakable
