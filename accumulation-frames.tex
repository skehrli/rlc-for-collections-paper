\section{Accumulation frames}
\label{sec:reset-must-call}

This section describes an enhancement to the theory of accumulation
analysis that allows \Tool to verify non-final, owning fields
that contain one or more resources during their lifetime
(\cref{sec:non-final-owning}).
This approach is also useful for
removing another set of false positives caused by unconnected sockets
(\cref{sec:unconnected-sockets}).

\subsection{Non-final, owning fields}
\label{sec:non-final-owning}

The analyses we have described so far assume that a single program
element will control at most one resource while it is in scope:
the analyses always issue an error if a non-final field is assigned an expression
that controls a resource.
In practice, however, non-final fields
may be assigned resources. Consider
the following example:

\begin{lstlisting}[frame=tb,belowskip=3mm]
  @MustCall("close") // default qualifier for uses of SocketContainer
  class SocketContainer {
    private @Owning Socket sock;
    public SocketContainer() { sock = ...; } 
    void close() { sock.close() };
    void reconnect() {
      if (!sock.isClosed()) {
        sock.close();
      }
      sock = ...;
    }
  }
\end{lstlisting}
In the lifetime of a \<SocketContainer> object, it might control any
number of resources, depending on the number of calls to
\<reconnect()>. In order to verify the code above (and uses of the
class), we need \emph{accumulation frames}---distinct regions of the program in
which a program element can be treated as the subject of an
accumulation analysis. At the boundary between each frame, \Tool
1) checks that the obligations of the old frame are fulfilled\todo{What is
  this check?  How can it fail?}, and
2) resets the obligations of the object.

For example, consider the following use of the \<SocketContainer> class
above:

\begin{lstlisting}[frame=tb,belowskip=3mm]
  SocketContainer sc = ...;
  sc.close()
  // frame boundary
  sc.reconnect();
  sc.close();
\end{lstlisting}
This code contains two accumulation frames: the first extends from the
constructor call to the first call to \<reconnect()>, and the second
extends from the call to \<reconnect()> (which creates another
obligation) to the end of the block. Because \<reconnect()> is guaranteed
to clean up the resources that it overwrites---note the call
to \<close()> in its implementation---this code is safe, even though \<sc>
corresponds to two different resources: one for each frame.

We introduce a new annotation, called \<@ResetMustCall>, to delineate
an accumulation frame. \<@ResetMustCall> is written on a method
declaration, and takes one argument (the \emph{target}), which
indicates to which in-scope program element the frame applies. The
definition for \<reconnect()> above would then become:

\begin{lstlisting}[frame=tb,belowskip=3mm]
  class SocketContainer {
    ...
    @ResetMustCall("this")
    void reconnect() { ... }
  }
\end{lstlisting}
When an \<@ResetMustCall>
method is invoked, \Tool removes any
inferred Called Methods information about the target (\<sc> in the example).
This rule means that even if \<close()> has already been called on \<sc>
(as in the example)
when \<reconnect()>
is called, \<close()> must be called again---corresponding to the intuition
that a new accumulation frame ``resets'' the obligations of \<sc>.
The inferred must-call methods for \<sc> could also increase:
if the declared type of \<sc>'s default must-call type includes any
methods that are not in the current must-call type for \<sc>, they are added.
Since \<@ResetMustCall> annotations can only increase its target's obligations,
no verification for them is needed: placing them anywhere in the program is sound
(but may reduce precision).

The Must Call Consistency Checker treats calls to \<@ResetMustCall> obligations
as creating a fresh obligation to check, since they mark the beginning of a new
accumulation frame.  In terms of our \Cref{sec:must-call-invoked} pseudocode,
assume we have a routine \textsc{RMCTargets}$(c)$ that returns a set of all
variables $p_i$ passed to a \<@ResetMustCall> target parameter of the method
invoked by $c$.  Then, we can update the \textsc{FactsFromCall} procedure of
\Cref{alg:helpers} to handle \<@ResetMustCall> as follows:
\begin{algorithmic}
  \Procedure{FactsFromCall}{$s$}
  \State $p \gets \textsc{LHS}(s), c \gets \textsc{RHS}(s)$
  \State \Return $\{ \langle \{ p_i \}, c \rangle\ |\ p_i \in \textsc{RMCTargets}(c) \}$ \newline
  \hspace*{5em} $\cup\ (\textsc{HasObligation}(c)$ ? $\{ \langle \{ p \}, c \rangle \}$ : $\emptyset)$
  \EndProcedure
  \end{algorithmic}
Additionally, if during analysis, we encounter a call to a \<@ResetMustCall>
method on a variable $p \in P$, we treat the \<@MustCall>
obligation of $p$ as \emph{satisfied}.  I.e., we add one more
case to the extended \textsc{MCSatisfiedAfter} predicate given in
\Cref{sec:ownership-transfer} ($[\ldots]$ stands for the cases shown previously):
\begin{algorithmic}
  \Procedure{MCSatisfiedAfter}{$P,s$}
  \State \Return $\exists p \in P .\ [\ldots] \vee p \in \textsc{RMCTargets}(s)$
  \EndProcedure
\end{algorithmic}

% \begin{align*}
%   \exists p \in P &.\ [\ldots] \vee p \in \textsc{RMCTargets}(s)
% \end{align*}
This is sound since the checker creates a new obligation for calls to
\<@ResetMustCall> methods, and the must-call obligation inference ensures the
\<@MustCall> type for the target will have a \emph{superset} of any methods present
before the call.
There is one exception to this consistency check: if an \<@ResetMustCall>
method is invoked within a method that also has an \<@ResetMustCall> annotation
with the same target---thus imposing the obligation on its caller---then
the new obligation can be treated as satisfied immediately.

Finally, the checker must enforce two new rules to ensure that
re-assignments to non-final, owning fields like \<sock> in the example
above are sound:
\begin{itemize}
\item any method that re-assigns a non-final, owning field of an object
  must be annotated with an \<@ResetMustCall> that targets that object.
\item when a non-final, owning field $f$ is re-assigned at statement $s$, its inferred \<@MustCall> obligation
must be contained in its \<@CalledMethods> type at the program point before $s$.
\end{itemize}
\noindent
Together, these rules guarantee that any re-assignment to a non-final, owning
field (1) ends the current accumulation frame with the obligation fulfilled,
and (2) starts a new accumulation frame, with the obligation re-imposed on
the target at the caller.

\subsection{Unconnected sockets}
\label{sec:unconnected-sockets}
Accumulation frames can also handle cases where an object, when created,
does not actually allocate a resource---but it will allocate a resource
later in its lifecycle. Consider, for example, the no-argument constructor
to \<java.net.Socket>. This constructor does not actually allocate an
operating system-level socket, but instead just creates the container
object, which permits the programmer to e.g. set options which will be used
when creating the physical socket. When such a \<Socket> is created, it
initially has no must-call obligation; it is only when the \<Socket> is
actually connected\todo{How does that happen?} that the must-call obligation is created.

If all \<Socket>s are treated as \<@MustCall(\{"close"\})>---that is,
without the accumulation frames described in this section---a false positive
would be reported
in code such as the below, which operates on an unconnected socket
(simplified from real code in Apache Zookeeper~\todo{cite}):

\begin{lstlisting}[frame=tb,belowskip=3mm]
  static Socket createSocket() {
    Socket sock = new Socket();
    sock.setSoTimeout(...);
    return sock;
  }
\end{lstlisting}

The call to \<setSoTimeout> can throw a \<SocketException> if the
socket is actually connected when it is called, so without an
accumulation frame our analysis would report a possible leak along
that path.  However, we can use an accumulation frame \emph{without an
  obligation}---i.e. in which the \<Socket> is \<@MustCall({})>---
that extends from the constructor call to a method that connects the
socket (such as \<connect()> or \<bind()>) to avoid this false
positive, by annotating those methods (in the definition of
\<java.net.Socket>) as \<@ResetMustCall("this")>.  When one of these
methods is then called on the \<Socket> in client code that is being
verified by \Tool, its inferred must-call obligations increase to include
\<close()>, because by default an arbitrary \<Socket>'s type is
\<@MustCall("close")>.

% LocalWords:  belowskip MustCall SocketContainer sc ResetMustCall
% LocalWords:  RMCTargets MCObligations HasMCReturn MCSatisfied
% LocalWords:  createSocket setSoTimeout SocketException
