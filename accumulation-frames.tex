\section{Creating new obligations}
\label{sec:reset-must-call}

% Move this table in the LaTeX source if needed. I (Martin)
% decided that the aesthetic value of this table appearing
% on the page just before the first reference to it is worth
% avoiding having a crowd of figures and tables all on the same
% page.
%%%%%%%%%%%%%%%%%%%%%%%%%%%%%%%%%%%%%%%%%%%%%%%%%%%%%%%
%% This table is here instead of the evaluation section
%% so that it is placed on the appropriate page.
%%%%%%%%%%%%%%%%%%%%%%%%%%%%%%%%%%%%%%%%%%%%%%%%%%%%%%%

% a line in tab:case-studies
% arguments: project name, original LoC, # of resources (-AcountMustCall), diff size, # of annotations, TPs, Confirmed TPs, FPs, run time in seconds
\newcommand{\osstableheader}[8]{\textbf{\smaller{#1}} & \mcoc{#2} & \mcocr{#3} & \mcoc{#4} & \mcocr{#5} & \mcoc{#6} & \mcocr{#7} & \mcoc{#8}}
\newcommand{\osstablerow}[8]{\textbf{\smaller{#1}} & #2 & #3 & #4 & #5 & #6 & #7 & #8}
% "mcoc" stands for ``multi-column, one, centered''.
\newcommand{\mcoc}[1]{\multicolumn{1}{c}{#1}}
% "mcoc" stands for ``multi-column, one, centered, right bar''.
\newcommand{\mcocr}[1]{\multicolumn{1}{c|}{#1}}

\begin{table*}
  \caption{
    Verifying the absence of resource leaks.
    % LoC is computed using scc. Be careful when computing LoC to run scc from relevant top-level
    % directory that only contains source code (not test code!) to make sure that the count is accurate.
    % for Zookeeper, this directory is zookeeper/zookeeper-server/src/main/java
    Throughout, ``LoC'' is lines of non-comment, non-blank Java code.
    ``Resources'' is the number of resources created by the program.
    ``Resource leaks'' are true positive warnings.
    ``False positives'' are where the tool reported a potential leak,
    but manual analysis revealed that no leak is possible.
    % to compute the diff size, do something like the following (this is what I did for Zookeeper) from the with-annotations branch:
    % > git diff origin/with-checker -- '*.java'
    % then count the number of individual changesets and report that
    ``Annotations'' and ``code changes'' are the number of edits to program text; see
    \cref{sec:annotations-and-code-changes} for details. ``Wall-clock time'' is the median
    of five trials.
  }
  \label{tab:case-studies}
  \posttablecaption

  \begin{tabular}{@{}lrc|cc|cc|c@{}}
    \osstableheader{}                                                 {}              {}          {Resource} {False}        {Annota-}  {Code}      {Wall-clock}         \\
    \osstableheader{}                                                 {LoC}           {Resources} {leaks}   {positives}     {tions}    {changes}   {time}              \\
    \hline                                                                                                        
    \osstablerow{apache/zookeeper:zookeeper-server}                   {45,248}        {177}       {13}      {48}            {122}      {\zph 5}    {1m 24s}           \\
    \osstablerow{apache/hadoop:hadoop-hdfs-project/hadoop-hdfs}       {151,595}       {365}       {23}      {49}            {117}      {13}        {16m 21s\zph}          \\
    \osstablerow{apache/hbase:hbase-server, hbase-client}             {220,828}       {\zph 55}   {\zph 5}  {22}        {\zph 45}      {\zph 5}    {7m 45s}           \\
    \osstablerow{plume-lib/plume-util}                                {10,187}        {109}       {\zph 8}  {\zph 2}    {\zzph 2}      {19}        {0m 15s}           \\
    \hline                                                                                                        
    \osstablerow{\textbf{Total}}                                      {427,858}       {706}       {49}      {121}           {286}      {42}        {-}                \\
  \end{tabular}
\end{table*}

% LocalWords:  LoC Annota tions apache hbase util


Every constructor of a class that has must-call obligations
implicitly creates obligations for the newly-created object.
However, non-constructor methods may also create obligations
when re-assigning non-final owning fields or allocating
new system-level resources.
To handle such cases soundly, we introduce a method post-condition annotation,
\CreatesMustCallFor,
to indicate expressions for which an obligation is created at a call.

At each call-site of a method annotated as \CreatesMustCallFor\<(>\|expr|\<)>, \tool removes any
inferred Called Methods information about \|expr|, reverting to
\CalledMethods\<(\{\})>.

When checking a call to a method annotated as
\CreatesMustCallFor\<(>\|expr|\<)>, the Must Call Consistency Checker
(1) treats the \MustCall
obligation of \|expr| as \emph{satisfied},
and (2) creates a fresh obligation to check.
%
We update the \textsc{FactsFromCall} and \textsc{MCSatisfiedAfter} procedures of
\cref{alg:helpers} as follows ($[\ldots]$ stands for the cases shown previously,
including those in \cref{sec:ownership-transfer}):
%\todo{I suggest putting the base code first and the addition with CMCFTargets second.}
\begin{algorithmic}
  \Procedure{FactsFromCall}{$s$}
  \State $p \gets s.LHS, c \gets s.RHS$
  \State \Return $\{ \langle \{ p_i \}, c \rangle\ |\ p_i \in \textsc{CMCFTargets}(c) \}$ \newline
  \hspace*{5em} $\cup\ (\textsc{HasObligation}(c)$ ? $\{ \langle \{ p \}, c \rangle \}$ : $\emptyset)$
  \EndProcedure
  \Procedure{MCSatisfiedAfter}{$P,s$}
  \State \Return $\exists p \in P .\ [\ldots] \vee p \in \textsc{CMCFTargets}(s)$
  \EndProcedure
  \Procedure{CMCFTargets}{$c$}
  \State \Return \{ $p_i$\ |\ $p_i$ passed to an \CreatesMustCallFor target for $c$'s callee \}
  \EndProcedure
\end{algorithmic}
\noindent
This change is sound: the checker creates a new obligation for calls to
\CreatesMustCallFor methods, and the must-call obligation checker (\cref{sec:must-call}) ensures the
\MustCall type for the target will have a \emph{superset} of any methods present
before the call.
There is an exception to this check: if an \CreatesMustCallFor
method is invoked within a method that has an \CreatesMustCallFor annotation
with the same target---imposing the obligation on its caller---then
the new obligation can be treated as satisfied immediately.\looseness=-1

\subsection{Non-final, owning fields}
\label{sec:non-final-owning}

\CreatesMustCallFor allows \tool to verify uses of non-final fields
that contain a resource, even if they are re-assigned. Consider
the following example:

\begin{lstlisting}[frame=tb,belowskip=3mm]
  @MustCall("close") // sets default qual. for uses of SocketContainer
  class SocketContainer {
    private @Owning Socket sock;
    public SocketContainer() { sock = ...; } 
    void close() { sock.close() };
    @CreatesMustCallFor("this")
    void reconnect() {
      if (!sock.isClosed()) {
        sock.close();
      }
      sock = ...;
    }
  }
\end{lstlisting}
In the lifetime of a \<SocketContainer> object, \<sock>
might be re-assigned arbitrarily many times: once at each
call to \<reconnect()>. This code is safe, however: \<reconnect()>
ensures that \<sock> is closed before re-assigning it.

\Tool must enforce two new rules to ensure that
re-assignments to non-final, owning fields like \<sock> in the example
above are sound:
\begin{itemize}
\item any method that re-assigns a non-final, owning field of an object
  must be annotated with an \CreatesMustCallFor annotation
  that targets that object.
\item when a non-final, owning field $f$ is re-assigned at statement $s$,
  its inferred \MustCall obligation must be contained in its \<@CalledMethods>
  type at the program point before $s$.
\end{itemize}
\noindent
The first rule ensures that \<close()> is called after the last call
to \<reconnect()>, and the second rule ensures that \<reconnect()>
safely closes \<sock> before re-assigning it. Because calling
an \CreatesMustCallFor method like \<reconnect()> resets
local type inferece for called methods, calls to \<close>
before the last call to \<reconnect()> are disregarded.

\subsection{Unconnected sockets}
\label{sec:unconnected-sockets}
\CreatesMustCallFor can also handle cases where object creation
does not allocate a resource, but the object will allocate a resource
later in its lifecycle. Consider the no-argument constructor
to \<java.net.Socket>. This constructor does not allocate an
operating system-level socket, but instead just creates the container
object, which permits the programmer to e.g. set options which will be used
when creating the physical socket. When such a \<Socket> is created, it
initially has no must-call obligation; it is only when the \<Socket> is
actually connected via a call to a method such as \<bind()>
or \<connect()> that the must-call obligation is created.

If all \<Socket>s are treated as \MustCall\<(\{"close"\})>,
a false positive would be reported
in code such as the below, which operates on an unconnected socket
(simplified from real code in Apache Zookeeper~\cite{zookeeper-create-socket}):

\begin{lstlisting}[frame=tb,belowskip=3mm]
  static Socket createSocket() {
    Socket sock = new Socket();
    sock.setSoTimeout(...);
    return sock;
  }
\end{lstlisting}

\noindent
The call to \<setSoTimeout> can throw a \<SocketException> if the
socket is actually connected when it is called. Using \CreatesMustCallFor,
however, \tool can soundly show that this socket is not connected:
the type of the result of the no-argument constructor
is \<@MustCall(\{\})>, and \CreatesMustCallFor annotations
on the methods that actually allocate the socket---\<connect()> or
\<bind()>---enforce that as soon as the socket is open,
it is treated as \<@MustCall("close")>.

% LocalWords:  belowskip MustCall SocketContainer sc CreateObligation expr
% LocalWords:  RMCTargets MCObligations HasMCReturn MCSatisfied CMCFTargets
% LocalWords:  createSocket setSoTimeout SocketException FactsFromCall
% LocalWords:  MCSatisfiedAfter CreateObligation isClosed CalledMethods
% LocalWords:  CreatesMustCallFor
