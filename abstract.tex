Resource leaks occur when a programmer-managed resource---like a socket or
database connection in a language like Java, or memory in C---is not properly disposed.
They are notoriously difficult to test for, and can cause serious problems such
as resource starvation, denial-of-service, or even security vulnerabilities.
There exist lightweight verification systems that, if they issue no
warnings about a program, guarantee that that program does not leak any resources.
However, these systems lack support for \emph{collections} of resources---that is,
data structures like arrays, lists, or maps: extant tools will always issue a warning
about storing a resource into such a data structure, because they lack the ability to
prove that those data structures manage the resources properly. This limitation
prevents them from verifying many real-world programs that manage resources in bulk.

We propose a novel extension to prior resource leak analyses that permits them to verify
that collections of resources are properly disposed. Our core approach assumes (and enforces) that
collections of resources are \emph{homogenous}: each element of the collection must call
the same methods to be disposed, and the elements are initialized and disposed en masse.
We extend this core approach with two new kinds of limited alias analysis that are important
for making the analysis useful in practice: a specialized variant of lightweight ownership,
and a novel aliasing logic for iterators over collections that permits collections of resources
to be modified via their iterators safely. We implemented these ideas as an extension of
the Checker Framework's Resource Leak Checker for Java and conducted a set of case studies that shows
their utility in practice \todo{give specific numbers}.

\label{dummy-label-for-etags:abstract}
