\section{Lightweight ownership tracking}
\label{sec:lightweight-ownership}

\todo{Add an example of how @Owning or @NotOwning helps reduce false positives.}

For the basic technique described in \Cref{sec:base-type-systems}, false
positives are often caused by cases where a \<@MustCall> obligation is satisfied
in another procedure, via parameter passing, return values, or object fields.
Here, we describe a \emph{lightweight ownership tracking} technique for reducing
false positives in such cases.  With lightweight ownership, programmer
annotations \emph{cannot} introduce any checker unsoundness; they can only to
reduce false positives.  Further, unlike other ownership type systems~\todo{cite},
lightweight ownership imposes no restriction on what operations can be performed
through an alias, and hence has a minimal impact on the programming model.

% We developed two novel approaches for tracking whether a pointer is
% owning: a lightweight ownership transfer approach using annotations
% that does not depend on the annotations' correctness---that is, adding
% ownership annotations can only reduce false positives, not false
% negatives (\cref{sec:ownership-transfer})---and a lightweight, local
% alias analysis for dealing with wrapper objects
% (\cref{sec:must-call-choice}).

\subsection{Ownership transfer}
\label{sec:ownership-transfer}

% \todo{Write about @Owning and @NotOwning, and show how they can only
%   reduce false positives.}

With lightweight ownership, a parameter, return, or field type may be annotated
as either \<@Owning> or \<@NotOwning> (we shall discuss default assumptions for
these annotations shortly).  In
the Must Call Consistency Checker (\Cref{sec:must-call-invoked}), if a tracked
variable $p$ is assigned to such an \<@Owning> location $l$, we \emph{transfer}
ownership to $l$, and $p$'s \<@MustCall> obligation
is treated as satisfied. Specifically, the condition checked by
\textsc{MCSatisfied}$(P,s)$ of \Cref{alg:helpers} is enhanced as follows
\todo{sloppy, make it better}:

\begin{align*}
  \exists p \in P &.\ \textsc{MCBefore}(p,s) \subseteq \textsc{CMBefore(p,s)}\\
                  &\vee (s \mathrm{\ is\ } \mathtt{return\ p} \wedge \mathrm{method\ return\ type\ is\ } \mathtt{@Owning})\\
                  &\vee (s \mathrm{\ is\ } \mathtt{q\ =\ foo(...,p,...)} \wedge \mathtt{p} \mathrm{\ is\ } \mathtt{@Owning})\\
                  &\vee (s \mathrm{\ is\ } \mathtt{q.f\ =\ p} \wedge \mathtt{f} \mathrm{\ is\ } \mathtt{@Owning})\\
\end{align*}

To ensure soundness, the analysis must ensure that the \<@MustCall> obligation
of any \<@Owning> location is satisfied.  We discuss checking of \<@Owning>
fields in \Cref{sec:owning-fields}.  For \<@Owning> method returns, we need only
update the \textsc{Has\-MCReturn} routine used in \Cref{alg:helpers} to check that
the return type both has a non-empty \<@MustCall> obligation \emph{and} is
\<@Owning>.  For \<@Owning> parameters, the loop at
Lines~\ref{li:start-init}--\ref{li:end-init} in \Cref{alg:consistency-checker}
must be extended to also add each \<@Owning> formal parameter as a dataflow
fact at the entry node of the CFG.  These relatively simple changes are all that
is required to add lightweight ownership tracking to our technique.\todo{but we
didn't discuss fields yet...}

By default, our checker assumes that unannotated return types are \<@Owning>,
but that parameter and field types are \<@NotOwning>.  These assumptions
coincide well with coding patterns we observed in practice, reducing the
annotation burden for programmers.  Further, this treatment of parameter and
return types ensures sound handling of unannotated third-party libraries: any
object returned from such a library will be tracked by default, and the checker
never assumes that passing an object to a library will satisfy its obligation.

Programmers can override defaults by writing explicit \<@Owning> and
\<@NotOwning> annotations on method returns, parameters, and fields
(though never both at the same location---they are
mutually exclusive).  Such explicit annotations can \emph{never}
compromise soundness.  If such an annotation is incorrect (e.g., an \<@Owning>
annotation on a reference that does not satisfy an obligation), it can only lead
to false positives, not false negatives.  Due to this design, there is great
scope for creating inference techniques for lightweight ownership, to further
reduce annotation burden; we plan to pursue such techniques in future work.
\todo{pointer to further discussion in evaluation?}

\todo{Add a para comparing to Rust ownership and the like?}

\subsection{Final owning fields}
\label{sec:owning-fields}

  Additional class-level checking is required for \<@Owning> fields, as the code
  satisfying their \<@MustCall> obligations nearly always spans multiple
  procedures.  We first discuss the case of final fields,\footnote{We only
  consider instance fields in this paper; we have not observed resource leaks
  through static fields in our benchmarks, and we leave their handling to future
  work.} which cannot be overwritten after initialization of the enclosing
  object.  Consider a final \<@Owning> field $f$ declared in class $C$, where
  $f$ has type \<@MustCall("m")>.  To
  verify that $f$'s \<@MustCall> obligation is satisfied, we check the following
  conditions:
  \begin{enumerate}
    \item All $C$ objects must have a type \<@MustCall("n")> for some method $C.n$.
    \item $C.n$ must always invoke \<this.f.m()>, thereby satisfying $f$'s
    \<@MustCall> obligation.
  \end{enumerate}
  In this manner, we compositionally verify the usage of $f$, by imposing
  a \<@MustCall> obligation on clients using the $C$ class.  Condition 2 can be
  checked by requiring an appropriate \<@EnsuresCalledMethods> annotation on
  $C.n$, which is then enforced by the Called Methods
  Checker~\cite{kellogg20verifying}.  \todo{this all needs an example...}

  While this scheme suffices for verifying final fields, we observed many cases
  in our benchmarks of non-final fields with \<@MustCall> obligations.  Checking
  non-final fields requires additional techniques, as the checker must ensure
  that overwriting the field does not lead to a \<@MustCall> violation.  We
  describe our novel handling of non-final fields in \Cref{sec:reset-must-call}.

