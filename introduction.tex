\section{Introduction}
\label{sec:intro}

%% Resource leaks are a classic problem, and extant approaches to preventing
%% them are either too expensive or unsound.

A resource leak occurs when some finite resource managed by the
programmer is not explicitly disposed of. In an unmanaged language
like C, that explicit resource might be memory; in a managed language
like Java, it might be a file descriptor, a socket, or a database
connection.  Resource leaks continue to cause severe failures, even in
modern, heavily-used Java applications~\cite{ghanavati2020memory}.
This state-of-the-practice does not differ much from two decades
ago~\cite{WeimerN04}.
Microsoft engineers consider resource leaks to be one of the most
significant development challenges~\cite{LoNZ2015}.
% The fact that
% resource leaks remain such a serious problem despite decades of
% research and improvements in languages and tooling shows that
Preventing resource leaks remains an urgent, difficult, open problem.

%% An ideal tool for preventing resource leaks would be fast, sound, and precise.

Ideally, a tool for preventing resource leaks would be:
\vspace{-5pt}
\begin{itemize}
\item \emph{applicable} to
  % leakable resources in
  existing code with few code changes,
\item \emph{sound}, so that undetected resource leaks do not slip into
  the program;
\item \emph{precise}, so that developers are not bothered by excessive false positive
  warnings; and
\item \emph{fast}, so that it scales to real-world programs and
  developers can use it regularly.
\end{itemize}
\vspace{-5pt}

%% Extant approaches fail at least one of these: bug finders like the analysis
%% in ecj are unsound; complex analyses based on typestate are either too slow
%% (cite the Eurosys 2019 paper on Grapple, with numbers on run time), unsound, or both.
\noindent
Extant approaches fail at least one of these criteria.
Language-based features may not apply to all uses of resource variables:
Java's try-with-resources statement~\cite{try-with-resources}, for example, can
only close resource types that implement the \<java.lang.AutoCloseable> interface,
and cannot handle
common resource usage patterns that span multiple procedures. 
Heuristic bug-finding tools for leaks, such as those built into Java IDEs including
Eclipse~\cite{ecj-resource-leak} and IntelliJ
IDEA~\cite{idea-resource-leak}, 
are fast and applicable to legacy
code, but they are unsound.
%% They are equally as imprecise as our tool, so don't bad-mouth them.
% and can be highly imprecise (\cref{sec:compare}).
Inter-procedural typestate or dataflow analyses~\cite{TorlakC10,zuo2019grapple}
achieve more precise
%% In our experiments, they found fewer defects, and we didn't compare complexity.
% results---and can find more complex defects than bug-finders,
results---though they usually remain unsound---but
their whole-program analysis can require hours to analyze a large-scale Java program.
Finally, ownership type
systems~\cite{clarke2013ownership} as employed in languages like
Rust~\cite{klabnik2018rust} can prevent nearly all resource leaks (see
\cref{sec:rw-language}), but using them would require a significant rewrite for
a legacy codebase, a substantial task which is often infeasible.
% require
% significantly rewriting the program, 
% and prevents them from being applicable to legacy codebases and coding styles.

The goal of a leak detector for a Java-like language is to ensure that required
methods (such as \<close()>) are called on all relevant objects; we deem this
a \emph{must-call} property.  Verifying a must-call property requires
checking that required methods (or \emph{must-call obligations}) have been
called at any point where an object may become unreachable.  A static
verifier does this by computing an
under-approximation of invoked methods.  Our key insight is that checking of
must-call properties is an \emph{accumulation problem}, and hence does not
require heavyweight whole-program analysis. Our contribution is a resource leak
verifier that leverages this insight to satisfy all four requirements: it is
applicable, sound, precise, and fast.\looseness=-1

An accumulation analysis~\cite{KelloggRSSE2020}
is a special-case of typestate analysis~\cite{StromY86}.
Typestate analysis attaches a finite-state machine (FSM)
to each program element of a given type, and transitions the state of the
FSM whenever a relevant operation is performed.
In an accumulation analysis,
the order of operations performed cannot change what is subsequently
permitted, and executing more operations cannot add additional
restrictions.  Unlike arbitrary typestate analyses, accumulation analyses can
be build in a sound, modular fashion without \emph{any} whole-program alias
analysis, improving scalability and usability.

Recent work~\cite{KelloggRSSE2020} presented an accumulation analysis for
verifying that certain methods are invoked on each object
before a specific call (e.g., \<build()>).
Resource leak checking is similar in that certain methods must be invoked
on each object before it becomes unreachable.
An object becomes unreachable when its references go out
of scope or are overwritten.
By making an analogy between object-unreachability points and
method calls, we show that resource leak checking is an accumulation
problem and hence is amenable to sound, modular, and lightweight analysis.

%% Old version of the above paragraph.
% Recent work~\cite{KelloggRSSE2020} presented an accumulation analysis for
% verifying that certain methods are invoked on each object reference
% before a specific call (e.g., \<build()>).
% Conceptually,
% we can build a resource leak verifier by extending this technique
% to check that required methods are invoked before a reference goes out
% of scope or is overwritten, either of which could cause an object to become unreachable.
% This reduction of possible object-unreachability points to
% method calls shows
% that resource leak checking is an accumulation
% problem and hence is amenable to sound, modular, and lightweight analysis.

There are two key challenges for this leak-checking approach.  First,
due to subtyping, the declared type of a reference may not accurately represent
its must-call obligations; we devised a simple type system to soundly capture
these obligations.  Second, the approach is sound, but highly imprecise (more so than in
previous work~\cite{KelloggRSSE2020}) without targeted reasoning about
aliasing.  The most important patterns to
handle are:
\begin{itemize}
\item copying of resources via parameters and returns, or storing of resources in
final fields (the RAII pattern~\cite{raii});
\item wrapper types, which share their must-call obligations with one of their fields; and,
\item storing resources in non-final fields, which might be lazily initialized or
  written more than once.
\end{itemize}
To address this need,
we introduced an intra-procedural
dataflow analysis for alias tracking, and extended it with three sound
techniques to improve precision:
\begin{itemize}
\item a lightweight ownership transfer system. This system
  indicates which reference is responsible for resolving a must-call
  obligation. Unlike typical ownership type systems, our approach does
  not impact the privileges of non-owning references.
\item resource aliasing, for cases
  %% "in which" is better English, but "when" saves a line.
  % in which
  when
  a resource's must-call obligations
  can be resolved by closing one of multiple references.
\item a system for creating new obligations at locations other than the
  constructor, which allows our system to handle lazy initialization or re-initialization.
\end{itemize}
Variants of some of these ideas exist in previous work.  We bring
them together in a general, modular manner, with full verification and
the ability for programmers to easily extend checking to their own
types and must-call properties.
%
Our approach occupies a novel point in the design space for a leak detector:
unlike most prior work, it is sound; it is an order of magnitude faster than
state-of-the-art whole-program analyses; it has a false positive rate similar
to a state-of-the-practice heuristic bug-finder; and, though it does require manual
annotations from the programmer, its annotation burden is reasonable: about
1 annotation for every 1,500 lines of non-comment, non-blank code. 

Our contributions are:
\begin{itemize}
\item the insight that the resource leak problem is an accumulation
  problem, and
  % a novel set of
  an analysis approach based on
  this fact (\cref{sec:base-type-systems}).
\item three
  % novel
  innovations that improve the precision of our analysis via targeted reasoning about aliasing:
  a lightweight ownership transfer system
  (\cref{sec:lightweight-ownership}), a lightweight resource-alias
  tracking analysis (\cref{sec:must-call-choice}), and a system for
  handling lazy or multiple initialization (\cref{sec:reset-must-call}).
\item an open-source implementation for Java,
  called \tool (\cref{sec:implementation}).
\item an empirical evaluation: case studies on heavily-used
  Java programs (\cref{sec:case-studies}),
  an ablation study that shows the contributions of each innovation to
  \tool's precision (\cref{sec:ablation}), and a comparison to
  other state-of-the-art approaches that demonstrates the unique strengths
  of our approach (\cref{sec:compare}).
%% \item a comparison to alternative approaches to resource leak
%%   checking, that demonstrates that our approach occupies a unique
%%   point in the design space: faster and more usable than heavy-weight
%%   typestate systems, but sound and able to find even subtle bugs,
%%   unlike heuristic bug-finders (\cref{sec:compare}).
\end{itemize}
  

% LocalWords:  unmanaged leakable finalizers RAII
