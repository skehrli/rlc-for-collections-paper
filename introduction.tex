\section{Introduction}
\label{sec:intro}

%% Resource leaks are a classic problem, and extant approaches to preventing
%% them are either too expensive or unsound.

A resource leak occurs when some finite resource managed by the
programmer is not explicitly disposed of. In an unmanaged language
like C, that explicit resource might be memory; in a managed language
like Java, it might be a file descriptor, a socket, or a database
connection.  Resource leaks continue to cause severe bugs, even in
modern, heavily-used Java applications~\cite{ghanavati2020memory};
this state-of-the-practice does not differ much from two decades
ago~\cite{WeimerN04}.  A survey of Microsoft engineers in 2015 found
that they consider resource leak bugs to be one of the most
significant challenges for modern developers~\cite{LoNZ2015}.  That
resource leaks remain such a serious problem despite decades of
research and improvements in languages and tooling shows that
preventing resource leaks remains an urgent, difficult, open problem.

%% An ideal tool for preventing resource leaks would be fast, sound, and precise.

Ideally, a tool for preventing resource leaks would be:
\begin{itemize}
\item \emph{fast}, so that it scales to real-world programs and
  developers can use it regularly;
\item \emph{applicable} to all uses of leakable resources legacy code with few or no code changes,
\item \emph{sound}, so that undetected resource leaks do not slip into
  the program; and
\item \emph{precise}, so that developers are not bothered by false positive
  warnings.
\end{itemize}

%% Extant approaches fail at least one of these: bug finders like the analysis
%% in ecj are unsound; complex analyses based on typestate are either too slow
%% (cite the Eurosys 2019 paper on Grapple, with numbers on run time), unsound, or both.

Extant approaches fail at least one of these criteria.
Language-based features do not apply to all uses of resource variables:
Java's try-with-resources statement, for example, can
only safely close local variables that implement the particular
interface (\<java.lang.AutoCloseable>) that it supports.
The most common other
approach encountered by developers, built into Java IDEs including
Eclipse~\cite{ecj-resource-leak} and IntelliJ
IDEA~\cite{idea-resource-leak} is heuristic bug-finding tools, which
are fast and applicable to legacy code---but either only report the most obvious of
errors or are wildly imprecise (\cref{sec:compare}).
State-of-the-art typestate or dataflow analyses such as
Grapple~\cite{zuo2019grapple} achieve more precise results---and
can find more subtle bugs than bug-finders, though they remain unsound---but
requires hours to analyze even a single real-world Java program, due
to their reliance on expensive whole-program alias analyses.
\todo{Find a good, recent example of a really heavyweight technique to attack here
  for not being applicable and too hard to use.}

Our key insight is that the resource leak problem is an extended
version of an accumulation problem~\cite{kellogg20verifying}. An
accumulation problem is a special-case of typestate problem
in which the order of operations performed cannot change what
is subsequently permitted and executing more operations cannot
add additional restrictions.
If a typestate problem meets these restrictions, then using an accumulation
analysis provides a major benefit: it does
not require a whole-program alias analysis for soundness.\todo{This description
  of accumulation analysis is too wordy; get the restrictions + benefit across
  with less words.}
This insight allows us to build a novel static analysis, based on a
standard type system without a whole-program alias analysis, that is
both sound---meaning it can find many more bugs than heuristic-based
approaches---and much faster than extant typestate-based approaches.
Compared to existing sound approaches, our accumulation-based system
is faster and more precise \todo{Flesh this out with a more direct
  comparison to compare against more "heavyweight" systems like Rust
  ownership or fancier type systems}.

To achieve precision in practice, we needed three
innovations beyond our key insight:
\begin{itemize}
\item a lightweight, heuristic ownership transfer system.
\item a lightweight, specialized must-alias analysis for resources.
\item \emph{accumulation frames}, an extension to the theory of
  accumulation analysis which allow a single program element to
  soundly have an accumulation analysis applied to it multiple times.
\end{itemize}

Our contributions are \todo{line these up with what we actually did,
  experiments-wise}:
\begin{itemize}
\item the insight that the resource leak problem is an accumulation problem,
  and a set of type systems designed to take advantage of this fact (\cref{sec:base-type-systems}).
\item a novel lightweight ownership tracking system (\cref{sec:lightweight-ownership}).
\item a novel lightweight resource-alias tracking analysis (\cref{sec:must-call-choice}).
\item a novel extension to accumulation analysis (\cref{sec:reset-must-call}).
\item an open-source implementation of our type systems as an extension to Java,
  augmented with these innovations, called \tool (\cref{sec:implementation}).
\item case studies that demonstrate that our implementation finds real
  bugs and has few false positives (\cref{sec:case-studies}).
\item an ablation study that demonstrates the contributions of
  each of our improvements to our basic analysis to the low false
  positive rate of our approach (\cref{sec:ablation}).
%% \item a comparison to alternative approaches to resource leak
%%   checking, that demonstrates that our approach occupies a unique
%%   point in the design space: faster and more usable than heavy-weight
%%   typestate systems, but sound and able to find even subtle bugs,
%%   unlike heuristic bug-finders (\cref{sec:compare}).
\end{itemize}
  
