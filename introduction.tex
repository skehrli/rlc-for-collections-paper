\section{Introduction}
\label{sec:intro}

Resource leaks are a classic problem, and extant approaches to preventing
them are either too expensive or unsound.

An ideal tool for preventing resource leaks would be fast, sound, and precise.

Extant approaches fail at least one of these: bug finders like the analysis
in ecj are unsound; complex analyses based on typestate are either too slow
(cite the Eurosys 2019 paper, with numbers on run time), imprecise, or both.

Our key insight is that the resource leak problem is an accumulation
problem~\cite{kellogg20verifying}.

This insight allows us to build a novel static analysis, based on a standard
type system without whole-program alias analysis, that is at least as good
as prior approaches along the three relevant dimensions (fast, sound, precise).

In particular, it beats bug finders in soundness, while still being reasonably
fast and precise. It beats typestate analysis on speed, while matching on
soundness and precision.

To achieve the necessary precision in practice, we needed two innovations
beyond our key insight:
\begin{itemize}
\item a lightweight, heuristic ownership transfer system.
\item the concept of \emph{accumulation frames}, which allow us to
  soundly reset the accumulation status of a single program element
  after showing that its requirements have been fulfilled. \todo{This
    description sucks!}
\end{itemize}

Our contributions are:
\begin{itemize}
\item the insight that the resource leak problem is an accumulation problem,
  and a type system designed to take advantage of this fact (\cref{sec:base-type-systems}).
\item a novel lightweight ownership tracking system (\cref{sec:lightweight-ownership}).
\item a novel extension to accumulation analysis (\cref{sec:reset-must-call}).
\item an implementation of our type system as an extension to Java,
  augmented with these two innovations (\cref{sec:implementation}).
\item case studies that demonstrate that our implementation finds real
  bugs and has few false positives (\cref{sec:case-studies}).
\item an ablation study that demonstrates the contributions of
  lightweight ownership and accumulation frames to the low false
  positive rate of our approach (\cref{sec:ablation}).
\item a comparison to alternative approaches to resource leak
  checking, that demonstrates that our approach occupies a unique
  point in the design space: faster and more usable than heavy-weight
  typestate systems, but sound and able to find even subtle bugs,
  unlike heuristic bug-finders (\cref{sec:compare}).
\end{itemize}
  
