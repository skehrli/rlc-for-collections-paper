\section{Introduction}
\label{sec:intro}

\todo{Make these real paragraphs.}

Resource leaks are bad, but recent work has proposed
lightweight verifiers that seem to actually work. Give some
evidence that people are using the RLC and/or RLC\#.

Lightweight verifiers for this problem are limited because they can't reason
about collections of resources: no prior work supports proving that a collection
of resources is safe. Give some data or evidence to argue that collections
of resources really matter in practice. 

Our core insight is that we're tackling a reduced form of the problem:
treating collections as homogeneous objects, in that all elements are treated
as having the same type. Argue that this restriction makes sense in practice.

Discuss the challenging and novel things that we did, and why our solutions aren't
obvious. Non-exhaustive list:
\begin{itemize}
  \item handling aliasing for collections (I went for rust-style, where there’s always exactly one owner. assigning an @OwningCollection to another takes away ownership from the RHS. Unlike Rust, passing to a function always transfers ownership to the receiver and makes call-site reference write-disabled,...)
\item maintaining homogeneity of collection (solved with “certificed” loops where necessary)
\item handling iterators, which introduce another form of aliasing entirely and require tracking of both iterators and calls to iter.next() and iter.remove()
\end{itemize}

Brag about the fact that we have a working tool, and give some numbers about our experiments.

Our contributions are:
\begin{itemize}
\item novel approach that permits us to soundly check that collections of resources are safely disposed of,
  by maintaining homogeneity (\cref{sec:homogeneity})
\item some new aliasing techniques that are particularly useful for this problem (ownership rules, iterators, anything else?) (\cref{sec:aliasing})
\item open-source implementation of sound resource leak analysis for collections in Java (\cref{sec:implementation})
\item evaluation of our implementation that shows it's great (\cref{sec:evaluation})
\end{itemize}
