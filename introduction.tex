\section{Introduction}
\label{sec:intro}

%% Resource leaks are a classic problem, and extant approaches to preventing
%% them are either too expensive or unsound.

A resource leak occurs when some finite resource managed by the
programmer is not explicitly disposed of. In an unmanaged language
like C, that explicit resource might be memory; in a managed language
like Java, it might be a file descriptor, a socket, or a database
connection.  Resource leaks continue to cause severe bugs, even in
modern, heavily-used Java applications~\cite{ghanavati2020memory}.
This state-of-the-practice does not differ much from two decades
ago~\cite{WeimerN04}. \todo{If we know of any more studies that show resource
  leaks are a big problem, we should add citations to them here, too.}
A survey of Microsoft engineers in 2015 found
that they consider resource leak bugs to be one of the most
significant challenges for modern developers~\cite{LoNZ2015}.  That
resource leaks remain such a serious problem despite decades of
research and improvements in languages and tooling shows that
preventing resource leaks remains an urgent, difficult, open problem.

%% An ideal tool for preventing resource leaks would be fast, sound, and precise.

Ideally, a tool for preventing resource leaks would be:
\begin{itemize}
\item \emph{fast}, so that it scales to real-world programs and
  developers can use it regularly;
\item \emph{applicable} to leakable resources in existing code with few or no code changes,
\item \emph{sound}, so that undetected resource leaks do not slip into
  the program; and
\item \emph{precise}, so that developers are not bothered by excessive false positive
  warnings.
\end{itemize}

%% Extant approaches fail at least one of these: bug finders like the analysis
%% in ecj are unsound; complex analyses based on typestate are either too slow
%% (cite the Eurosys 2019 paper on Grapple, with numbers on run time), unsound, or both.

Extant approaches fail at least one of these criteria.
Language-based features may not \emph{apply} to all uses of resource variables:
Java's try-with-resources statement, for example, can
only close resource types implementing the \<java.lang.AutoCloseable> interface,
and cannot handle
common resource usage patterns that span multiple procedures.
Heuristic bug-finding tools for leaks, built into Java IDEs including
Eclipse~\cite{ecj-resource-leak} and IntelliJ
IDEA~\cite{idea-resource-leak}, 
are \emph{fast} and \emph{applicable} to legacy
code---but they are \emph{unsound} and can be highly \emph{imprecise}
(\Cref{sec:compare}).  
Inter-procedural typestate or dataflow analyses~\cite{TorlakC10,zuo2019grapple}
achieve more \emph{precise}
results---and
can find more complex bugs than bug-finders, though they usually remain
\emph{unsound}---but
they can require hours to analyze a large-scale modern Java program, due
to the costliness of precise whole-program analysis.  \todo{Find a good, recent example of a really heavyweight technique to attack here
for not being applicable and too hard to use.} \todo{Does this work? --Manu} Finally, ownership type
systems~\cite{clarke2013ownership} as employed in languages like Rust
\todo{cite} can prevent nearly all resource leaks, but they cannot easily be
\emph{applied} to existing large Java code bases due to the additional restrictions
they impose.

Our key insight is that the resource leak problem is an extended
version of an accumulation problem~\cite{KelloggRSSE2020}. An
accumulation problem is a special-case of typestate problem~\cite{StromY86}:
one which can be solved by attaching a finite-state machine (FSM)
to each program element, and transitioning the state of each
FSM whenever a relevant operation is performed---such an analysis
is called a typestate analysis.
Accumulation analysis is a type state analysis in which
the order of operations performed cannot change what is subsequently
permitted and executing more operations cannot add additional
restrictions.  If a typestate problem meets these restrictions, then
using an accumulation analysis provides a major benefit: it does not
require a whole-program alias analysis for soundness.
This insight allows
us to build a novel static analysis, based on a standard type system
without a whole-program alias analysis, that is both sound\footnote{Mostly
  sound~\cite{livshits2015defense}; see
\cref{sec:threats} for the usual implementation caveats.}---meaning it
can find many more bugs than heuristic-based approaches---and much
faster than extant typestate-based approaches.  Compared to existing
sound approaches, our accumulation-based system is faster and more
precise \todo{Flesh this out with a more direct comparison to compare
  against more "heavyweight" systems like Rust ownership or fancier
  type systems}.

To achieve precision in practice, we needed three
innovations beyond our key insight:
\begin{itemize}
\item a lightweight, heuristic ownership transfer system.
\item a lightweight, specialized must-alias analysis for resources.
\item \emph{accumulation frames}, an extension to the theory of
  accumulation analysis which allow a single program element to
  soundly have an accumulation analysis applied to it multiple times.
\end{itemize}

Our contributions are:
\begin{itemize}
\item the insight that the resource leak problem is an accumulation
  problem, and a novel set of type systems designed to take advantage
  of this fact (\cref{sec:base-type-systems}).
\item three novel innovations that improve the precision of our type
  systems: a lightweight ownership tracking system
  (\cref{sec:lightweight-ownership}), a lightweight resource-alias
  tracking analysis (\cref{sec:must-call-choice}), and an extension to
  accumulation analysis (\cref{sec:reset-must-call}).
\item an open-source implementation of our type systems as an extension to Java,
  augmented with these innovations, called \tool (\cref{sec:implementation}).
\item an extensive empirical evaluation: case studies on heavily-used
  Java programs that handle many resources (\cref{sec:case-studies}),
  an ablation study that shows the contributions of each innovation to
  \tool's precision (\cref{sec:ablation}), and a comparison to
  other state-of-the-art approaches that demonstrates the unique strengths
  of our approach (\cref{sec:compare}).
%% \item a comparison to alternative approaches to resource leak
%%   checking, that demonstrates that our approach occupies a unique
%%   point in the design space: faster and more usable than heavy-weight
%%   typestate systems, but sound and able to find even subtle bugs,
%%   unlike heuristic bug-finders (\cref{sec:compare}).
\end{itemize}
  
