\section{Background on pluggable types}
\label{sec:background}

\Cref{sec:must-call,sec:called-methods} describe
\emph{pluggable type systems}~\cite{FosterFFA99}
that are layered on top of the type system of the host
language.  Types in a pluggable type system are composed of two parts:
a \emph{type qualifier} and a base type. The type qualifier is the
part of the type that is unique to the pluggable type system; the base
type is a type from the host language. Our implementation is for Java
(see \cref{sec:implementation}), so we use the Java syntax for type
qualifiers: ``\<@>'' before a type indicates that it is a type
qualifier, and a type without ``\<@>'' is a base type.
This paper sometimes omits the basetype when it is obvious from context.\looseness=-1

A type system checks programmer-written types.  Our system requires the
programmer to write types on method signatures, but within method bodies it
uses flow-sensitive type refinement, a dataflow analysis that performs type
inference.  This permits an expression to have different types on different
lines of the program.


% the technical sections about the core accumulation systems: called methods,
% must-call, and the must call invoked worklist algorithm.

\section{Leak detection via accumulation}
\label{sec:base-type-systems}

This section presents a sound, modular, accumulation-based
resource leak checker (``\tool'').
\Cref{sec:lightweight-ownership,sec:reset-must-call,sec:must-call-choice}
soundly enhance its precision.

\Tool is composed of three cooperating analyses:
\todo{This is where the structure finally becomes clear, but this was too
  late for some of the referees.}
\begin{enumerate}
\item a taint tracking type system (\cref{sec:must-call}) computes a conservative
  \emph{overapproximation} of the set of methods that might need to be called
  on each expression in the program.
\item an accumulation type system (\cref{sec:called-methods}) computes
  a conservative \emph{underapproximation} of the set of methods that are
  actually called on each expression in the program.
\item a dataflow analysis (\cref{sec:must-call-invoked}) checks consistency of the results of the two
  above-mentioned type systems.  It issues an error if some
  method that might need to be called on an expression is not always invoked before the
  expression goes out of scope.
\end{enumerate}

\begin{figure}
  \lstinputlisting{simplesocket.txt}
  \prefigcaption
  \caption{A safe use of a \<Socket> resource.}
  \label{fig:example}
\end{figure}

% \noindent
% This section uses \cref{fig:example} as a motivating example.
% It shows
% a safe use of a \<Socket>---a resource that must be closed before
% it is deallocated.


\subsection{A type system for must-call obligations}
\label{sec:must-call}

\begin{figure}

\begin{tikzpicture}[->, shorten >= 1pt, auto, node distance=0.3cm]
  \tikzstyle{every state}=[fill=none,draw=none,text=black, minimum size = 0.5cm, shape = rectangle]
  
  \node[state]        (TOP)                                   {\parbox{4cm}  {\centering \small \<@MustCallUnknown> $= \top$}};
  \node[state]         (MCAB)      [below   = of TOP]        {\parbox{4cm}  {\centering \small \<@MustCall(\{"a", "b"\})>}};
  \node[state]         (MCA)    [below  = of MCAB, xshift=-2.2cm]         {\parbox{4cm} {\centering \small \<@MustCall(\{"a"\})>}};
  \node[state]         (MCB)    [below  = of MCAB, xshift=2.2cm]         {\parbox{4cm} {\centering \small \<@MustCall(\{"b"\})>}};
  \node[state]         (BOT)     [below      = of MCAB, yshift=-0.8cm]       {\parbox{4cm}  {\centering \small \<@MustCall(\{\})> $= \bot$}};

  \path

  (MCAB)        edge        node {} (TOP)
  
  (MCA)         edge        node {} (MCAB)
  (MCB)         edge        node {} (MCAB)
  
  (BOT)         edge        node {} (MCA)
  (BOT)         edge        node {} (MCB)

  ;

\end{tikzpicture}
% \prefigcaption
\caption{Part of the \<MustCall> type hierarchy for representing which methods must be
  called; the full hierarchy is a
  lattice of arbitrary size.
  If an expression's type has qualifier \<@Must\-Call(\{"a", "b"\})>, then
  the methods ``\<a>'' and ``\<b>'' might need to be called before the
  expression is deallocated.
  Arrows represent
  subtyping relationships.
}
\label{fig:must-call-hierarchy}
\end{figure}

\todo{Mike added ``---and our entire analysis---'' to the below paragraph.}

The Must Call type system tracks which methods might need to be called
on a given object before the object is deallocated.  This type system---and
our entire analysis---is not specific to resource leaks. Another such
  property is that the
  \<build()> method of a builder~\cite{designpatterns} should always
  be called.

The Must Call type system supports two qualifiers: \MustCall and
\MustCallUnknown. The \MustCall qualifier's arguments are the
methods that
% "the annotated value" is inaccurate (it is accurate to say "the value of the
% expression with that annotated type"), but is probably clear enough.
the annotated value
must call. The declaration
\MustCall\<(\{"a"\}) Object obj> means that before \<obj> is
deallocated, \<obj.a()> might need to be called.
\Tool conservatively requires all these methods to be called,
and it issues a warning if they are not.

For example, consider \cref{fig:example}. The expression \<null> has type
\MustCall\<(\{\})>---it has no obligations
to call particular methods---so \<s> has that type after its initialization.
The \<new> expression has type \MustCall\<("close")>, and therefore
\<s> has that type after the assignment.
At the start of the \<finally> block, where both values for \<s> flow,
the type of \<s> is their least upper bound, which is \MustCall\<("close")>.

% Note that the type \MustCall\<("close")>
% can represent anything that \emph{might} need to
% call \<close()>: for example, at the entrance to
% the \<finally> block in \cref{fig:example}, \<s>'s
% actual value might either be \<null>, which does not
% need to call any methods, or an open \<Socket>, which does.
% Thus, either the obligation to close or no obligation at all
% can be represented by the static
% type \MustCall\<(\{"close"\}) Socket>, which can be read as ``a
% Socket that might need to call close before it is deallocated''.

Part of the type hierarchy appears in \cref{fig:must-call-hierarchy}.
All types are subtypes of \MustCallUnknown.
The subtyping relationship for \MustCall type qualifiers is:
\trule{A \subseteq B}{\MustCall\<(A)> \sqsubseteq \MustCall\<(B)>}
The default type qualifier is \MustCall\<(\{\})> for base types without a
programmer-written type qualifier.\footnote{For unannotated local variable types,
  flow-sensitive type refinement infers a qualifier.}
% By writing another type qualifier on the declaration of a class, a
% user can specify a different default for raw types of that class.
Our implementation
provides JDK annotations which require that every 
\<Closeable> object must have the \<close()> method called before
it is deallocated.  \todo{Should we add one sentence saying this can be
overridden in a subclass like \<ByteArrayOutputStream>?  It's a nice aspect of
our approach.}

% also provides an ``inheritable'' version of the \MustCall annotation,
% which allows us to annotate a class (or interface) and all of its subtypes.
% For example, we use such an annotation on \<java.io.Closeable> to indicate
% that all \<Closeable> objects must have the \<close()> method called before
% they are deallocated.

%% This is redundant, I think.
% A benefit of using a type system for tracking must-call obligations is
% that \tool can use local type inference to refine the obligations of
% a particular variable. For example, if the only assignment to \<s>
% in \cref{fig:example} were the initial assignment to \<null>, then
% \tool would be able to determine that \<s> does not have a must-call
% obligation.


\subsection{A type system for called methods}
\label{sec:called-methods}

% \todo{Mike changed ``our types system for inferring'' to ``our type system
%   tracks''.  Let's lay off use of ``infer'' because it implies we have
%   written a type inference tool, when what we have really written is a
%   type-checking tool (with local type inference).}

The Called Methods type system tracks a conservative underapproximation of which methods have been called on an object.
It is an extension of a similar system
from prior work~\cite{KelloggRSSE2020}.  The primary difference in our
version is that a method is considered called even if it throws an
exception---a necessity in Java because the \<close()> method
in \<java.io.Closeable> is specified to possibly throw an \<IOException>.
In the prior work, a method was only considered ``called'' when it terminated
successfully.
The remainder of this section is a brief summary
of the prior work~\cite{KelloggRSSE2020}.

% This checker is an \emph{accumulation analysis}: a special case
% of typestate analysis~\cite{StromY86} in which:
% \begin{enumerate}
% \item the order in which operations are performed cannot affect what is
%   subsequently permitted, and
% \item executing more operations does not add restrictions---that is,
%   the set of permitted operations after executing some operation is always
%   a superset of the set of permitted operations before executing that operation.
% \end{enumerate}
% This special case of typestate analysis is useful because it does not
% require a whole-program alias analysis for soundness, and instead
% can be implemented as a standard pluggable type system.

The checker is an accumulation analysis whose accumulation qualifier is \<@CalledMethods>.
The type \<@CalledMethods(>$A$\<) Object>
represents an object on which the methods in the set $A$ have definitely
been called; other methods not in $A$ might also have been called.
The subtyping
rule is:
\trule{B \subseteq A}{\<@CalledMethods(A)> \sqsubseteq \<@CalledMethods(B)>}
The top type is \<@CalledMethods(\{\})>.
The qualifier \CalledMethodsBottom is a subtype of every \<@CalledMethods> qualifier.

Thanks to flow-sensitive type refinement,
Called Methods types are inferred within method bodies.
In \cref{fig:example} the type of \<s> is initially \<@CalledMethods(\{\})>,
but it transitions to \<@CalledMethods("close")> after the call to \<close>.


\subsection{Consistency checking}
\label{sec:must-call-invoked}

\begin{algorithm}[t]
  \caption{Finding unfulfilled \MustCall obligations in a method.
    \Cref{alg:helpers} defines helper functions.
    \todo{Manu, ``$\langle N, e \rangle$'' should be ``$\{ \langle N, e \rangle \}$''.}
  }
  \label{alg:consistency-checker}
  \begin{algorithmic}[1]
  \Procedure{FindMissedCalls}{$CFG$}
  \State \textit{// D maps each statement s to a set of dataflow facts reaching}
  \State \textit{// s.  Each fact is of the form $\langle P, e \rangle$, where P is a set of variables}
  \State \textit{// that must-alias e and e is an expression with a nonempty}
  \State \textit{// must-call obligation.}
  \State $D \gets \textsc{InitialObligations}(CFG)$ \label{li:call-initial-obs}
  \While{$D$ has not reached fixed point}
    \For{$s \in \mathit{CFG.statements}$, $\langle P, e \rangle \in D(s)$}
    \If{$s$ is $\mathit{exit}$} \label{li:end-scope}
      \State report a must-call violation for $e$
    \ElsIf{$\neg \textsc{MCSatisfiedAfter}(P,s)$} \label{li:check-satisfied}
%    \State // \textit{propagate to successors}
    \State $kill \gets $ $s$ assigns a variable ? $\{s.LHS\}$ : $\emptyset$ \label{li:compute-kill}
    \State $gen \gets \textsc{CreatesAlias}(P,s)$ ? $\{s.LHS\}$ : $\emptyset$ \label{li:compute-gen} 
    % \If{$s$ is \lstinline{p = q} and $q \in P$}
    % \State $gen \gets \{p\}$
    % \EndIf
    \State $N \gets (P - kill) \cup gen$ \label{li:compute-new-mc-aliases}
 %   \State // \textit{we even propagate }$\emptyset$\textit{; error reported at exit}
    \State $\forall t \in \mathit{CFG.succ}(s)\ .\ D(t) \leftarrow D(t) \cup \langle
    N, e \rangle$ \label{li:prop-to-succs}
    \EndIf
    \EndFor
  \EndWhile \label{li:alg-loop-end}
  \EndProcedure
  \Procedure{InitialObligations}{$CFG$}
  \State $D \gets \{ s \mapsto \emptyset\ |\ s \in \mathit{CFG.statements} \}$\label{li:start-init}
  \For{$p \in \mathit{CFG.formals}$, $t \in \mathit{CFG.succ}(\mathit{CFG.entry})$}\label{li:init-formals}
    \If{$\textsc{HasObligation}(p)$}
      \State $D(t) \gets D(t) \cup \lbrace \langle \{p\}, p \rangle \rbrace$
    \EndIf
  \EndFor \label{li:end-init-formals}
  \For{$s \in \mathit{CFG.statements}$ of the form \lstinline{p = m(p1, p2, ...)}} \label{li:init-calls}
    \State $\forall t \in \mathit{CFG.succ}(s)\ .\ D(t) \leftarrow D(t) \cup \textsc{FactsFromCall}(s)$
  \EndFor  \label{li:end-init}
  \State \Return $D$
  \EndProcedure
  \end{algorithmic}
\end{algorithm}


\begin{algorithm}[h]
  \caption{Helper functions for \cref{alg:consistency-checker}.  Except for
  \textsc{MCAfter} and \textsc{CMAfter}, all functions will be replaced with
  more sophisticated versions in
  \cref{sec:lightweight-ownership,sec:must-call-choice,sec:reset-must-call}.
  \todo{Ensure the two algorithms appear on the same page.}
}
  \label{alg:helpers}
  \begin{algorithmic}[1]
  \State // \textit{Does e introduce a must-call obligation to check?}
  \Procedure{HasObligation}{$e$}
  \State \Return $e$ has a declared \MustCall type
  \EndProcedure
  \State // \textit{s must be a call statement p = m(p1, p2, ...)}
  \Procedure{FactsFromCall}{$s$}
  \State $p \gets s.LHS, c \gets s.RHS$
  \State \Return $\textsc{HasObligation}(c)$ ? $\{ \langle \{ p \}, c
  \rangle \}$ : $\emptyset$
  \EndProcedure
  \State // \textit{Is the must-call obligation for P satisfied after
    s?}
  \Procedure{MCSatisfiedAfter}{$P,s$}
  \State \Return $\exists p \in P .\ \textsc{MCAfter}(p,s) \subseteq \textsc{CMAfter}(p,s)$
  \EndProcedure
  \State // \textit{Does s introduce a must alias for a var in P?}
  \Procedure{CreatesAlias}{$P,s$}
    \State \Return $\exists \mathtt{q} \in P\ .\ s$ is of the form \texttt{p = q}
  \EndProcedure
  \Procedure{MCAfter}{$p,s$}
  \State \Return methods in \MustCall type of $p$ after $s$
  \EndProcedure
  \Procedure{CMAfter}{$p,s$}
  \State \Return methods in \<@CalledMethods> type of $p$ after $s$
  \EndProcedure
  \end{algorithmic}

\end{algorithm}


Given \MustCall and \<@CalledMethods> types, the Must
Call Consistency Checker ensures that the \MustCall methods for each object
are always invoked before it becomes unreachable,
via an intra-procedural dataflow analysis.  Here, we present
a simple, sound version of the analysis, with limited reasoning about aliasing.
\Cref{sec:lightweight-ownership,sec:must-call-choice,sec:reset-must-call}
describe sound enhancements to this basic approach.

\paragraph{Language} For simplicity, we present the analysis over a simple
assignment language in three-address form.  An expression $e$ in the language is
\<null>, a variable \<p>, a field read \<p.f>, or a method call \<m(p1,p2,\ldots)> (constructor
calls are treated as method calls).  A statement $s$ takes one of three forms:
\<p = e>, where $e$ is an expression; \<p.f = p'>, for a field write; or
\<return p>.  Methods are represented by a control-flow graph (CFG) where nodes
are statements and edges indicate possible control flow.  We elide control-flow
predicates because the consistency checker is path-insensitive.

% \manu{Martin, is the
% must-call checker path-sensitive?  As in, does it interpret null checks?}
% Martin's response: the Must Call Checker is path sensitive in the sense that
% it's flow sensitive and the type of null is bottom. It doesn't have any
% special logic to handle null checks, but it ``does the right thing''
% anyway.

For a method CFG, $\mathit{CFG.statements}$ is the statements,
$\mathit{CFG.formals}$ is the formal parameters,
$\mathit{CFG.entry}$ is its entry node,
$\mathit{CFG.exit}$ is its exit node, and
$\mathit{CFG.succ}$ is its successor relation.
For a statement
$s$ of the form \<p = e>, $s.LHS = p$ and $s.RHS = e$.

\paragraph{Pseudocode} \Cref{alg:consistency-checker} gives pseudocode for the
basic version of our checker, with helper functions in \cref{alg:helpers}.  At a
high level, the dataflow analysis computes a map $D$ from each statement $s$ in
a CFG to a \emph{set} of facts of the form $\langle P, e \rangle$, where $P$ is
a set of variables and $e$ is an expression.  The meaning
of $D$ is as follows: if $\langle P, e \rangle
\in D(s)$, then $e$ has a declared \MustCall type, and all variables in $P$
are \emph{must aliases} for the value of $e$ at the program point before $s$.
Computing a set of must aliases is useful since any must alias may be used to
satisfy the must-call obligation of $e$.  Using $D$, the analysis finds any $e$
that does not have its \MustCall obligation fulfilled, and reports an error.

\Cref{alg:consistency-checker} proceeds as follows.  \Cref{li:call-initial-obs}
invokes \textsc{InitialObligations} to initialize $D$.  Only formal parameters
or method calls can introduce obligations to be checked (reads of local
variables or fields cannot).
% \textsc{InitialObligations}
% (lines~\ref{li:start-init}--\ref{li:end-init}) scans for such expressions and
% adds initial facts as appropriate, using helper functions \textsc{HasObligation}
% and \textsc{FactsFromCall} from \cref{alg:helpers}.
The fixed-point loop
iterates over all facts $\langle P, e \rangle$ present in any
$D(s)$ (our implementation uses a worklist for efficiency).  If $s$ is the exit
node (\cref{li:end-scope}), the obligation for $e$ has not been satisfied, and
an error is reported.  Otherwise, the algorithm checks if the obligation for $e$
is satisfied after $s$ (\cref{li:check-satisfied}).  For the basic checker,
\textsc{MCSatisfiedAfter} in \cref{alg:helpers} checks whether there is some $p
\in P$ such that after $s$, the set of methods in $p$'s \MustCall type are contained
in the set of methods in its
\<@CalledMethods> type; if true, all \MustCall methods have already been
invoked.  This check uses the inferred flow-sensitive \MustCall and
\<@CalledMethods> qualifiers described in
\cref{sec:must-call,sec:called-methods}.
%% Mike doesn't see a way to resolve this comment.  Once MCAfter and
%% CMAfter are defined in the algorithm, that will tip off readers that
%% they are intended to be different and are not a typo.
% \todo{The names CMAfter and
% MCAfter are confusing - they're nearly identical. Can we think of better ones
% that are equally useful?} \manu{I don't have an idea that is also terse, but I
% am open to one.}

If the obligation for $e$ is not yet satisfied, the algorithm propagates the
fact to successors with an updated set $N$ of must aliases.  $N$ is computed in
a standard gen-kill style on
lines~\ref{li:compute-kill}--\ref{li:compute-new-mc-aliases}.  The kill set
simply consists of whatever variable (if any) appears on the left-hand side of
$s$.  The gen set is computed by checking if $s$ creates a new must alias for
some variable in $P$, using the \textsc{CreatesAlias} routine.  Since our
analysis is accumulation, \textsc{CreatesAlias} could simply return false
without impacting soundness.  In \cref{alg:helpers}, \textsc{CreatesAlias}
handles the case of a variable copy where the right-hand side is in $P$.
(\Cref{sec:must-call-choice} presents more sophisticated handling.) Finally,
\cref{li:prop-to-succs} propagates the new fact to successors.  The process
continues until $D$ reaches a fixed point.

\begin{figure}
  \begin{minipage}{0.4\columnwidth}
    \begin{lstlisting}
s = new Socket(...); // 1
if (...) {
  s = null; // 2
} else {
  t = s; // 3
  close(t); // 4
}
\end{lstlisting}
  \end{minipage}
  \begin{minipage}{0.55\columnwidth}
  \includegraphics[width=\linewidth,keepaspectratio]{cfg-example.pdf}
  \end{minipage}
  \prefigcaption
  \vspace{-5pt}
  \caption{Example code and CFG for illustrating \cref{alg:consistency-checker}.
    ``\<e>'' is ``\<new Socket(...)>''.
    Non-shaded facts are created by
    %% Commented out to save a line
    % procedure
    \textsc{InitialObligations}, and
    shaded facts are propagated by the fixed-point loop.
} \label{fig:cfg-example}
\end{figure}

\paragraph{Example} To illustrate our analysis, \cref{fig:cfg-example} shows a
simple program (irrelevant details elided) and its corresponding CFG.
The CFG shows the dataflow facts propagated along each edge.
For initialization, statement 1 introduces the fact $\langle \{ s \}, e
\rangle$ (where $e$ is the \<new Socket(...)> call) to $D(2)$ and $D(3)$.  At
statement 2, $s$ is killed, causing $\langle \emptyset , e \rangle$ to be added
to $D(\mathit{exit})$.  This leads to an error being reported for statement 1, as
the socket is not closed on this path.  Statement 3 creates a must alias $t$ for
$s$, causing $\langle \{ s, t \}, e \rangle$ to be added to $D(4)$.  For
statement 4, $\textsc{MCSatisfiedAfter}(\{s,t\},\mathtt{close(t)})$ holds, so no
facts are propagated from 4 to $\mathit{exit}$.

% At this point, we have described a sound checker for \MustCall obligations.
% % But for real code, this checker emits too many false positives.  In particular,
% % the dataflow analysis is purely intra-procedural, and it will report errors in
% % cases where obligations are satisfied via parameter passing, returns, or fields.
% Subsequent sections make the checker more precise.

% The Ownership Checker is a simple worklist algorithm that operates over the CFG.
% It maintains a set of owning pointers to objects, using a set of simple
% ownership rules:
% \begin{itemize}
% \item a newly-allocated object is owned
% \item the value returned by any method called is owned
% \end{itemize}

% These rules guarantee that there is at least one owning pointer to each
% object that might contain a resource. \Cref{sec:lightweight-ownership}
% gives more details on how ownership is transferred.

% When an owning pointer goes out of scope, the Ownership Checker
% compares the types computed by the Must Call Checker and the Called
% Methods Checker to determine if the requirements on the expression
% going out of scope have been fulfilled using the following process,
% supposing that some expression \<expr> is going out of scope at
% program point $P$:
% \begin{enumerate}
%   \item The Ownership Checker requests a Must Call type from the Must
%     Call Checker for \<expr> at $P$. Suppose this type is
%     \MustCall\<(>$A$\<)> for some set of methods $A$ (if the type is
%     \MustCallUnknown, the Ownership Checker always issues an
%     error).
%   \item The Ownership Checker requests a Called Methods type from the
%     Called Methods Checker for \<expr> at $P$. Suppose this type is
%     \<@CalledMethods(>$B$\<)> for some set of methods $B$.
%   \item The Ownership Checker compares the sets $A$ and $B$. If
%     $A \supset B$, the Ownership Checker issues an error.
% \end{enumerate}

% The Ownership Checker also does a simple, intra-procedural must-alias
% analysis, to avoid issuing duplicate errors for e.g. constructor
% invocations that are assigned to local variables. At most one error
% for each must-alias set is ever issued. 

% LocalWords:  simplesocket MustCallUnknown MCAB MustCall MCA xshift MCB
% LocalWords:  yshift Closeable CalledMethods CalledMethodsBottom IsExit
% LocalWords:  FindMissedCalls MCSatisfied CreatesAlias MCObligations p'
% LocalWords:  HasMCReturn EndOfScope CMBefore MCBefore basetype MCAfter
% LocalWords:  CMAfter HasObligation FactsFromCall MCSatisfiedAfter intra
% LocalWords:  InitialObligations worklist xleftmargin makeSocket addr
% LocalWords:  ByteArrayOutputStream
