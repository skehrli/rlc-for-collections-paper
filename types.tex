% the technical sections about the core accumulation systems: called methods,
% must-call, and the must call invoked worklist algorithm.

\section{Accumulation type systems for resource leak detection}
\label{sec:base-type-systems}

\tool is composed of two cooperating type systems---a taint tracking
type system tracking which methods must be called (the Must Call
Checker of \cref{sec:must-call}) and an accumulation type system
tracking which methods have been called (the Called Methods Checker of
\cref{sec:called-methods})---and a dataflow analysis that tracks
owning pointers and compares the results of the two type systems when
an owning pointer goes out of scope (the Ownership Checker of
\cref{sec:must-call-invoked}). This section describes the basics of
these analyses; further details on novel features are provided in
\cref{sec:lightweight-ownership} and \cref{sec:reset-must-call}.

These analyses work in three stages:
\begin{enumerate}
\item the Must Call Checker (\cref{sec:must-call}) computes a conservative
  \emph{overapproximation} of the set of methods that might need to be called
  on each expression in the program.
\item the Called Methods Checker (\cref{sec:called-methods}) computes
  a conservative \emph{underapproximation} of the set of methods that are
  actually called on each expression in the program.
\item the Ownership Checker (\cref{sec:must-call-invoked}) computes a set
  of owning pointers that contains at least one element for every resource.
  For each owning pointer, the Ownership Checker compares the results
  of the Must Call Checker and the Called Methods Checker at each program
  point at which the pointer might go out of scope, and issues an error
  if any method reported by the Must Call Checker was not reported as
  definitely called by the Called Methods Checker.
\end{enumerate}

\subsection{Background on type qualifiers}
\label{sec:background}

Both the Must Call Checker and the Called Methods Checker are
\emph{pluggable type systems}~\cite{FosterFFA99}---specialized type
systems that are layered on top of the type system of the host
language.  Types in a pluggable type system are composed of two parts:
a \emph{type qualifier} and a base type. The type qualifier is the
part of the type that is unique to the pluggable type system; the base
type is a type from the host language. Our implementation is in Java
(see \cref{sec:implementation}), so we use the Java syntax for type
qualifiers: an \<@> symbol before a type indicates that it is a type
qualifier (a type without an \<@> symbol is a base type).

\subsection{A type system for must-call obligations}
\label{sec:must-call}

\begin{figure}

\begin{tikzpicture}[->, shorten >= 1pt, auto, node distance=0.3cm]
  \tikzstyle{every state}=[fill=none,draw=none,text=black, minimum size = 0.5cm, shape = rectangle]
  
  \node[state]        (TOP)                                   {\parbox{4cm}  {\centering \small \<@MustCallUnknown> $= \top$}};
  \node[state]         (MCAB)      [below   = of TOP]        {\parbox{4cm}  {\centering \small \<@MustCall(\{"a", "b"\})>}};
  \node[state]         (MCA)    [below  = of MCAB, xshift=-2.2cm]         {\parbox{4cm} {\centering \small \<@MustCall(\{"a"\})>}};
  \node[state]         (MCB)    [below  = of MCAB, xshift=2.2cm]         {\parbox{4cm} {\centering \small \<@MustCall(\{"b"\})>}};
  \node[state]         (BOT)     [below      = of MCAB, yshift=-0.8cm]       {\parbox{4cm}  {\centering \small \<@MustCall(\{\})> $= \bot$}};

  \path

  (MCAB)        edge        node {} (TOP)
  
  (MCA)         edge        node {} (MCAB)
  (MCB)         edge        node {} (MCAB)
  
  (BOT)         edge        node {} (MCA)
  (BOT)         edge        node {} (MCB)

  ;

\end{tikzpicture}
%\precaptionspace
%\vspace{0.25cm}
\caption{Part of the type hierarchy for representing which methods must be
  called.
  If an expression's type has qualifier \<@Must\-Call(\{"a", "b"\})>, then
  the methods ``\<a>'' and ``\<b>'' might need to be called before the
  expression is deallocated.
  Arrows represent
  subtyping relationships.
  The diagram shows a part of the type hierarchy; the full hierarchy is a
  lattice of arbitrary size. 
}
\label{fig:must-call-hierarchy}
\end{figure}

This checker tracks the methods that an object should call before it
is deallocated. It produces a conservative overapproximation of the
set of methods that might actually need to be called on an object.

For example, an object whose type is \<java.io.OutputStream> might
have an obligation to call the \<close()> method, to release an
underlying file resource. Or, such an \<OutputStream> might not have
such an obligation, if the underlying resource is, for example, a byte
array. Either of these obligations can be represented by the static
type \<@MustCall(\{"close"\}) OutputStream>, which can be read as "an
OutputStream that might need to call close before it is deallocated".

The MustCall Checker supports two qualifiers: \<@MustCall> and
\<@MustCallUnknown>. The \<@MustCall> qualifier's arguments are the
methods that the annotated type should call. The type
\<@MustCall(\{"a"\}) Object obj> means that before \<obj> is
deallocated, \<obj.a()> should be called.

Part of the type hierarchy appears in \cref{fig:must-call-hierarchy}.
All types are subtypes of \<@MustCallUnknown>.
The subtyping relationship is for two \<@MustCall> types with sets
of required methods $A$ and $B$ is:
\trule{A \subseteq B}{\<@MustCall(A)> \sqsubseteq \<@MustCall(B)>}

The default type qualifier for raw base types is \<@MustCall(\{\})>.
By writing another type qualifier on the declaration of a class, a
user can specify a different default for raw types of that class.
Using this mechanism, the qualified type of the expression \<new
java.net.Socket()> is \<@MustCall({"close"}) Socket>.

\subsection{A type system for called methods}
\label{sec:called-methods}

Our Called Methods type system is an extension of a similar system
from prior work~\cite{kellogg20verifying}. We made some bugfixes and
small enchancements to that system; the primary difference in our
version is that methods are considered called even if they throw an exception
(in the prior work, a method was only considered ``called'' when it terminated
successfully). The remainder of this section is a brief explanation
of the Called Methods type system; see~\cite{kellogg20verifying} for
a more full description.

The Called Methods Checker is an \emph{accumulation analysis}: a special case
of typestate analysis~\cite{StromY86} in which:
\begin{enumerate}
\item the order in which operations are performed cannot affect what is
  subsequently permitted, and
\item executing more operations does not add restrictions.
\end{enumerate}
This special case of typestate analysis is useful because it does not
require a whole-program alias analysis for soundness, and instead
can be implemented as a standard pluggable type system.

The checker supports two qualifiers: \<@CalledMethods>
and \<@CalledMethodsBottom>. The type \<@CalledMethods(>$A$\<) Object>
represents an object on which the methods in the set $A$ have definitely
been called (other methods not in $A$ might also have been called).
\<@CalledMethodsBottom> is a subtype of every other type; the subtyping
rule between any two \<@CalledMethods> annotations with method sets $A$
and $B$ is:
\trule{B \subseteq A}{\<@CalledMethods(A)> \sqsubseteq \<@CalledMethods(B)>}
The resulting lattice has \<@CalledMethods(\{\})> as its top type.

Called Methods type are inferred within method bodies---when a method
\<foo()> is called on an object \<obj>, <obj>'s type is refined to a
set including \<"foo"> in the remainder of the method body (or until
it might be side-effected).

\subsection{A worklist algorithm for comparing the results of these two type systems}
\label{sec:must-call-invoked}

\todo{The section title needs work...}

\todo{This section is currently an outline.}

The Ownership Checker is a simple worklist algorithm that operates over the CFG.
It maintains a set of owning pointers to objects, using a set of simple
ownership rules:
\begin{itemize}
\item a newly-allocated object is owned
\item the value returned by any method called is owned
\end{itemize}

These rules guarantee that there is at least one owning pointer to each
object that might contain a resource. \Cref{sec:lightweight-ownership}
gives more details on how ownership is transferred.

When an owning pointer goes out of scope, the Ownership Checker
compares the types computed by the Must Call Checker and the Called
Methods Checker to determine if the requirements on the expression
going out of scope have been fulfilled using the following process,
supposing that some expression \<expr> is going out of scope at
program point $P$:
\begin{enumerate}
  \item The Ownership Checker requests a Must Call type from the Must
    Call Checker for \<expr> at $P$. Suppose this type is
    \<@MustCall(>$A$\<)> for some set of methods $A$ (if the type is
    \<@MustCallUnknown>, the Ownership Checker always issues an
    error).
  \item The Ownership Checker requests a Called Methods type from the
    Called Methods Checker for \<expr> at $P$. Suppose this type is
    \<@CalledMethods(>$B$\<)> for some set of methods $B$.
  \item The Ownership Checker compares the sets $A$ and $B$. If
    $A \supset B$, the Ownership Checker issues an error.
\end{enumerate}

The Ownership Checker also does a simple, intra-procedural must-alias
analysis, to avoid issuing duplicate errors for e.g. constructor
invocations that are assigned to local variables. At most one error
for each must-alias set is ever issued. 
