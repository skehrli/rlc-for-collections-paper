\section{Related Work}
\label{sec:relatedwork}

\subsection{Resource leak detection}
\label{sec:rw-resource-leaks}

\todo{Do a literature search on this topic, and add anything pertinent to
  the below.}

\todo{Re-organize and edit the below in a way that makes sense.}

Tracker~\cite{TorlakC10} uses a dataflow analysis over the control-flow
graph to reason about resource leaks, while avoiding the problem of
whole-program alias analysis using an access-path approach. Like our tool,
it scales to large Java programs; unlike \tool, however, it is unsound.

Grapple~\cite{zuo2019grapple} models alias and dataflow analyses as
dynamic transitive-closure computation of graph representations of
the target programs, enabling extraordinary precision with a whole-program
alias analysis. Compared to \tool, it is more precise but suffers
from unsoundness and an order of magnitude longer run times.

The Eclipse Compiler for Java includes a simple dataflow-based bug-finder
for resource leaks~\cite{ecj-resource-leak}. Their analysis uses a fixed
set of ownership heuristics and a fixed list of wrapper classes; unlike
\tool, it is also unsound. It is fast and \todo{possibly} precise.

Weimer and Necula~\cite{WeimerN04} proposed a language feature called
a compensation stack to avoid resource leak problems in Java.

\subsection{Ownership tracking}

\todo{Manu: Describe relationship to rust's borrow checker.}

\todo{Is there other relevant related work? Find out.}

\subsection{Typestate analysis}
\label{sec:rw-typestate}

\tool relies on an accumulation
analysis~\cite{kellogg20verifying,FahndrichLeino03} to avoid the need
for a whole-program alias analysis, while retaining soundness.  We
build on the core analysis described in~\cite{kellogg20verifying},
adding the concept of accumulation frames.

Accumulation analysis is a special-case of typestate analysis~\cite{StromY86}.
In general, a typestate analysis requires a whole-program alias analysis
for soundness, making them impractical for resource leak detection
in industrial-size Java programs. \todo{However, people have probably tried
  so go figure that out.}

An alternative to applying a typestate analysis to an existing program
is to rewrite the program in a typestate-oriented programming
language~\cite{AldrichSSS2009}. \todo{Aldrich's group probably has a
  paper about doing resource leak detection in Plaid, so find and cite
  that here.}
