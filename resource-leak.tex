%%
%% This is file `sample-sigplan.tex',
%% generated with the docstrip utility.
%%
%% The original source files were:
%%
%% samples.dtx  (with options: `sigplan')
%% 
%% IMPORTANT NOTICE:
%% 
%% For the copyright see the source file.
%% 
%% Any modified versions of this file must be renamed
%% with new filenames distinct from sample-sigplan.tex.
%% 
%% For distribution of the original source see the terms
%% for copying and modification in the file samples.dtx.
%% 
%% This generated file may be distributed as long as the
%% original source files, as listed above, are part of the
%% same distribution. (The sources need not necessarily be
%% in the same archive or directory.)
%%
%% The first command in your LaTeX source must be the \documentclass command.
\documentclass[sigconf,screen]{acmart}
\settopmatter{printccs=true, authorsperrow=4}

%%
%% \BibTeX command to typeset BibTeX logo in the docs
\AtBeginDocument{%
  \providecommand\BibTeX{{%
    \normalfont B\kern-0.5em{\scshape i\kern-0.25em b}\kern-0.8em\TeX}}}

%% Rights management information.  This information is sent to you
%% when you complete the rights form.  These commands have SAMPLE
%% values in them; it is your responsibility as an author to replace
%% the commands and values with those provided to you when you
%% complete the rights form.
% \setcopyright{acmcopyright}
% \copyrightyear{2018}
% \acmYear{2018}
% \acmDOI{10.1145/1122445.1122456}

%% These commands are for a PROCEEDINGS abstract or paper.
% \acmConference[Woodstock '18]{Woodstock '18: ACM Symposium on Neural
%   Gaze Detection}{June 03--05, 2018}{Woodstock, NY}
% \acmBooktitle{Woodstock '18: ACM Symposium on Neural Gaze Detection,
%   June 03--05, 2018, Woodstock, NY}
% \acmPrice{15.00}
% \acmISBN{978-1-4503-9999-9/18/06}


%%
%% Submission ID.
%% Use this when submitting an article to a sponsored event. You'll
%% receive a unique submission ID from the organizers
%% of the event, and this ID should be used as the parameter to this command.
%%\acmSubmissionID{123-A56-BU3}

%%
%% The majority of ACM publications use numbered citations and
%% references.  The command \citestyle{authoryear} switches to the
%% "author year" style.
%%
%% If you are preparing content for an event
%% sponsored by ACM SIGGRAPH, you must use the "author year" style of
%% citations and references.
%% Uncommenting
%% the next command will enable that style.
%%\citestyle{acmauthoryear}


% tables
\usepackage[utf8]{inputenc} 
\usepackage[T1]{fontenc}
\usepackage{microtype}
\usepackage{tabularx}
\usepackage{multirow}
\usepackage{booktabs}
\usepackage{algorithm}
\usepackage[noend]{algpseudocode}

% xspace command
\usepackage{xspace}

% From https://tex.stackexchange.com/questions/177025/
\makeatletter
\newcounter{algorithmicH}% New algorithmic-like hyperref counter
\let\oldalgorithmic\algorithmic
\renewcommand{\algorithmic}{%
  \stepcounter{algorithmicH}% Step counter
  \oldalgorithmic}% Do what was always done with algorithmic environment
\renewcommand{\theHALG@line}{ALG@line.\thealgorithmicH.\arabic{ALG@line}}
\makeatother

% lstlisting command
\usepackage{listings}
\usepackage[scaled]{beramono}
\newcommand*\LSTfont{\Small\fontencoding{T1}\ttfamily\SetTracking{encoding=*}{-60}\lsstyle}
\lstset{language=Java,
  frame=none,
  aboveskip=1.5pt,
  belowskip=0pt,
  showstringspaces=false,
  columns=flexible,
  basicstyle=\LSTfont,
  numbers=none,
  numberstyle=\tiny\color{black},
  keywordstyle=\color{black},
  commentstyle=\color{black},
  stringstyle=\color{black},
  breaklines=true,
  breakatwhitespace=true,
  tabsize=3,
  %emph={@NonNegative,@Positive,@GTENegativeOne,@LTLengthOf,@LTEqLengthOf,@IndexFor,@IndexOrHigh,@IndexOrLow,@HasSubsequence,@LessThan,@SameLen,@SearchIndexFor,@MinLen,@ArrayLen,@IntVal,@IntRange,@LengthOf,@UpperBoundUnknown,@LowerBoundUnknown,int,double,List,Map,Object,SerialDate,Long,Integer,DefaultPolarItemRenderer,LegendItem,PolarPlot,XYDataset,long,T,String,string,byte,InputStream,CategoryDataset,DatasetRenderingOrder,ArrayList,Entry,Values,Number,ValuesContract,ImmutableIntArray,Dataset,XYZDataset}, emphstyle=\color{blue}
}

% Graphics
\usepackage{tikz}
\usetikzlibrary{arrows,automata,positioning}

% Change font and line spacing for figure captions
% \usepackage{setspace,caption}
% \captionsetup{labelfont={small,bf}, textfont={small,bf,stretch=0.8}, labelsep=colon, margin=0pt}

\usepackage{balance} % balanced columns on last page

% cref command; best to load last
\usepackage{cleveref}
\newcommand{\crefrangeconjunction}{--}

%%% The following is specific to ESEC/FSE '21 and the paper
%%% 'Lightweight and Modular Resource Leak Verification'
%%% by Martin Kellogg, Narges Shadab, Manu Sridharan, and Michael D. Ernst.
%%%

\setcopyright{rightsretained}
\acmPrice{}
\acmDOI{10.1145/3468264.3468576}
\acmYear{2021}
\copyrightyear{2021}
\acmSubmissionID{fse21main-p307-p}
\acmISBN{978-1-4503-8562-6/21/08}
\acmConference[ESEC/FSE '21]{Proceedings of the 29th ACM Joint European Software Engineering Conference and Symposium on the Foundations of Software Engineering}{August 23--28, 2021}{Athens, Greece}
\acmBooktitle{Proceedings of the 29th ACM Joint European Software Engineering Conference and Symposium on the Foundations of Software Engineering (ESEC/FSE '21), August 23--28, 2021, Athens, Greece}

%%
%% end of the preamble, start of the body of the document source.
\begin{document}

%%% Todo comments

%% Comment out one of these two definitions.
%\newcommand{\todo}[1]{\relax}
\newcommand{\todo}[1]{{\color{red}\bfseries [[#1]]}}


%%
%% The "title" command has an optional parameter,
%% allowing the author to define a "short title" to be used in page headers.
\title{Lightweight and Modular Resource Leak Verification}

%%
%% The "author" command and its associated commands are used to define
%% the authors and their affiliations.
%% Of note is the shared affiliation of the first two authors, and the
%% "authornote" and "authornotemark" commands
%% used to denote shared contribution to the research.

\author{Martin Kellogg}
\affiliation{University of Washington, Seattle
 \country{USA}}
\email{kelloggm@cs.washington.edu}
\authornote{Both authors contributed equally to this research.}

\author{Narges Shadab}
\affiliation{University of California, Riverside
 \country{USA}}
\email{nshad001@ucr.edu}
\authornotemark[1]

\author{Manu Sridharan}
\affiliation{University of California, Riverside
 \country{USA}}
\email{manu@cs.ucr.edu}

\author{Michael D. Ernst}
\affiliation{University of Washington, Seattle
  \country{USA}}
\email{mernst@cs.washington.edu}

%%
%% By default, the full list of authors will be used in the page
%% headers. Often, this list is too long, and will overlap
%% other information printed in the page headers. This command allows
%% the author to define a more concise list
%% of authors' names for this purpose.
%\renewcommand{\shortauthors}{Trovato and Tobin, et al.}

%%
%% The abstract is a short summary of the work to be presented in the
%% article.
\begin{abstract}
  A resource leak occurs when a program allocates a resource, such as a
  socket or file handle, but fails to deallocate
  it.  Resource leaks
  % are an important, common problem that
  cause resource
  starvation, slowdowns, and crashes.  Previous techniques
  to prevent resource leaks are either unsound,
  imprecise, inapplicable to existing code, slow, or a combination
  of these.
  
  Static detection of resource leaks requires checking that de-allocation methods
  are always invoked on relevant objects before they become unreachable.
  % We observe that detecting a resource leak for an object involves three
  % parts: 1) tracking its \emph{must-call obligations},
  % 2) tracking which methods have been called on it, and 3) comparing
  % these to check if the obligations have been fulfilled.
  Our key insight is that leak detection can be reduced to an
  accumulation problem, a class of typestate problems amenable
  to sound and modular checking without the need for a heavyweight,
  whole-program alias analysis.
  % We developed a
  % baseline leak checker via this approach.
  % Rashmi said the following was too negative, and I agree:
  % , but found
  % that it suffered from an excess of false positives.
  The precision of an accumulation analysis can be improved by computing
  targeted aliasing information, and we augmented our baseline checker with
  three such novel techniques:
  a lightweight ownership transfer system;
  a specialized resource alias analysis;
  and a system
  to create a fresh obligation when a non-final resource field
  is updated.
  % check that re-assignments to non-final fields
  % that hold a resource are safe.
  %% MK: the below is too wordy - less jargon in the abstract, just the summary
  %%
  %% Lightweight ownership tracking is a
  %% technique to help determine which alias is responsible for releasing
  %% a resource.  Since our goal is only to prevent resource leaks, our
  %% ownership tracking avoids many of the heavyweight aliasing
  %% restrictions associated with other ownership type systems.
  %% Resouce alias analysis is a specialized, lightweight must-alias analysis
  %% that identifies program elements that definitely control the same resource.
  %% Accumulation frames extend the theory of accumulation analysis to permit
  %% certain kinds of cycles in the typestate automaton.

  Our approach occupies a unique slice of the design space: it is sound
  and runs relatively quickly
  %\todo{mention ``unsound''?}
  (taking minutes on programs that a
  state-of-the-art approach
  took hours to analyze).
  We implemented our techniques for Java in an open-source tool
  called \tool.
  \Tool revealed 49 real resource leaks in widely-deployed software.
  It scales well, has a manageable false positive rate
  (comparable to the high-confidence
  resource leak analysis built into the Eclipse IDE), and
  imposes only a small annotation burden (1/1500 LoC) for developers.
\end{abstract}

%%
%% The code below is generated by the tool at http://dl.acm.org/ccs.cfm.
%% Please copy and paste the code instead of the example below.
%%
\begin{CCSXML}
<ccs2012>
<concept>
<concept_id>10011007.10011074.10011099</concept_id>
<concept_desc>Software and its engineering~Software verification and validation</concept_desc>
<concept_significance>500</concept_significance>
</concept>
</ccs2012>
\end{CCSXML}

%\ccsdesc[500]{Software and its engineering~Software verification and validation}

\ccsdesc[500]{Software and its engineering~Software verification}

%%
%% Keywords. The author(s) should pick words that accurately describe
%% the work being presented. Separate the keywords with commas.
\keywords{Pluggable type systems, accumulation analysis, static analysis,
  typestate analysis, resource leaks}

%% A "teaser" image appears between the author and affiliation
%% information and the body of the document, and typically spans the
%% page.
%% \begin{teaserfigure}
%%   \includegraphics[width=\textwidth]{sampleteaser}
%%   \caption{Seattle Mariners at Spring Training, 2010.}
%%   \Description{Enjoying the baseball game from the third-base
%%   seats. Ichiro Suzuki preparing to bat.}
%%   \label{fig:teaser}
%% \end{teaserfigure}

%%
%% This command processes the author and affiliation and title
%% information and builds the first part of the formatted document.

\maketitle

\section{Introduction}
\label{sec:intro}

Resource leaks are a classic problem, and extant approaches to preventing
them are either too expensive or unsound.

An ideal tool for preventing resource leaks would be fast, sound, and precise.

Extant approaches fail at least one of these: bug finders like the analysis
in ecj are unsound; complex analyses based on typestate are either too slow
(cite the Eurosys 2019 paper on Grapple, with numbers on run time), unsound, or both.

Our key insight is that the resource leak problem is an accumulation
problem~\cite{kellogg20verifying}.

This insight allows us to build a novel static analysis, based on a standard
type system without whole-program alias analysis, that is sound while retaining
acceptable speed and precision.

In particular, it beats bug finders in soundness, while still being reasonably
fast and precise. It beats typestate analysis on speed and soundness, while losing on precision. \todo{Maybe also want to compare against more "heavyweight" systems like Rust ownership or fancier type systems.  We are more lightweight while still sound for detecting leaks.}

To achieve the necessary precision in practice, we needed two innovations
beyond our key insight:
\begin{itemize}
\item a lightweight, heuristic ownership transfer system.
\item the concept of \emph{accumulation frames}, which allow us to
  soundly reset the accumulation status of a single program element
  after showing that its requirements have been fulfilled. \todo{This
    description sucks!}
\end{itemize}

Our contributions are:
\begin{itemize}
\item the insight that the resource leak problem is an accumulation problem,
  and a type system designed to take advantage of this fact (\cref{sec:base-type-systems}).
\item a novel lightweight ownership tracking system (\cref{sec:lightweight-ownership}).
\item a novel extension to accumulation analysis (\cref{sec:reset-must-call}).
\item an implementation of our type system as an extension to Java,
  augmented with these two innovations (\cref{sec:implementation}).
\item case studies that demonstrate that our implementation finds real
  bugs and has few false positives (\cref{sec:case-studies}).
\item an ablation study that demonstrates the contributions of
  lightweight ownership and accumulation frames to the low false
  positive rate of our approach (\cref{sec:ablation}). \todo{this sounds great!  we need to be sure we can implement these experiments.}
\item a comparison to alternative approaches to resource leak
  checking, that demonstrates that our approach occupies a unique
  point in the design space: faster and more usable than heavy-weight
  typestate systems, but sound and able to find even subtle bugs,
  unlike heuristic bug-finders (\cref{sec:compare}).
\end{itemize}
  


\section{Background on type qualifiers}
\label{sec:background}

\Cref{sec:must-call,sec:called-methods} describe
\emph{pluggable type systems}~\cite{FosterFFA99}
that are layered on top of the type system of the host
language.  Types in a pluggable type system are composed of two parts:
a \emph{type qualifier} and a base type. The type qualifier is the
part of the type that is unique to the pluggable type system; the base
type is a type from the host language. Our implementation is for Java
(see \cref{sec:implementation}), so we use the Java syntax for type
qualifiers: ``\<@>'' before a type indicates that it is a type
qualifier, and a type without ``\<@>'' is a base type.
This paper sometimes omits the basetype when it is obvious from context.

A type system checks programmer-written types.  Our system requires the
programmer to write types on method signatures, but within method bodies it
uses flow-sensitive type refinement, a dataflow analysis that performs type
inference.  This permits an expression to have different types on different
lines of the program.


% the technical sections about the core accumulation systems: called methods,
% must-call, and the must call invoked worklist algorithm.

\section{Cooperating analyses for leak detection}
\label{sec:base-type-systems}

This section presents a sound, modular, accumulation-based
resource leak checker.
\Cref{sec:lightweight-ownership,sec:reset-must-call,sec:must-call-choice}
enhance its precision.

\Tool is composed of three cooperating analyses:
\begin{enumerate}
\item a taint tracking type system (\cref{sec:must-call}) computes a conservative
  \emph{overapproximation} of the set of methods that might need to be called
  on each expression in the program.
\item an accumulation type system (\cref{sec:called-methods}) computes
  a conservative \emph{underapproximation} of the set of methods that are
  actually called on each expression in the program.
\item a dataflow analysis (\cref{sec:must-call-invoked}) computes a set
  of owning pointers that contains at least one pointer for every resource.
  For each owning pointer, it checks the consistency of the results
  of the two type systems at each program
  point at which the pointer might go out of scope:
  if any method that might need to be called is not in the set of methods
  that definitely were called, it issues an error.
\end{enumerate}

\begin{figure}
  \lstinputlisting{simplesocket.txt}
  \caption{A safe use of a \<Socket> resource.}
  \label{fig:example}
\end{figure}

\noindent
This section uses \cref{fig:example} as a motivating example.
It shows
a safe use of a \<Socket>---a resource that must be closed before
it is deallocated.

\subsection{A type system for must-call obligations}
\label{sec:must-call}

\begin{figure}

\begin{tikzpicture}[->, shorten >= 1pt, auto, node distance=0.3cm]
  \tikzstyle{every state}=[fill=none,draw=none,text=black, minimum size = 0.5cm, shape = rectangle]
  
  \node[state]        (TOP)                                   {\parbox{4cm}  {\centering \small \<@MustCallUnknown> $= \top$}};
  \node[state]         (MCAB)      [below   = of TOP]        {\parbox{4cm}  {\centering \small \<@MustCall(\{"a", "b"\})>}};
  \node[state]         (MCA)    [below  = of MCAB, xshift=-2.2cm]         {\parbox{4cm} {\centering \small \<@MustCall(\{"a"\})>}};
  \node[state]         (MCB)    [below  = of MCAB, xshift=2.2cm]         {\parbox{4cm} {\centering \small \<@MustCall(\{"b"\})>}};
  \node[state]         (BOT)     [below      = of MCAB, yshift=-0.8cm]       {\parbox{4cm}  {\centering \small \<@MustCall(\{\})> $= \bot$}};

  \path

  (MCAB)        edge        node {} (TOP)
  
  (MCA)         edge        node {} (MCAB)
  (MCB)         edge        node {} (MCAB)
  
  (BOT)         edge        node {} (MCA)
  (BOT)         edge        node {} (MCB)

  ;

\end{tikzpicture}
%\precaptionspace
%\vspace{0.25cm}
\caption{Part of the \<MustCall> type hierarchy for representing which methods must be
  called; the full hierarchy is a
  lattice of arbitrary size.
  If an expression's type has qualifier \<@Must\-Call(\{"a", "b"\})>, then
  the methods ``\<a>'' and ``\<b>'' might need to be called before the
  expression is deallocated.
  Arrows represent
  subtyping relationships.
}
\label{fig:must-call-hierarchy}
\end{figure}

The Must Call type system tracks
which methods might need to be called on
a given object before the object is deallocated.
This type system is general---it is not specific
to resource leaks.

\todo{It would be helpful to explicitly state whether this analysis is an
  accumulation analysis.}

The Must Call type system supports two qualifiers: \<@MustCall> and
\<@MustCallUnknown>. The \<@MustCall> qualifier's arguments are the
methods that the annotated type must call. The type
\<@MustCall(\{"a"\}) Object obj> means that before \<obj> is
deallocated, \<obj.a()> might need to be called.
\Tool conservatively requires all these methods to be called,
and it issues a warning if they are not.

For example, consider \cref{fig:example}. The expression \<null> has type
\<@MustCall(\{\})>---it has no obligations
to call particular methods---and \<s> has that type after its initialization.
The \<new> expression has type \<@MustCall("close")>, and therefore
\<s> has that type after the assignment.
At the start of the \<finally> block, where both values for \<s> flow,
the type of \<s> is their least upper bound, which is \<@MustCall("close")>.

% Note that the type \<@MustCall("close")>
% can represent anything that \emph{might} need to
% call \<close()>: for example, at the entrance to
% the \<finally> block in \cref{fig:example}, \<s>'s
% actual value might either be \<null>, which does not
% need to call any methods, or an open \<Socket>, which does.
% Thus, either the obligation to close or no obligation at all
% can be represented by the static
% type \<@MustCall(\{"close"\}) Socket>, which can be read as ``a
% Socket that might need to call close before it is deallocated''.

Part of the type hierarchy appears in \cref{fig:must-call-hierarchy}.
All types are subtypes of \<@MustCallUnknown>.
The subtyping relationship is for two \<@MustCall> types with sets
of required methods $A$ and $B$ is:
\trule{A \subseteq B}{\<@MustCall(A)> \sqsubseteq \<@MustCall(B)>}

The default type qualifier is \<@MustCall(\{\})> for base types without a
programmer-written type qualifier.\footnote{For unannotated local variables,
  flow-sensitive type refinement infers a type.}
% By writing another type qualifier on the declaration of a class, a
% user can specify a different default for raw types of that class.
Our implementation
provides JDK annotations to require that every 
\<Closeable> object must have the \<close()> method called before
it is deallocated.

% also provides an ``inheritable'' version of the \<@MustCall> annotation,
% which allows us to annotate a class (or interface) and all of its subtypes.
% For example, we use such an annotation on \<java.io.Closeable> to indicate
% that all \<Closeable> objects must have the \<close()> method called before
% they are deallocated.

%% This is redundant, I think.
% A benefit of using a type system for tracking must-call obligations is
% that \Tool can use local type inference to refine the obligations of
% a particular variable. For example, if the only assignment to \<s>
% in \cref{fig:example} were the initial assignment to \<null>, then
% \Tool would be able to determine that \<s> does not have a must-call
% obligation.


\subsection{A type system for called methods}
\label{sec:called-methods}

\todo{Mike changed ``our types system for inferring'' to ``our type system
  tracks''.  Let's lay off use of ``infer'' because it implies we have
  written a type inference tool, when what we have really written is a
  type-checking tool (with local type inference).}

The Checked Methods type system tracks a conservative underapproximation of which methods have been called on an object.
It is an extension of a similar system
from prior work~\cite{KelloggRSSE2020}.  The primary difference in our
version is that a method is considered called even if it throws an
exception---a necessity in Java because the \<close()> method
in \<java.io.Closeable> is specified to possibly throw an \<IOException>.
In the prior work, a method was only considered ``called'' when it terminated
successfully.
The remainder of this section is a brief summary
of the prior work~\cite{KelloggRSSE2020}.

% This checker is an \emph{accumulation analysis}: a special case
% of typestate analysis~\cite{StromY86} in which:
% \begin{enumerate}
% \item the order in which operations are performed cannot affect what is
%   subsequently permitted, and
% \item executing more operations does not add restrictions---that is,
%   the set of permitted operations after executing some operation is always
%   a superset of the set of permitted operations before executing that operation.
% \end{enumerate}
% This special case of typestate analysis is useful because it does not
% require a whole-program alias analysis for soundness, and instead
% can be implemented as a standard pluggable type system.

The checker is an accumulation analysis whose accumulation qualifier is \<@CalledMethods>.
The type \<@CalledMethods(>$A$\<) Object>
represents an object on which the methods in the set $A$ have definitely
been called; other methods not in $A$ might also have been called.
The subtyping
rule is:
\trule{B \subseteq A}{\<@CalledMethods(A)> \sqsubseteq \<@CalledMethods(B)>}
The top type is \<@CalledMethods(\{\})>.
The qualifier \<@CalledMethodsBottom> is a subtype of every \<@CalledMethods> qualifier.

Thanks to flow-sensitive type refinement,
Called Methods types are inferred within method bodies.
In \cref{fig:example} the type of \<s> is initially \<@CalledMethods({})>,
but it transitions to \<@CalledMethods("close")> after the call to \<close>.


\subsection{Consistency checking}
\label{sec:must-call-invoked}

\begin{algorithm}[t]
  \caption{Finding unfulfilled \<@MustCall> obligations.
    \todo{Please write a comment about what $D$ is.  The pseudocode is not
      readable without that, and we should not force readers to hunt in the
      text for that explanation.}
    \todo{D contains statements and formal parameters as the second element
      of its paris.  Why doesn't it just store (declared) types as the
      second element of the pair?}
    \todo{D filters out empty MustCall obligations.  I think the algorithm
      would be easier to read if it did not do so.  Then D could contain
      empty MustCall obligations, and there would be a test of MCSatisfied
      febore reporting a must-call violation.  That puts the testing at a
      more logical place, I think.}
    \todo{I think it would be helpful to note that \|CFG| is for a method.}
    \todo{I think it would be better to write \|CFG.entry| and
      \|CFG.Statement| rather than just \|entry| and \|Statement| (etc.),
      because it is more explicit and clear.}
    \todo{It's odd that \|Statement| is singular but \|Formals| is plural.
      I would make both plural.}
    \todo{I think lines~\ref{li:start-init}--\ref{li:end-init} would be
      shorter and clearer as: ``for every assignment, add $\langle \|LHS|,
      \mbox{declared type of \|RHS|}\rangle$''.}

    % Structured like pseudocode in~\cite{TorlakC10}.
    % \todo{Are elements of $D$ must-call obligations?  If so, say that.  If
    %   not, say what they are.}
    % \todo{Elements of D ``are of the form $\langle P, s' \rangle$'', but
    %   line 5 uses $\mapsto$ instead of $\langle \rangle$.}
    % \todo{The ``not at a fixed point'' test is \emph{not} the same as ``the
    %   value of D is the same as on the previous iteration'', which makes
    %   the algorithm a bit harder to read.}
  }
  \label{alg:consistency-checker}
  \begin{algorithmic}[1]
  \Procedure{FindMissedCalls}{$CFG$}
  \State $D \gets \{ s \mapsto \emptyset\ |\ s \in \mathit{Statement} \}$\label{li:start-init}
  \For{each $p \in \mathit{Formals}$ and $t \in \mathit{succ}(\mathit{entry})$}\label{li:init-formals}
    \If{$\textsc{AddsObligation}(p)$}
      \State $D(t) \gets D(t) \cup \langle \{p\}, p \rangle$
    \EndIf
  \EndFor \label{li:end-init-formals}
  \For{$s \in \mathit{Statement}$ of the form \lstinline{p = m(p1, p2, ...)}} \label{li:init-calls}
    \State $\forall t \in \mathit{succ}(s)\ .\ D(t) \leftarrow D(t) \cup \textsc{FactsFromCall}(s)$
  \EndFor  \label{li:end-init}
  \While{$D$ has not reached fixed point}
    \For{each $s \in \mathit{Statement}$ and $\langle P, e \rangle \in D(s)$}
    \If{$s$ is $\mathit{exit}$} \label{li:end-scope}
      \State report a must-call violation for $e$
    \ElsIf{$\neg \textsc{MCSatisfied}(P,s)$} \label{li:check-satisfied}
%    \State // \textit{propagate to successors}
    \State $kill \gets $ $s$ assigns a variable ? $\{\textsc{LHS}(s)\}$ : $\emptyset$ \label{li:compute-kill}
    \State $gen \gets \textsc{CreatesAlias}(P,s)$ ? $\{\textsc{LHS}(s)\}$ : $\emptyset$ \label{li:compute-gen} 
    % \If{$s$ is \lstinline{p = q} and $q \in P$}
    % \State $gen \gets \{p\}$
    % \EndIf
    \State $N \gets (P - kill) \cup gen$ \label{li:compute-new-mc-aliases}
 %   \State // \textit{we even propagate }$\emptyset$\textit{; error reported at exit}
    \State $\forall t \in \mathit{succ}(s)\ .\ D(t) \leftarrow D(t) \cup \langle
    N, e \rangle$ \label{li:prop-to-succs}
    \EndIf
    \EndFor
  \EndWhile \label{li:alg-loop-end}
  \EndProcedure
  \end{algorithmic}
\end{algorithm}

\begin{algorithm}[h]
  \caption{Helper functions for \Cref{alg:consistency-checker}. \todo{Just
  merge with other algorithm?}
    \todo{For consistency and clarity, rename MCSatisfied to MCSatisfiedAfter.}
    \todo{Please define MCAfter and CMAfter.  The algorithm is not complete
      without those definitions.}
    \todo{In AddsObligation, clarify where the type of $e$ is being
      interpreted.  I think it's the declared type of $e$ rather than its
      flow-sensitively-refined type, but this is not clear.}
}
  \label{alg:helpers}
  \begin{algorithmic}[1]
  \State // \textit{Does e introduce a must-call obligation to check?}
  \Procedure{AddsObligation}{$e$}
  \State \Return $e$ has non-empty \<@MustCall> type
  \EndProcedure
  \State // \textit{s must be a call statement p = m(p1, p2, ...)}
  \Procedure{FactsFromCall}{$s$}
  \State $p \gets \textsc{LHS}(s), c \gets \textsc{RHS}(s)$
  \State \Return $\textsc{AddsObligation}(c)$ ? $\{ \langle \{ p \}, c
  \rangle \}$ : $\emptyset$
  \EndProcedure
  \State // \textit{Is the must-call obligation for P satisfied by
    s?}\todo{I think that should be ``satisfied \emph{after} $s$'', not
    ``satisfied by $s$'', but in any event please clarify.}
  \Procedure{MCSatisfied}{$P,s$}
  \State \Return $\exists p \in P .\ \textsc{MCAfter}(p,s) \subseteq \textsc{CMAfter}(p,s)$
  \EndProcedure
  \State // \textit{Does s introduce a must alias for a var in P?}
  \Procedure{CreatesAlias}{$P,s$}
    \State \Return $s$ is of the form \lstinline{p = q} $\wedge\ \lstinline{q} \in P$
  \EndProcedure
  \end{algorithmic}

\end{algorithm}

Given \<@MustCall> and \<@CalledMethods> types, the Must
Call Consistency Checker ensures that \<@MustCall> methods for each object
are always invoked before the object becomes unreachable.  This property is
checked via an intra-procedural dataflow analysis.  Here, we present
a simple, sound version of the analysis, with limited reasoning about aliasing.
\Cref{sec:lightweight-ownership,sec:must-call-choice,sec:reset-must-call}
describe enhancements to this basic approach.

\paragraph{Language} For simplicity, we present the analysis over a simple
assignment language in three-address form.  An expression $e$ in the language is
a variable \<p>, a field read \<p.f>, a method call \<m(p1,p2,\ldots)> (constructor
calls are treated as method calls), or \<null>.  A statement $s$ takes one of three forms:
\<p = e>, where $e$ is an expression; \<p.f = p'>, for a field write; or
\<return p>.  Methods are represented by a control-flow graph (CFG) where nodes
are statements and edges indicate possible control flow.  We elide control-flow
predicates as the consistency checker is path-insensitive\todo{I think this
  ``path-insensitive'' should be ``flow-insensitive''.}. \manu{Martin, is the
must-call checker path-sensitive?  As in, does it interpret null checks?}

For a method CFG, $\mathit{Statement}$ is the statements,
$\mathit{Formals}$ is the formal parameters,
$\mathit{entry}$ is its entry node,
$\mathit{exit}$ is its exit node, and
$\mathit{succ}$ is its successor relation.
For a statement
$s$ of the form \<p = e>, $\textsc{LHS}(s) = p$ and $\textsc{RHS}(s) = e$.   

\paragraph{Pseudocode} \Cref{alg:consistency-checker} gives pseudocode for the
basic version of our checker, with helper functions in \Cref{alg:helpers}.  At a
high level, the dataflow analysis computes a map $D$ from each statement $s$ in
a CFG to a \emph{set} of facts of the form $\langle P, e \rangle$, where $P$ is
a set of variables and $e$ is either a method call expression or a formal
parameter variable\todo{Why not just store a MustCall type?}.  The meaning of $D$ is as follows: if $\langle P, e \rangle
\in D(s)$, then $e$ has a non-empty\todo{Is the fact that it is non-empty
  important?  Storing only those feels more like an implementation optimization.} \<@MustCall> type, and all variables in $P$
are \emph{must aliases} for the value of $e$ at the program point before $s$.
Computing a set of must aliases is useful since any must alias may be used to
satisfy the must-call obligation of $e$.  Using $D$, the analysis finds any $e$
that does not have its \<@MustCall> obligation fulfilled, and reports an error.

\Cref{alg:consistency-checker} proceeds as follows.
Lines~\ref{li:start-init}--\ref{li:end-init} initialize $D$, by scanning for
expressions that add must-call obligations to be checked.  
In the basic checker, such expressions include any formal parameter or method
call with a non-empty \<@MustCall> type.  The fixed-point loop then proceeds by
iterating over all facts $\langle P, e \rangle$ present in any $D(s)$ (our
implementation uses a worklist for efficiency).  If $s$ is the exit node
(\cref{li:end-scope}), the obligation for $e$ has not been satisfied, and an
error is reported.  Otherwise, the algorithm checks if the obligation for $e$ is
satisfied by $s$ (\cref{li:check-satisfied}).  For the basic checker, \textsc{MCSatisfied} in
\Cref{alg:helpers} checks whether there is some $p \in P$ such that
the \<@MustCall> type for $p$ after $s$, $\textsc{MCAfter}(p,s)$, is
contained in its \<@CalledMethods> type at that point, $\textsc{CMAfter}(p,s)$;
if true, all \<@MustCall> methods have already been invoked.  \textsc{MCAfter}
and \textsc{CMAfter} leverage the flow-sensitive inference for \<@MustCall> and
\<@CalledMethods> qualifiers described in \Cref{sec:must-call,sec:called-methods}.
\todo{The names CMAfter and
MCAfter are confusing - they're nearly identical. Can we think of better ones
that are equally useful?} \manu{I don't have an idea that is also terse, but I
am open to one}

If the obligation for $e$ is not yet satisfied, the algorithm propagates the
fact to successors with an updated set $N$ of must aliases.  $N$ is computed in
a standard gen-kill style on
lines~\ref{li:compute-kill}--\ref{li:compute-new-mc-aliases}.  The kill set
simply consists of whatever variable (if any) appears on the left-hand side of
$s$.  The gen set is computed by checking if $s$ creates a new must alias for
some variable in $P$, using the \textsc{CreatesAlias} routine.  Since our
analysis is accumulation, \textsc{CreatesAlias} could simply return
$\emptyset$\todo{The return type is boolean; I think you mean ``true'' here.}
without impacting soundness.  In \Cref{alg:helpers}, \textsc{CreatesAlias}
handles the case of a variable copy where the right-hand side is in $P$;
\Cref{sec:must-call-choice} presents more sophisticated handling. Finally,
\Cref{li:prop-to-succs} propagates the new fact to successors.  The process
continues until $D$ reaches a fixed point.

\begin{figure}
  \includegraphics[width=0.65\columnwidth,keepaspectratio]{cfg-example.pdf}
  \caption{Example CFG for illustrating \Cref{alg:consistency-checker}.}
  \todo{Add label on arrows where it is missing.}
  \todo{Note where an error is reported.}
  \label{fig:cfg-example}
\end{figure}

\paragraph{Example} \Cref{fig:cfg-example} shows an example CFG for a program
similar to\todo{This is confusing.  Make it the same, or show the source
  code for that CFG.} that of \Cref{fig:example}, to illustrate our analysis.  For
initialization of $D$, statement 1 introduces the fact $\langle \{ s \}, e
\rangle$ (where $e$ is the \<new Socket(...)> call) to $D(2)$ and $D(3)$.  At
statement 2, $s$ is killed, causing $\langle \emptyset , e \rangle$ to be added
to $D(\mathit{exit})$; this leads to an error being reported, as the socket is
not closed on this path.  Statement 3 creates a must alias $t$ for $s$, causing
$\langle \{ s, t \}, e \rangle$ to be added to $D(4)$.  For statement 4,
$\textsc{MCSatisfied}(\{s,t\},\mathtt{close(t)})$ holds, so no facts are
propagated from 4 to $\mathit{exit}$.

At this point, we have described a sound checker for \<@MustCall> obligations.
But for real code, this checker emits too many false positives.  Subsequent
sections make the checker more precise.

% The Ownership Checker is a simple worklist algorithm that operates over the CFG.
% It maintains a set of owning pointers to objects, using a set of simple
% ownership rules:
% \begin{itemize}
% \item a newly-allocated object is owned
% \item the value returned by any method called is owned
% \end{itemize}

% These rules guarantee that there is at least one owning pointer to each
% object that might contain a resource. \Cref{sec:lightweight-ownership}
% gives more details on how ownership is transferred.

% When an owning pointer goes out of scope, the Ownership Checker
% compares the types computed by the Must Call Checker and the Called
% Methods Checker to determine if the requirements on the expression
% going out of scope have been fulfilled using the following process,
% supposing that some expression \<expr> is going out of scope at
% program point $P$:
% \begin{enumerate}
%   \item The Ownership Checker requests a Must Call type from the Must
%     Call Checker for \<expr> at $P$. Suppose this type is
%     \<@MustCall(>$A$\<)> for some set of methods $A$ (if the type is
%     \<@MustCallUnknown>, the Ownership Checker always issues an
%     error).
%   \item The Ownership Checker requests a Called Methods type from the
%     Called Methods Checker for \<expr> at $P$. Suppose this type is
%     \<@CalledMethods(>$B$\<)> for some set of methods $B$.
%   \item The Ownership Checker compares the sets $A$ and $B$. If
%     $A \supset B$, the Ownership Checker issues an error.
% \end{enumerate}

% The Ownership Checker also does a simple, intra-procedural must-alias
% analysis, to avoid issuing duplicate errors for e.g. constructor
% invocations that are assigned to local variables. At most one error
% for each must-alias set is ever issued. 

% LocalWords:  simplesocket MustCallUnknown MCAB MustCall MCA xshift MCB
% LocalWords:  yshift Closeable CalledMethods CalledMethodsBottom IsExit
% LocalWords:  FindMissedCalls MCSatisfied CreatesAlias MCObligations
% LocalWords:  HasMCReturn EndOfScope CMBefore MCBefore


\section{Lightweight ownership tracking}
\label{sec:lightweight-ownership}

We developed two novel approaches for tracking whether a pointer is
owning: a lightweight ownership transfer approach using annotations
that does not depend on the annotations' correctness---that is, adding
ownership annotations can only reduce false positives, not false
negatives (\cref{sec:ownership-transfer})---and a lightweight, local
alias analysis for dealing with wrapper objects
(\cref{sec:must-call-choice}).

\subsection{Ownership transfer}
\label{sec:ownership-transfer}

\todo{Write about @Owning and @NotOwning, and show how they can only
  reduce false positives.}

\subsection{Lightweight alias analysis for wrapper types}
\label{sec:must-call-choice}

The Java standard library contains a number of \emph{wrapper types}:
classes that wrap a resource variable that might contain a resource.
For example, consider the class \<java.io.BufferedOutputStream>.  The
must-call obligation for an arbitrary \<BufferedOutputStream> is
\<@MustCall(\{``close''\})>---the stream might be writing to a file,
for example.  \<BufferedOutputStream>'s constructors all have a
parameter which is another \<OutputStream>. When \<close()> is called
on the resulting \<BufferedOutputStream>, the implementation calls
\<close()> on the wrapped \<OutputStream>. Since there are no
additional resources allocated by the \<BufferedOutputStream> that
were not part of the \<OutputStream> passed to the constructor,
calling \<close()> on either stream is equally correct (though most
Java style guides suggest calling \<close()> on the outermost
wrapper \todo{cite that claim?}).

Code that uses these streams usually calls \<close()> on only one
of the streams---either the wrapper stream or the wrapped stream.
A tool that reported an error for one stream if the other was closed
would report too many false positives to be usable in practice.
Prior work uses a pre-determined list of wrapper classes, built into
the tool, to handle this issue~\cite{TorlakC10}. \todo{Also cite ecj here?
  I think it also uses the same scheme...}

Our approach is more general. We introduce a new type qualifier,
\<@MustCallChoice>, to represent the wrapper relationship. \<@MustCallChoice>
qualifiers must always appear in pairs---one on a parameter of a method,
and another on its return type.

A pair of \<@MustCallChoice> annotations has two effects, in different
checkers:
\begin{itemize}
\item the Must Call Checker treats the annotated elements as polymorphic---the
  must-call obligations of the parameter are copied to the return type.
  Therefore, if the wrapped type has no must-call obligations (such as
  for a stream definitely wrapping a byte array), the wrapper will not
  have must-call obligations, either.
\item the Ownership Checker will treat the pair of streams as aliases,
  for the purposes of resolving must-call obligations. Therefore, calling
  \<close()> on one stream will be treated as calling \<close()> on both.
\end{itemize}

A pair of \<@MustCallChoice> annotations can be verified (though our current
implementation does not do so, since almost all appear in the standard
library rather than checked code) \todo{fix this!}. There are two
procedures to check
that a pair of \<@MustCallChoice> annotations on a method \<m>'s return type
and its parameter \<p>:
\begin{enumerate}
\item if \<p> is passed to another method or constructor in an
  \<@MustCallChoice> position, and \<m> returns the result of that method
  (or the method is a \<super()> constructor call appropriately annotated
  with \<@MustCallChoice>), then the code can be verified.
\item if \<p> is stored in an \<@Owning> field of the class, and the
  class declaration has an \<@MustCall(>$A$\<)> annotation for some set
  of methods $A$, and at least one of the methods in $A$ has an
  \<@EnsuresCalledMethods> annotation naming at least the
  must-call obligations of \<p>'s static type, then the code can be verified.
\end{enumerate}

Using \<@MustCallChoice> has another benefit: must-call obligations
can be shared even by objects that cannot possibly fulfill the
obligation. For example, consider the method
\<java.io.RandomAccessFile\#getFd()>, which returns a file descriptor
object for the file owned by the \<RandomAccessFile>. This file
descriptor cannot be closed directly---it has no \<close()> method.
However, it can be passed to a wrapper stream, such as
\<java.io.FileOutputStream>---which can be closed, fulfilling
the original must-call obligation. By creating a chain of \<@MustCallChoice>
annotations, we can verify code like the below:

\begin{lstlisting}[frame=tb,belowskip=3mm]
  RandomAccessFile file = new RandomAccessFile(myFile, "rws");
  FileInputStream in = null;
  try {
    in = new FileInputStream(file.getFD());
    // do something with in  
    in.close();
  } catch (IOException e){
    file.close();
  }
\end{lstlisting}


\section{Resource aliasing}
\label{sec:must-call-choice}

This section introduces a lightweight, specialized must-alias analysis
that tracks \emph{resource alias} sets---sets of pointers that
definitely correspond to the same underlying system resource.  Because
resource aliases correspond to the same underlying resource, closing
one alias also closes the other---so our sound resource alias analysis
permits \Tool to avoid issuing false positive warnings about resources
that have already been closed through their resource aliases.  Prior
work on accumulation analyses showed that limited, specialized forms
of alias analysis can provide precision to an accumulation analysis
where needed, without incurring the cost of a whole-program
analysis~\cite{KelloggRSSE2020}; our resource alias analysis is no
different.

\subsection{Wrapper types}

Precise leak detection for Java requires reasoning about \emph{wrapper types},
which wrap another type that may contain a resource.  For example, the Java
\<BufferedOutputStream> type adds buffering to some other \<OutputStream>, which
may or may not represent a resource that needs closing.  Wrapper types
introduce two additional complexities for \<@MustCall> checking:
\begin{enumerate}
  \item If a wrapped object has no \<@MustCall> obligation, the corresponding
  wrapper object should also have no obligation.
  \item Satisfying the obligation of \emph{either} the wrapped object or the
  wrapper object is sufficient.
\end{enumerate}
For example, if a \<BufferedOutputStream> $b$ wraps a stream with no underlying
resource (e.g., a \<ByteArrayOutputStream>), $b$ should have an empty
\<@MustCall> obligation, since $b$ has no resource of its own.  On the other
hand, if $b$ wraps a stream managing a resource, like a \<FileOutputStream> $f$,
then \<close()> must be invoked on \emph{either} $b$ or $f$. Calling \<close()> on $b$
is sufficient since $b$\<.close()> invokes \<close()> on its wrapped stream $f$.

Previous work has shown that a reasoning about wrapper types is
required to avoid excessive false positive and duplicate
reports~\cite{TorlakC10,ecj-resource-leak}.  Wrapper types in earlier
work were handled with hard-coded specifications of which library
types are wrappers and also, in the work of Torlak and
Chandra~\cite{TorlakC10}, a clustering technique to group warnings
involving a resource and its wrappers.

Our technique handles wrapper types more generally by tracking \emph{resource
aliases} during analysis.  Two references $r_1$ and $r_2$ are resource aliases
iff:
\begin{itemize}
\item satisfying $r_1$'s \<@MustCall> obligation also satisfies $r_2$'s
  obligation, and vice-versa; or
\item if $r_1$ and $r_2$ are must-aliased pointers.
\end{itemize}
We also introduce a new type
qualifier \mccannot to indicate where an API method creates a resource-alias
relationship between distinct objects, like a wrapper and the wrapped resource.
\mccannot can be used to annotate a \<BufferedOutputStream> constructor as follows:
\begin{lstlisting}
@MustCallAlias BufferedOutputStream(@MustCallAlias OutputStream arg0);
\end{lstlisting}

\mccannot qualifiers must always appear in pairs---one on a parameter
of a method, and another on its return type.  A pair of \mccannot
annotations has two effects on our inference and checking.  First,
must-call obligation inference (\Cref{sec:must-call}) treats the
annotated elements as polymorphic---the must-call obligations of the
parameter are copied to the return type. Therefore, if the wrapped
type has no must-call obligations (like a \<ByteArrayOutputStream>),
the wrapper will not have must-call obligations, either.

Second, at call sites, the Must Call Consistency Checker
(\Cref{sec:must-call-choice}) treats the \mccannot parameter and return as
aliases.  Specifically, the \textsc{CreatesAlias} routine in \cref{alg:helpers} is
modified as follows \todo{better notation!}:
\begin{algorithmic}
  \Procedure{CreatesAlias}{$P,s$}
    \State \Return $s$ is \lstinline{p = q} $\wedge\ q \in P$ \newline
         \hspace*{4.5em} $\vee$ $(s$ is \lstinline{p = foo(p1, p2, ...)} \newline
         \hspace*{6em} $\wedge\ \exists p_i \in P.\ q_i$ is
         \lstinline{@MustCallAlias} in \lstinline{foo})
    \EndProcedure
\end{algorithmic}
Beyond must-aliasing between local variables, the routine now treats the
assigned variable $p$ of a call statement as a new resource alias if a tracked
variable in $P$ is passed to some \mccannot parameter.
With this change, \Cref{alg:consistency-checker} will treat the case of a
resource alias exactly the same as a standard must alias.
% As shown in
% \cref{alg:consistency-checker}, the returned set is added to the set of tracked
% variables for a resource (\cref{li:compute-new-mc-aliases}), and the algorithm
% treats satisfaction of the \<@MustCall> obligation for any variable in the set
% as satisfying all the others.\todo{need another pass on this prose...}

\subsection{Beyond wrapper types}

\mccannot can also be employed in scenarios beyond direct wrapper types, a
capability not present in previous work.  In certain cases, a resource gets
shared between objects via an intermediate object that cannot directly close the
resource.  For example, \<java.io.RandomAccessFile> (which must be closed) has
a method \<getFd()> that returns a \<FileDescriptor>
object for the file. This file
descriptor cannot be closed directly---it has no \<close()> method.
However, the descriptor can be passed to a wrapper stream such as
\<FileOutputStream>, which if closed satisfies the original must-call
obligation.  By adding \mccannot annotations to the \<getFd()> method, our
technique can verify code like the below (from \todo{which case study?}):
\begin{lstlisting}[frame=tb,belowskip=3mm]
  RandomAccessFile file = new RandomAccessFile(myFile, "rws");
  FileInputStream in = null;
  try {
    in = new FileInputStream(file.getFD());
    // do something with in  
    in.close();
  } catch (IOException e){
    file.close();
  }
\end{lstlisting}
We assign \<FileDescriptor> a special \<@MustCall(?)> type that cannot be satisfied by any calls,
but which is converted back into \<@MustCall("close")> when passed to the \<@MustCallAlias> constructor
of \<FileInputStream>. We found this capability to be
useful for verifying multiple code patterns in our case studies.

\subsection{Verification of \mccannot}

A pair of \mccannot annotations on a method or constructor \<m>'s return type
and its parameter \<p> can be verified via one of two
procedures:
\begin{enumerate}
\item if \<p> is passed to another method or constructor in an
  \mccannot position, and \<m> returns the result of that method
  (or the method is a \<super()> constructor call appropriately annotated
  with \mccannot), then the code can be verified.
\item if \<p> is stored in an \<@Owning> field of the class, and the
  class declaration has an \<@MustCall(>$A$\<)> annotation for some set
  of methods $A$, and at least one of the methods in $A$ has an
  \<@EnsuresCalledMethods> annotation naming at least the
  must-call obligations of \<p>'s static type, then the code can be verified.
\end{enumerate}
These verification procedures permit a programmer to soundly specify a resource-aliasing
relationship in their own code, a capability that was not present in prior work
that relies on a hard-coded list of wrapper types, and that we used in our case studies
\todo{3 in Zookeeper + however many in the rest} times.

% As noted in previous work, p
% The
% must-call obligation for an arbitrary \<BufferedOutputStream> is
% \<@MustCall(\{``close''\})>---the stream might be writing to a file,
% for example.  \<BufferedOutputStream>'s constructors all have a
% parameter which is another \<OutputStream>. When \<close()> is called
% on the resulting \<BufferedOutputStream>, the implementation calls
% \<close()> on the wrapped \<OutputStream>. Since there are no
% additional resources allocated by the \<BufferedOutputStream> that
% were not part of the \<OutputStream> passed to the constructor,
% calling \<close()> on either stream is equally correct (though most
% Java style guides suggest calling \<close()> on the outermost
% wrapper \todo{cite that claim?}).

% Code that uses these streams usually calls \<close()> on only one
% of the streams---either the wrapper stream or the wrapped stream.
% A tool that reported an error for one stream if the other was closed
% would report too many false positives to be usable in practice.
% Prior work uses a pre-determined list of wrapper classes, built into
% the tool, to handle this issue~\cite{TorlakC10}. \todo{Also cite ecj here?
%   I think it also uses the same scheme...}

% LocalWords:  BufferedOutputStream OutputStream MustCall FileOutputStream
% LocalWords:  ByteArrayOutputStream MustCallAlias arg0 CreatesAlias getFd
% LocalWords:  FileDescriptor belowskip RandomAccessFile myFile rws getFD
% LocalWords:  FileInputStream EnsuresCalledMethods


\section{Accumulation frames}
\label{sec:reset-must-call}

This section describes an enhancement to the theory of accumulation
analysis that allows \Tool to verify non-final, owning fields
that contain one or more resources during their lifetime
(\cref{sec:non-final-owning}).
This approach is also useful for
removing another set of false positives caused by unconnected sockets
(\cref{sec:unconnected-sockets}).

\subsection{Non-final, owning fields}
\label{sec:non-final-owning}

The analyses we have described so far assume that a single program
element will control at most one resource while it is in scope:
the analyses always issue an error if a non-final field is assigned an expression
that controls a resource.
In practice, however, non-final fields
may be assigned resources. Consider
the following example:

\begin{lstlisting}[frame=tb,belowskip=3mm]
  @MustCall("close") // default qualifier for uses of SocketContainer
  class SocketContainer {
    private @Owning Socket sock;
    public SocketContainer() { sock = ...; } 
    void close() { sock.close() };
    void reconnect() {
      if (!sock.isClosed()) {
        sock.close();
      }
      sock = ...;
    }
  }
\end{lstlisting}
In the lifetime of a \<SocketContainer> object, it might control any
number of resources, depending on the number of calls to
\<reconnect()>. In order to verify the code above (and uses of the
class), we need \emph{accumulation frames}---distinct regions of the program in
which a program element can be treated as the subject of an
accumulation analysis. At the boundary between each frame, \Tool
1) checks that the obligations of the old frame are fulfilled\todo{What is
  this check?  How can it fail?}, and
2) resets the obligations of the object.

For example, consider the following use of the \<SocketContainer> class
above:

\begin{lstlisting}[frame=tb,belowskip=3mm]
  SocketContainer sc = ...;
  sc.close()
  // frame boundary
  sc.reconnect();
  sc.close();
\end{lstlisting}
This code contains two accumulation frames: the first extends from the
constructor call to the first call to \<reconnect()>, and the second
extends from the call to \<reconnect()> (which creates another
obligation) to the end of the block. Because \<reconnect()> is guaranteed
to clean up the resources that it overwrites---note the call
to \<close()> in its implementation---this code is safe, even though \<sc>
corresponds to two different resources: one for each frame.

We introduce a new annotation, called \ResetMustCall, to delineate
an accumulation frame. \ResetMustCall is written on a method
declaration, and takes one argument (the \emph{target}), which
indicates to which in-scope program element the frame applies. The
definition for \<reconnect()> above would then become:

\begin{lstlisting}[frame=tb,belowskip=3mm]
  class SocketContainer {
    ...
    @ResetMustCall("this")
    void reconnect() { ... }
  }
\end{lstlisting}
When an \ResetMustCall
method is invoked, \Tool removes any
inferred Called Methods information about the target (\<sc> in the example).
This rule means that even if \<close()> has already been called on \<sc>
(as in the example)
when \<reconnect()>
is called, \<close()> must be called again---corresponding to the intuition
that a new accumulation frame ``resets'' the obligations of \<sc>.
The inferred must-call methods for \<sc> could also increase:
if the declared type of \<sc>'s default must-call type includes any
methods that are not in the current must-call type for \<sc>, they are added.
Since \ResetMustCall annotations can only increase its target's obligations,
no verification for them is needed: placing them anywhere in the program is sound
(but may reduce precision).

The Must Call Consistency Checker treats calls to \ResetMustCall obligations
as creating a fresh obligation to check, since they mark the beginning of a new
accumulation frame.  In terms of our \cref{sec:must-call-invoked} pseudocode,
assume we have a routine \textsc{RMCTargets}$(c)$ that returns a set of all
variables $p_i$ passed to a \ResetMustCall target parameter of the method
invoked by $c$.  Then, we can update the \textsc{FactsFromCall} procedure of
\cref{alg:helpers} to handle \ResetMustCall as follows:
\begin{algorithmic}
  \Procedure{FactsFromCall}{$s$}
  \State $p \gets s.LHS, c \gets s.RHS$
  \State \Return $\{ \langle \{ p_i \}, c \rangle\ |\ p_i \in \textsc{RMCTargets}(c) \}$ \newline
  \hspace*{5em} $\cup\ (\textsc{HasObligation}(c)$ ? $\{ \langle \{ p \}, c \rangle \}$ : $\emptyset)$
  \EndProcedure
  \end{algorithmic}
Additionally, if during analysis, we encounter a call to a \ResetMustCall
method on a variable $p \in P$, we treat the \<@MustCall>
obligation of $p$ as \emph{satisfied}.  I.e., we add one more
case to the extended \textsc{MCSatisfiedAfter} predicate given in
\cref{sec:ownership-transfer} ($[\ldots]$ stands for the cases shown previously):
\begin{algorithmic}
  \Procedure{MCSatisfiedAfter}{$P,s$}
  \State \Return $\exists p \in P .\ [\ldots] \vee p \in \textsc{RMCTargets}(s)$
  \EndProcedure
\end{algorithmic}

% \begin{align*}
%   \exists p \in P &.\ [\ldots] \vee p \in \textsc{RMCTargets}(s)
% \end{align*}
This is sound since the checker creates a new obligation for calls to
\ResetMustCall methods, and the must-call obligation inference ensures the
\<@MustCall> type for the target will have a \emph{superset} of any methods present
before the call.
There is one exception to this consistency check: if an \ResetMustCall
method is invoked within a method that also has an \ResetMustCall annotation
with the same target---thus imposing the obligation on its caller---then
the new obligation can be treated as satisfied immediately.

Finally, the checker must enforce two new rules to ensure that
re-assignments to non-final, owning fields like \<sock> in the example
above are sound:
\begin{itemize}
\item any method that re-assigns a non-final, owning field of an object
  must be annotated with an \ResetMustCall that targets that object.
\item when a non-final, owning field $f$ is re-assigned at statement $s$, its inferred \<@MustCall> obligation
must be contained in its \<@CalledMethods> type at the program point before $s$.
\end{itemize}
\noindent
Together, these rules guarantee that any re-assignment to a non-final, owning
field (1) ends the current accumulation frame with the obligation fulfilled,
and (2) starts a new accumulation frame, with the obligation re-imposed on
the target at the caller.

\subsection{Unconnected sockets}
\label{sec:unconnected-sockets}
Accumulation frames can also handle cases where an object, when created,
does not actually allocate a resource---but it will allocate a resource
later in its lifecycle. Consider, for example, the no-argument constructor
to \<java.net.Socket>. This constructor does not actually allocate an
operating system-level socket, but instead just creates the container
object, which permits the programmer to e.g. set options which will be used
when creating the physical socket. When such a \<Socket> is created, it
initially has no must-call obligation; it is only when the \<Socket> is
actually connected\todo{How does that happen?} that the must-call obligation is created.

If all \<Socket>s are treated as \<@MustCall(\{"close"\})>---that is,
without the accumulation frames described in this section---a false positive
would be reported
in code such as the below, which operates on an unconnected socket
(simplified from real code in Apache Zookeeper~\todo{cite}):

\begin{lstlisting}[frame=tb,belowskip=3mm]
  static Socket createSocket() {
    Socket sock = new Socket();
    sock.setSoTimeout(...);
    return sock;
  }
\end{lstlisting}

The call to \<setSoTimeout> can throw a \<SocketException> if the
socket is actually connected when it is called, so without an
accumulation frame our analysis would report a possible leak along
that path.  However, we can use an accumulation frame \emph{without an
  obligation}---i.e. in which the \<Socket> is \<@MustCall({})>---
that extends from the constructor call to a method that connects the
socket (such as \<connect()> or \<bind()>) to avoid this false
positive, by annotating those methods (in the definition of
\<java.net.Socket>) as \ResetMustCall\<("this")>.  When one of these
methods is then called on the \<Socket> in client code that is being
verified by \Tool, its inferred must-call obligations increase to include
\<close()>, because by default an arbitrary \<Socket>'s type is
\<@MustCall("close")>.

% LocalWords:  belowskip MustCall SocketContainer sc ResetMustCall
% LocalWords:  RMCTargets MCObligations HasMCReturn MCSatisfied
% LocalWords:  createSocket setSoTimeout SocketException


\section{Implementation}
\label{sec:implementation}

We implemented \Tool on top of the Checker Framework~\cite{PapiACPE2008},
an industrial-strength framework for building pluggable type systems
for Java. The checkers which propagate and infer \MustCall and
\<@CalledMethods> annotations are implemented
directly as Checker Framework checkers.  The Must Call Consistency
Checker (\cref{alg:consistency-checker}) is
implemented as a post-analysis pass over the control flow graph
produced by the Checker Framework's dataflow analysis, which is invoked
when the other two checkers terminate. The framework provides the
checkers with flow-sensitive local type
inference, support for Java generics and qualifier polymorphism, and
other conveniences. Our implementation is open-source and available
at \todo{create an anonymous copy of the repo}.


\section{Evaluation}

% TODO: if this table is after tab:case-studies, then it doesn't appear? WTF

% a line in tab:ablation
% note that the FULL column is always zero, so it's not included here
% arguments: project name, NONE, LO-ONLY, AF-ONLY
\newcommand{\abltablerow}[4]{\textbf{\smaller{#1}} & #2 & #3 & #4 & 0}

\begin{table}
  \caption{An ablation study on the contribution of the lightweight
    ownership (\emph{NO-LO}), resource aliasing (\emph{NO-RA}),
    and accumulation frames (\emph{NO-AF})
    features in reducing false positives. Each entry is the number of extra
    false positive warnings reported by the variant with the given feature disabled on the given project.}
  \label{tab:ablation}
  
  \begin{tabularx}{\columnwidth}{@{}Xrrrr@{}}
    Project                              &      \emph{NO-LO} & \emph{NO-RA} & \emph{NO-AF} & \emph{FULL}      \\
    \hline
    \abltablerow{apache/zookeeper}              {?}            {?}             {?}                               \\
    \abltablerow{apache/hfds}                   {?}            {?}             {?}                               \\
    \abltablerow{apache/hbase}                  {?}            {?}             {?}                               \\
    \hline
    \abltablerow{\textbf{Total}}                {?}            {?}             {?}                               \\
  \end{tabularx}
\end{table}


% a line in tab:case-studies
% arguments: project name, original LoC, # of resources (-AcountMustCall), diff size, # of annotations, TPs, Confirmed TPs, FPs, run time in seconds
\newcommand{\osstablerow}[9]{\textbf{\smaller{#1}} & #2 & #3 & #4 & #5 & #6 & #7 & #8 & #9 s}

\begin{table*}
  \caption{Verifying the absence of resource leaks in case studies.
    Throughout, ``LoC'' is lines of non-comment, non-blank Java code.
    ``Resources'' is the number of resources created by the program.
    ``Diff size'' is the difference in LoC between the original and
    annotated programs, counting both annotations and modified code.
    ``Annos.'' is number of manually-written annotations to specify
    existing methods.
    ``TPs'' is true positives. ``Conf'' is confirmed true positives. 
    ``FPs'' is false positives, where the our analysis could not
  guarantee that the call was safe, but manual analysis revealed that no
  run-time failure was possible.
    ``RT(s)'' is the wall-clock run time of our analysis in seconds.}
  \label{tab:case-studies}

  \begin{tabular}{@{}lrr|rr|rrrr@{}}
    Project                              &      LoC      & Resources   &  Diff size  & Annos.   & TPs  & Conf    & FPs & RT(s)      \\
    \hline
    \osstablerow{apache/zookeeper:zookeeper-server}              {94,872}        {?}            {?}          {99}        {12}     {?}      {44}   {?}        \\
    \osstablerow{apache/hadoop:hdfs}                   {?}        {?}            {?}          {?}        {?}     {?}      {?}   {?}        \\
    \osstablerow{apache/hbase:?}                  {?}        {?}            {?}          {?}        {?}     {?}      {?}   {?}        \\
    \hline
    \osstablerow{\textbf{Total}}                {?}        {?}            {?}          {?}        {?}     {?}      {?}   {-}        \\
  \end{tabular}
\end{table*}


Our evaluation has three parts:
\begin{itemize}
\item case studies on open-source projects, which show that our approach
  scales to realistic programs and finds real bugs (\cref{sec:case-studies}).
\item an ablation study that demonstrates the contributions of
  lightweight ownership system (\cref{sec:lightweight-ownership}),
  resource aliasing (\cref{sec:resource-aliasing}), and
  accumulation frames (\cref{sec:reset-must-call}) to the false positive
  rate on the case studies in \cref{sec:case-studies} (\cref{sec:ablation}).
\item a comparison study that shows the advantages of our approach against
  two traditional approaches: heuristic bug-finders and heavy-weight
  typestate analysis (\cref{sec:compare}).
\end{itemize}

\subsection{Case studies on open-source projects}
\label{sec:case-studies}

We selected \todo{3} popular open-source projects with significant
usage of resources (and thus high danger of resource leaks), by
convenience.
For each project, we selected and analyzed one module
containing significant uses of leakable resources.
We modified the build system of each project to run our
analysis. We manually annotated each program with must-call,
called-methods, and ownership annotations. We also
made small changes to the programs, where possible, to avoid
common false positives of our analysis. We then examined all
remaining warnings, and categorized them as either true
positives---real resource leaks---or false positives---code that our
system was insufficiently precise to prove correct. We submitted bug
reports and patches to the projects describing true positives, when time permitted.
We also measured the number of possible resource leaks in each project
and the run time of our analysis using \todo{a standardized machine similar to the one used
  in the Grapple paper}. Each case study took one author a few hours;
most of the time was spent understanding the code under analysis, which no
authors were familiar with \emph{a priori}, writing corresponding annotations
where appropriate, and reasoning through warnings emitted by the tool by hand
to determine whether a warning was a true or false positive.

\Cref{tab:case-studies} summarizes the results. \tool finds multiple
serious resource leak bugs in all of the examined programs. Though
there are more false positives than true positives in each program,
the number is small enough to be examined by a single developer in a
few hours---a small price to pay for knowing that the program is
definitely free of resource leaks.  The annotations in the program are
also a benefit: as a form of machine-checked documentation, they
express the programmer's intent and, unlike traditional comments,
cannot become out-of-date if the checker is passing.

\todo{Describe some examples of true and false positives.}

\subsection{Ablating lightweight ownership, resource aliasing, and accumulation frames}
\label{sec:ablation}

Without lightweight ownership (\cref{sec:lightweight-ownership}),
resource aliasing (\cref{sec:must-call-choice}), and
accumulation frames (\cref{sec:reset-must-call}), our analysis produces significantly
more false positives. To show the contribution of each to our tool's precision,
we performed an ablation study on the same programs we used as case studies in
\cref{sec:case-studies} by building four versions of our tool:
one with lightweight ownership disabled (\emph{NO-LO}),
one with resource aliaisng disabled (\emph{NO-RA}),
and one with accumulation frames disabled(\emph{NO-AF}),
and the original with all three enabled (\emph{FULL}).
We re-ran the experiments in \cref{sec:case-studies} using each variant, and
recorded the change in the number of warnings. Since our analysis has no false
negatives, the difference between the total number of warnings (i.e. the sum of
the ``TPs'' and ``FPs'' columns in \cref{tab:case-studies}) produced by the \emph{FULL}
variant and the total number of false positives produced by each variant is
the number of false positives that are prevented by each feature.

\Cref{tab:ablation} has an entry for each variant on each case study project,
which is the number of extra warnings that that variant reported on the project,
compared to \emph{FULL}.

\subsection{Comparison to other tools}
\label{sec:compare}

Our approach represents a novel point in the design space of a resource leak checker.

We compared our approach with two other tools that purport to detect resource leaks:
\begin{itemize}
\item an analysis built into the Eclipse Compiler for Java (ecj).
\item Grapple~\cite{}, an unsound, typestate-based analysis that represents a significant, recent
  improvement in the scalability of typestate-based tools that require a whole-program alias analysis.
\end{itemize}

The Grapple paper already evaluated on earlier versions of \todo{some of?} the case study programs in \cref{sec:case-studies}.

We also ran our analysis and ecj's on these versions of the case study programs.

The results presumably show that our tool is faster than grapple and finds more bugs, but has more false positives.
Compared to ecj, we expect to be slower, find more bugs, and have similar numbers of false positives.

\todo{would we present these results as a table?}


\section{Limitations and threats to validity}
\label{sec:threats}

Like any tool that analyzes source code, \tool only
gives guarantees for code that it checks: the guarantee
excludes native code, the implementation of unchecked libraries (such as the JDK),
and code generated dynamically or by other annotation processors
such as Lombok.
%% MK: the sentence says that we can't guarantee that code generated by annotation
%% processors is checked, which is true, because they can misbehave in the way
%% that Lombok does. Most annotation processors don't, but the point is that
%% they can, so we can't promise that our analysis will be applied.
%% \todo{nit: is it annotation processors in general that cause an
%% issue, or Lombok because of the insane things it does?  I think for a typical
%% annotation processor doing code generation we shouldn't have problems?}
Though
the Checker Framework can handle 
reflection soundly~\cite{BarrosJMVDdAE2015}, by default (and in our case studies)
\tool compromises this guarantee
by assuming that objects returned by reflective invocations
do not carry must-call obligations.  (Users can customize this behavior.)
% Like any sound static analysis, \Tool does
% issue some false positives; these must be suppressed by
% the programmer and the correctness of the relevant code
% must be proved using another method.
Within the bounds
of a user-written warning suppression, \tool assumes that 1)
any errors issued can be ignored, and 2) all annotations
written by the programmer are correct.

The soundness of
\tool is with respect to specifications of which types of objects have a
\<@MustCall> obligation that must be satisfied.  We wrote such specifications
for the Java standard library, focusing on IO-related code in the \<java.io> and
\<java.nio> packages.  Any missing specifications of \<@MustCall> obligations
could lead \tool to miss resource leaks.

The results of our experiments may not generalize, compromising the
external validity of the experimental results.
% In particular, our subject programs are
% heavily-used, heavily-tested, and contain a high-density of resource
% usage---so, in a sense, they represent a worst-case scenario for
% \tool.  Nevertheless,
\Tool may produce more false positives, require
more annotations, or be more difficult to use if applied to other
programs.
Case studies on legacy code represents a worst case for a source code
analysis tool.  Using \tool from
the inception of a project would be easier, since programmers know their
intent as they write code and annotations could be written along with the
code.  It would also be more useful, since it would guide the programmers
to a better design that requires fewer annotations and has no resource leaks.
The need for annotations could be viewed as a limitation of our approach.
However, the annotations serve as concise documentation of
properties relevant to resource leaks---and unlike traditional, natural-language
documentation, machine-checked annotations cannot become out-of-date.

Like any practical system, it is possible that there might
be defects in the implementation of \tool or in the design of
its analyses. We have mitigated this threat with code review and an extensive
test suite for \tool:
% to collect these numbers, run these commands and sum the results:
% cd object-construction-checker/tests
% scc mustcall socket nolightweightownership mustcall-onlyjdk noresourcealias noaccumulationframes
% cd ../../must-call-checker/tests
% scc
119 test classes containing 3,776 lines of non-comment, non-blank code.
\todo{Update the numbers in the previous paragraph immediately before submission.}
This test suite is publicly available and distributed with \tool.

% LocalWords:  checkable


\section{Related Work}
\label{sec:relatedwork}

\subsection{Resource leak detection}
\label{sec:rw-resource-leaks}

\todo{Do a literature search on this topic, and add anything pertinent to
  the below.}

\todo{Re-organize and edit the below in a way that makes sense.}

\subsubsection{Static analysis}

Tracker~\cite{TorlakC10} uses a dataflow analysis over the control-flow
graph to reason about resource leaks, while avoiding the problem of
whole-program alias analysis using an access-path approach. Like our tool,
it scales to large Java programs; unlike \tool, however, it is unsound.

Grapple~\cite{zuo2019grapple} models alias and dataflow analyses as
dynamic transitive-closure computation of graph representations of
the target programs, enabling extraordinary precision with a whole-program
alias analysis. Compared to \tool, it is more precise but suffers
from unsoundness and an order of magnitude longer run times.
We compare directly to Grapple in \cref{sec:grapple}.

The Eclipse Compiler for Java includes a simple dataflow-based
bug-finder for resource leaks~\cite{ecj-resource-leak}. Their analysis
uses a fixed set of ownership heuristics and a fixed list of wrapper
classes; unlike \tool, it is also unsound. It is fast and
\todo{possibly} precise.  Similar analyses---with similar trade-offs
compared to \tool---are present in other heuristic bug-finding tools,
including SpotBugs~\cite{spotbugs-resource-leak},
PMD~\cite{pmd-resource-leak}, Infer~\cite{infer-resource-leak}, and others.
We compare to the Eclipse analysis directly in \cref{sec:eclipse}.

Weimer and Necula~\cite{WeimerN04} proposed a language feature called
a compensation stack to avoid resource leak problems in Java.

Relda and Relda2~\cite{guo2013characterizing,wu2016relda2} are unsound
resource-leak detection approaches that are augmented with call graphs
to model the Android framework's use of callbacks for releasing
resources.

The DroidLeaks benchmark~\cite{liu2019droidleaks} is a set of Android
apps with known resource leak bugs. Unfortunately, it includes only
the compiled apps, not the source code, so we were unable to run \tool
on its contents (\tool requires source code as its input).

The CLOSER~\cite{dillig2008closer} automatically inserts Java code to
dispose of resources when they are no longer ``live'' according to its
dataflow analysis, but requires an expensive alias analysis for
soundness, as well as manually-provided aliasing specifications for
linked libraries.

\todo{This~\cite{ghanavati2020memory} is a recent study on what kinds
  of resource and memory leak bugs happen in large Java projects. I'm
  sure we can use it somewhere.}

\subsubsection{Dynamic analysis}

Resco~\cite{dai2013resco} operates similarly to a garbage collector,
tracking resources whose program elements have become
unreachable. When a given resource (such as file descriptors) is close
to exhaustion, the runtime runs Resco to clean up any resources of
that type that are unreachable.  With a static approach such as ours,
a tool like Resco becomes unnecessary---leaks become impossible.

Automated test generation can also be used to try to detect resource
leaks. For example, leaks in Android applications can be found by
repeatedly running neutral---eventually returning to the same
state---GUI actions~\cite{wu2018sentinel,zhang2016automated}.
Other techniques focus on taking advantage of common misuse of
the Android activity lifecycle~\cite{amalfitano2020memories}.
Testing can only show the presence of bugs, not their absence;
\tool verifies that no resource leaks are present.

\subsection{Ownership tracking}

\todo{Manu: Describe relationship to rust's borrow checker.}

\todo{Is there other relevant related work? Find out.}

\subsection{Typestate analysis}
\label{sec:rw-typestate}

\tool relies on an accumulation
analysis~\cite{KelloggRSSE2020,FahndrichLeino03} to avoid the need
for a whole-program alias analysis, while retaining soundness.  We
build on the core analysis described in~\cite{KelloggRSSE2020},
adding the concept of accumulation frames.

Accumulation analysis is a special-case of typestate analysis~\cite{StromY86}.
In general, a typestate analysis requires a whole-program alias analysis
for soundness, making them impractical for resource leak detection
in industrial-size Java programs. \todo{However, people have probably tried
  so go figure that out.}

An alternative to applying a typestate analysis to an existing program
is to rewrite the program in a typestate-oriented programming
language~\cite{AldrichSSS2009}. \todo{Aldrich's group probably has a
  paper about doing resource leak detection in Plaid, so find and cite
  that here.}


\section{Conclusion}

We have developed a new, sound, modular approach for detecting and preventing
resource leaks in large-scale Java programs.  \Tool consists of sound core
analyses, built on the insight that leak checking is an accumulation problem,
augmented by three new features to handle common aliasing patterns: lightweight
ownership transfer, resource aliasing, and obligation creation by
non-constructor methods.

\Tool
discovered 49 resource leaks in heavily-used, heavily-tested Java code.
Its analysis speed is an order of magnitude faster
than whole-program analysis, and its false positive rate is similar to a
state-of-the-practice heuristic bug-finder.
It reads and verifies user-written specifications; the annotation burden
is about 1 annotation per 1,500 lines of code.

%% In future work, we plan to develop
%% improved inference techniques for lightweight ownership annotations.  Since
%% these annotations can be added anywhere without impacting soundness (they
%% are verified, not trusted), genetic
%% search
%% and machine-learning techniques could be used to introduce them, using the
%% warnings emitted by \tool as the fitness function.


\section*{Acknowledgments}
Thanks to Rashmi Mudduluru, Ben Kushigian, Chandrakana Nandi, and the anonymous
reviewers for their comments on earlier versions of this paper.  This research
was supported in part by the National Science Foundation under grants
CCF-2005889 and CCF-2007024, and also by a gift from Oracle Labs and a Google
Research Award.


%% The next two lines define the bibliography style to be used, and
%% the bibliography file.
\bibliographystyle{ACM-Reference-Format}
\balance
\bibliography{temp,bib/bibstring-abbrev,bib/types,bib/dispatch,bib/ernst,bib/soft-eng,bib/invariants,bib/crossrefs}

%%
%% If your work has an appendix, this is the place to put it.

\end{document}
\endinput
%%
%% End of file `sample-sigplan.tex'.

% LocalWords:  Kushigian Chandrakana Nandi LoC CCF
