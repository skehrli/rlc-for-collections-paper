% a line in tab:grapple
% these numbers come from tables 2 and 3 in the grapple paper from EuroSys '19.
% for TPs and FPs, I added together the IO and Socket columns of table 2.
%\newcommand{\grappletablerow}[4]{\textbf{\smaller{#1}} & #2 & #3 & #4}
\newcommand{\grappletableproject}[1]{\textbf{\smaller{#1}}}
% "ac" = "acronym character"
\newcommand{\ac}[1]{\textbf{\underline{#1}}}

\begin{table}
  \caption{Comparison of resource leak checking tools:  \ac{Ecl}ipse,
    \ac{Gr}apple, and the \ac{R}esource \ac{L}eak \ac{C}hecker.
    Different tools were run on different versions of the case study
    programs.
    As is standard,
    recall is the ratio of reported leaks to all leaks present in the code,
    and precision is the ratio of true positive warnings to all tool warnings.
    The number of leaks and the
    recall are computed over the code that is common to all versions of the
    programs, so recall is directly comparable within rows.
    Precision is computed over one version of the program.  It may not be
    directly comparable within rows; for example, different versions of ZooKeeper might
    contain different mixes of code patterns that lead to false positive warnings.}
  \todo{Clarify relationship of ``HDFS'' in this table to our corresponding
    benchmark.}
  \label{tab:tool-comparison}
  \posttablecaption
  \begin{tabular}{l|rccc|ccc}
                 &  & \multicolumn{3}{c|}{\textbf{Recall}} & \multicolumn{3}{c}{\textbf{Precision}*} \\
    Project & leaks & Ecl & Gr & RLC  & Ecl & Gr & RLC \\
    \hline
    \grappletableproject{ZooKeeper}      & 6  & XX\% & XX\% & XX\% & XX\% & XX\% & XX\% \\
    \grappletableproject{HDFS}           & 5  & XX\% & XX\% & XX\% & XX\% & XX\% & XX\% \\
    \grappletableproject{HBase}          & 7  & XX\% & XX\% & XX\% & XX\% & XX\% & XX\% \\
  \end{tabular}
\end{table}


% LocalWords:  HDFS rcccr TPs FPs ZooKeeper HBase
